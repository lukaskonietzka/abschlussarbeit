\chapter{Ausblick}
\label{chap:schlussfolgerung} Der Tooth Analyser bietet mit dem aktuellen Stand bereits
einen großen Mehrwert für die Ärzte in der Klinik für Zahnerhaltung. So lässt
sich sagen, dass diese vorliegende Arbeit durchaus von Erfolg gekrönt ist. Es
wird aber auch klar, dass noch viel Potenzial im Tooth Analyser steckt. Das
meiste Potenzial ist hierbei in den Limitierungen zu finden. Eine hervorragende Ergänzung
dieser Arbeit würde eine Vorverarbeitung der Bilder bieten. Es soll also in
Zukunft möglich sein, die \ac{CT}-Aufnahmen in einer Vorverarbeitung zu
komprimieren und sie dann mittels eines Verfahren zu bearbeiten. Hierbei muss sich
die Vorverarbeitung nicht auf die Komprimierung beschränken, wie die Diskusion
gezigt hat. Dies schließt auch eine Verarbeitung aller Formate ein. Ein schöner
Nebeneffekt dieser Ergänzung ist, dass durch eine Komprimierung der Bilder die Laufzeit
sinkt. Der verlorene Detailgrad dieser Komprimierung ist dabei nicht störend. Somit
lässt sich sagen, dass durch das Akzeptieren unterschiedlicher Formate der
größte Mehrwert für den Tooth Analyser gewonnen werden kann. Des Weiteren lässt der
Bereich Analysen im Tooth Analyser noch viel Spielraum. Hier gibt es diverse
Möglichkeiten, das Modul noch mit weiteren Funktionen zu bestücken. Eine gute
Idee ist hier eine Integration des Python-Pakets \textit{radiomics}. Damit
lassen sich Radiomics-Merkmale aus Bilder Extrahieren und isoliert analysieren. Den
letzten Ausblick, der hier gegeben werden soll, bezieht sich wieder auf das Verfahren
der anatomischen Segmentierung. In der aktuellen Version nimmt das Verfahren nur
eine Aufteilung in Schmelz und Dentin vor. Der Teil der Pulpa wird dem Hintergrund
zugeordnet und nicht betrachtet. Die zusätzliche Segmentierung der Pulpa würde
die Arbeit ebenfalls hervorragend ergänzen. Jedoch sei auch gesagt, das dies die
wohl herausforderndste Aufgabe ist. Die Pulpa hebt sich nur sehr leicht aus dem Hintergrund
hervor und ist demnach schwer zu segmentieren.

Ordnet man die hier genannten Punkte nach ihrer Wichtigkeit ein, so lässt sich
sagen, das durch das Erweitern des Moduls auf mehrere Bildformate der größte Mehrwert
für den Tooth Analyser gewonnen werden kann. Dies würde weitere Funktionalität
nachsichziehen. Jedoch sei auch gesagt, dass die übrigen Punkte ebenfalls einen
großen Mehrwert für den Tooth Analyser und damit für die praktizierenden Ärzte
am Klinikum für Zahnerhaltung in München bietet.
% ---------------------------------------------------------------------------------------