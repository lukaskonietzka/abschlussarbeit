\chapter{Ergebnisse}
\label{chap:ergebnisse} Dieses Kapitel präsentiert alle Ergebnisse, die in
dieser Arbeit erzielt wurde. Dabei spielen nicht nur die erfolgreichen Ziele eine
Rolle, sondern auch die Misserfolge. Zunächst wird die entwickelte Erweiterung in
seiner finalen Form beschrieben, gefolgt von einer Darstellung der zentralen
Funktionen. Anschließend wird auf die Konzeptionen und Umsetzungen der verschiedenen
Teile eingegangen. Abschließend soll die Performance und die verschiedenen
Anwendungsszenarien genauer analysiert werden. Daraus ergeben sich auch Limitierungen
für die Software. Mit diesen erstellten Analysen kann unter Berücksichtigung
eine Aussage bezogen auf die Forschungsfrage gestellt werden.
% ---------------------------------------------------------------------------------------

\section{Tooth Analyser}
Im Rahmen dieser vorliegenden Arbeit ist eine 3D Slicer Extension entstanden, die
den Namen Tooth Analyser trägt und für die Forschung im Dentalbereich eingesetzt
wird. In erster Linie können mit diesem Plugin Micro CT Aufnahmen anatomisch
segmentiert werden. Das Modul schmiegt sich wie alle anderen Module gut in die Kernanwendung
ein und bietet eine \ac{UI}. Neben der eigentlichen Implementierung ist auch ein
Logo für das Plugin entstanden, das es nach außen repräsentiert. Die Abbildung \ref{fig:logo_tooth_analyser}
zeigt dies.

\begin{figure}[h]
	\centering
	\includegraphics[width=0.9\textwidth]{img/SlicerToothAnalyser.png}
	\caption{Logo der 3D Slicer Erweiterung "Tooth Analyser", welche im Rahmen dieser
	Arbeit entwickelt wurde. Logodesign: Dr. Elias Walter}
	\label{fig:logo_tooth_analyser}
\end{figure}

Des Emblem des Tooth Analyser bildet einen Zahn ab, dessen Hauptsegmenten (Schmelz,
Dentin, Pulpa) mit den unterschiedlichen Farben (grün, gelb, orange) visualisiert
werden. Dies verdeutlicht die Analogie zur anatomischen Segmentierung und lässt
gleich vermuten, dass sich dieses Modul mit einer Segmentierung beschäftigt. Der
Untertitel des Logos lässt darauf deuten, dass es um die Segmentierung von Micro
\ac{CT} Aufnahmen geht. Wurde der Tooth Analyser installiert, so ist er über den
Menüpunkt Module in Slicer auswählbar. Hier wird er in dem Unterpunkt \textit{Segmentierung}
eingruppiert, was ein weiteres Indiz auf die grobe Funktionalität liefert. Wird
also der Tooth Analyser gestartet so erhält man die Ansicht der Kernanwendung
mit der Entsprechenden \ac{UI}. Die Abbildung \ref{fig:tooth_analyser_start_up}
soll genau diese Ansicht verdeutlichen

\begin{figure}[h]
	\centering
	\includegraphics[scale=0.2, width=\textwidth]{img/toothAnalyserStarUp.png}
	\caption{Startansicht der Erweiterung Tooth Analyser nach dem ersten Aufruf}
	\label{fig:tooth_analyser_start_up}
\end{figure}

Die Ansicht zeigt die Kernanwendung (rechts die verschiedenen Fenster) und die
\ac{UI} des jeweiligen Moduls. Die Kernanwendung kann auch als Szene beschrieben
werden und übernimmt alle generischen Handhabungen der Bilder. Neben den Szenen ist
auch immer eine Sidebar zu sehen, welche die \ac{UI} des jeweiligen Moduls abbildet.
Im Falle der Abbildung \ref{fig:tooth_analyser_start_up} ist es die \ac{UI} des Tooth
Analyser. Das manuelle Laden eines Bildes in die Szene ist Teil der Slicer
Kernanwendung und nicht teil der Modullogik. Das bereits geladene Bild ist demnach
unabhängig von der Slicer Erweiterung entstanden. Betrachtet man die
Benutzerschnittstelle genauer, so fällt sofort auf, dass diese in vier Bereiche
unterteilt ist Diese Aufteilung in Bereiche ist ein Ergebnis der
Literaturrecherche und eine gute Konvention ind er Welt von 3D Slicer. Der Bereich
\textit{Help and Acknowledgement} stellt Hilfen und Informationen über das Modul
bereit. Über diesen Abschnitt ist auch die offizielle Dokumentation über dieses Modul
erreichbar. Zu Beachten ist, dass dieser Bereich nicht eigens für den Tooth Analyser
entwickelt wurde. Es handelt sich hier um eine Funktionalität, die automatisch allen
\ac{SEM} zur Verfügung steht. Bei den übrigen Abschnitten handelt es sich im Features
die spezifische für den Tooth Analyser entwickelt wurden. Bevor genauer auf die
Funktionalitäten des Tooth Analyser eingegangen wird, sei zunächst auf die Abbildung
... verwiesen, welche die Kernfunktionalitäten zeigt.

\begin{figure}[h]
	\centering
	\includegraphics[scale=0.2, width=\textwidth]{img/toothAnalyserFullView.png}
	\caption{Ergebnisansicht der Erweiterung Tooth Analyser nachdem die Analysen
	und die anatomische Segmentierung erstellt wurden.}
	\label{fig:tooth_analyser_full_view}
\end{figure}

Der Analysebereich des Tooth Analyser ermöglicht es das Histogramm eines gegebenen
Bildes zu erstellen. Dies ist besonders interessant, wenn ein Algorithmus für die
anatomische Segmentierung ausgewählt werden muss. Diese Algorithmen sind
Schwellwertverfahren, die auf das Histogramm eines Bildes basieren, um es zu
segmentieren. Das erstellte Histogramm ist rechts oben in der Abbildung \ref{fig:tooth_analyser_full_view}
zu erkennen. Es wird in einem Plot-Node dargestellt und kann über diesen auch
verändert werden. Hierzu ist die Pinnnadel im Fenster des Plot-Node zu wählen.
Durch die Speicherfunktion der Kernanwendung kann der Plot auch problemlos
gespeichert werden. Bevor die Analysen erstellt werden können, müssen
Parametereinstellungen gewählt werden. Hierbei ist der wichtigste Parameter der,
indem das konkrete Bild ausgewählt wird. Bei diesem Parameter handelt es sich um
ein Dropdown, indem nur Bilder mit dem Typ \texttt{vtkMRMLScalarVolumeNode} ausgewählt
werden können. Dies trägt zur Stabilität und Ausfallsicherheit des Systems bei
und sorgt dafür, dass nicht jedes beliebige Bild geladen werden kann. Ist keine
\ac{CT} Aufnahme ausgewählt, so bleibt der Button zum Starten der Analysen
deaktiviert. Wird ein Bild in die Szene geladen, während der Parameter für das
zu analysierende Bild leer ist, wählt der Tooth Analyser automatisch das Bild aus,
dass als Erstes in die Szene geladen wurde. So spart der Benutzer einige Klicks.
Durch die Checkbox \textit{Show Histogram} wird der Erweiterung signalisiert das
beim Starten der Analysen ein Histogramm des übergebenen Bildes erstellt weden soll.

Die Hauptfunktionalität des Tooth Analyser ist die anatomische Segmentierung welche
in Kapitel \ref{sec:verwwandte_arbeit} detaliert erläutert wurde. Die konkreten
Ergebnisse dieser Segmentierung sind in der Abbildung
\ref{fig:tooth_analyser_full_view} in den Fenstern (blau, rot, grün, gelb) zu sehen.
Neben der eigentlichen Segmentierung sind auch hier die medialen Flächen für die
Segmente Dentin (rot) und Schmelz (grün) gut sichtbar. Hinzu kommt ein \ac{3D}
Modell das auf basis der erstellten Segmentierung generiert wurde und nur der
Visualisierung dient. Ein Abspeichern dieses 3D Modells als Netz ist nicht möglich.
Um überhaupt eine anatomische Segmentierung eines Zahnes erstellen zu können, sieht
der Algorithmus zunächst drei Parameter vor, die eingestellt werden müssen. Um die
Komplexität gering zu halten, wurde bewusst auf viele Parameter verzichtet. Ähnlich
wie bei den Analysen ist auch hier die Wahl des zu segmentierenden Volumens der entscheidende
Parameter. Dessen Bedeutung gleicht der der Analysen, insbesondere in Bezug auf das
Verhalten. Damit schnell ein Ergebnis genneriert werden kann, wurde für die übrigen
zwei Parameter ein Vorauswhal definiert, die zum vollen Ergebnisumfang des Tools
führt. So ergibt sich die Situation, das nach dem Laden eines \ac{CT}s in die
Szene nur auf den Button für das Ausführen gedrückt werden muss, damit eine anatomische
Segmentierung erstellt wird. Dies nimmt dem Benutzer viel Arbeit ab und sorgt
für eine gut \ac{UX}. Sind jedoch Einstellungen in de Parametern gewünscht, so können
diese natürlich getätigt werden. Über den Parameter \textit{Segmentation
algorithm} kann das entsprechenden Schwellwertverfahren gewählt werden, mit dem
der Zahn segmentiert werden soll. Dies mag nur geringfügig eine Änderung auf die
Ergebnismenge ausmachen, kann aber dennoch wichtig sein. Die Checkboc \textit{calculate
medial surface} ermöglicht eine optinale Erstellung der medialen Flächen. Wird diese
Funktion ohnehin nicht gebraucht, so kann diese hier ausgelassen werden und
damit Laufzeit eingespart werden.

Die letzte Funktionalität, die geboten wird stellt kein neues Verfahren da,
sondern nur eine andere Art der Ausführung. Der letzte Abschnitt in der \ac{UI}
beschäfftigt sich mit dem Batch Verfahren für die Anatomische Segmentierung.
% ---------------------------------------------------------------------------------------

\section{Konzeptionen}
- Wireframes

- Klassendiagramme
% ---------------------------------------------------------------------------------------

\section{Technische Umsetzung}
- ichtige Codeausschnitte

- Das logic interface

- die aufteilung in Module
% ---------------------------------------------------------------------------------------

\section{Performance}
Die Laufzeitanalyse des Systems.
% ---------------------------------------------------------------------------------------

\section{Anwendungsszenarien}
Wie wird das tool eingesetzt, wer setzt es ein, Segmentierung des gazen Zahnes ist
auch möglich.

Um den Nutzen dieser Erweiterung noch weiter zu erhöhen, können die
unterschiedlichen Segmente über das Modul \textit{Data} unsichtbar geschalten werden.
So lässt sich der Fokus auf einzelne Teile lenken

- Segmentieren

- überlappen der asnzeige mit Medialflächen

% ---------------------------------------------------------------------------------------

\section{Limitierungen}

- kleine bilder

- preprocessing
% ---------------------------------------------------------------------------------------