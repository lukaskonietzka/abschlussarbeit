\chapter{Methodik}
\label{chap:methodik} Wie das vorherige Kapitel bereits eingeleitet hat, soll es
hier um das \textit{Wie} gehen. Es wird also aufgezeigt, welches methodische Herangehen
an die Fragestellung verfolgt wurden, um ein aussagekräftiges Ergebnis zu
erzielen. Dabei wurde mit einer umfangreichen Anforderungsanalyse gestartet,
welche die Domäne und die Ausgangslage klären soll. Sind diese Konzepte klar, so
wird die endgültige Problemstellung in mehrere kleine Teilaufgaben zerlegt. Für jede
dieser Teilaufgaben wurde dann eine Recherche zum Stand der Technik durchgeführt,
um bereits existierenden Lösungen ausfindig zu machen. Sofern es für eine Teilaufgabe
noch keine Lösung gibt, werden anschließend konkrete Lösungsansätzen für die
einzelnen Teilprobleme erarbeitet. Sollte für eine Aufgabe mehrere Ansätze existieren,
so werden diese im letzten Abschnitt miteinander verglichen und ein passender
Ansatz gewählt.

Bei der Durchführung dieser Schritte zum erreichen eines Ergebnisses, soll der in
der Softwareentwicklung allgemeine bekannte Ansatzt \textit{make it run, make it
right, make it fast} verfolgt werden. Dieser beschreibt, dass zunächst dafür
gesorgt werden soll, dass ein Problem überhaupt gelöst wird. Im Anschluss soll
es so umgebaut werdem dass eine Wartbarkeit und Erweiterbarkeit entsteht. Erst an
aller letzter Stellte steht die Performance. QUELLE

\section{Anforderungsanalyse}
\label{sec:anforderungsanalyse} Nach genauerem Betrachten der Fragestellung aus Kapitel
\ref{chap:fragestellung} wird klar, dass im Rahmen dieser vorliegenden Arbeit
eine Extension für die Plattform 3D Slicer entwickelt werden soll. Diese Erweiterung
beinhaltet das Segmentierungsverfahren nach Hoffmann \citep[vgl.][]{hoffmann2020},
wie es in Kapitel \ref{sec:verwwandte_arbeit} beschrieben wurde. Das Verfahren
segmentiert Micro-CT Aufnahmen der Zahnklinik in München und wird zu Forschung
eingesetzt. Da Ärzte keine Softwareentwickler sind, ist es wichtig, dass das Verfahren
eine UI erhält die eingängig und übersichtlich ist. Außerdem ist eine stabile
Anwendung gefragt, die sich gut in die Kernanwendung von 3D Slicer einfügt. Für
einen Überblick über die wichtigsten Eigenschaften von 3D Slicer sei auf das
Kapitel \ref{sec:3d_slicer} verwiesen.

Die Extension selber soll neben einer Einzelbildbearbeitung auch einen Batch-Prozess
ermöglichen. So können Beispielsweise Parameter an einem Bild erprobt werden und
diese anschließend in eine Batch-Prozess für viele Bilder überführt werden.
Außerdem soll es möglich sein, verschiedenen Segmentierungsverfahren, die in
Hoffman vorgesehen sind, auch in der Extention auszuwählen.

Ein wichtiger Softwartechnischer Anspruch an die Extension ist die Erweiterbarkeit.
Es soll ohne große Mühen möglich sein, ein weiteres Verfahren zu implementieren ohne
das große Anpassungen an der UI oder der Erweiterung selbst, unternommen werden
müssen. Für die Architektur und die Bentzung dieser Software soll es selbstverständlich
eine Dokumentation mit Benutzerhandbuch geben. Auch das Testen der Software soll
natürlich nicht zu kurz kommen, weswegen ein Reihe an Unit-Tests (Das Testen
eine konkreten Programmeinheit) vorgesehen ist

Diese doch breite Palette an Anforderungen, lässt sich unmöglich auf einmal
bearbeiten. Hierzu sieht diese Arbeit eine Aufteilung in Teilprobleme vor. Der
nächste Abschnitt Blickt auf die herausgearbeiteten Anforderungen in diesem
Kapitel und leitet daraus Teilprobleme ab.

\section{Zerlegung in Teilprobleme}

\textbf{Architekturdesign}

\textbf{Leere Extension}

\textbf{Pseude Extension}

\textbf{Modularisierung Hoffmann}

\textbf{Erstellung UI}

\section{Recherche zum Stand der Kunst}

\section{Erarbeiten von Lösungsansätzen}

\section{Auswahl von Lösungsansätzen}