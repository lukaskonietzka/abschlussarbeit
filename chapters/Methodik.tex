\chapter{Methodik}
\label{chap:methodik} Dieses Kapitel zeigt auf, welches methodische Herangehen
an die Fragestellung verfolgt wurden, um ein aussagekräftiges Ergebnis zu erzielen.
Dabei wurde mit einer umfangreichen Anforderungsanalyse gestartet, welche die Domäne
und die Ausgangslage klären soll. Sind diese Konzepte klar, so wird die endgültige
Problemstellung in mehrere kleine Teilaufgaben zerlegt. Für jede dieser
Teilaufgaben wurde dann eine Recherche zum Stand der Technik durchgeführt, um bereits
existierenden Lösungen ausfindig zu machen. Sofern es für eine Teilaufgabe noch
keine Lösung gibt, werden anschließend konkrete Lösungsansätzen für die einzelnen
Teilprobleme erarbeitet. Sollte für eine Aufgabe mehrere Ansätze existieren, so werden
diese im letzten Abschnitt miteinander verglichen und ein passender Ansatz gewählt.

Bei der Durchführung dieser Schritte zum Erreichen eines Ergebnisses, soll der
in der Softwareentwicklung allgemeine bekannte Ansatz \textit{make it run, make it
right, make it fast} verfolgt werden. Dieser beschreibt, dass zunächst dafür gesorgt
werden soll, dass ein Problem überhaupt gelöst wird, bevor man viel Zeit in ein Refactoing
oder eine Optimierung steckt. QUELLE
% ---------------------------------------------------------------------------------------

\section{Anforderungsanalyse}
\label{sec:anforderungsanalyse} Nach genauerem Betrachten der Fragestellung aus Kapitel
\ref{chap:fragestellung} und den Zielen aus \ref{sec:ziel_der_arbeit} können bereits
einige Anforderungen abgeleitet werden, die für die Erweiterung gelten sollen. Neben
den Anforderungen aus der Fragestellung und den konkreten Zielsetzung, wurde
auch die Klinik für Zahnerhaltung mit in diesen Prozess eingebunden. Hierzu wurde
inerhalb eines Meetings mit dem Verantwortlichen Arzt ,Dr. Elias Walter, ein
Anforderungskatalog ausgearbeitet. Diese Anforderungen waren vorallem zu Beginn der
Entwicklung sehr wichtig um einen ersten Anhhaltspunkt zu gewinnen. Im Laufe des
Entwicklungsprozesses wurden Statusberichte eingeplant, die ein reagieren auf Anforderungsänderungen
ermöglichen sollen.

In erster Linie wird klar, dass im Rahme dieser vorliegenden Arbeit eine
Extension für die Plattform 3D Slicer entwickelt werden soll. Diese Erweiterung soll
die anatomische Segmentierung nach Hoffmann \citep[vgl.][]{hoffmann2020}
beinhalten, wie sie in Kapitel \ref{sec:verwwandte_arbeit} beschrieben wurde.
Greift man das Ziel dieser Arbeit aus der Einleitung \ref{sec:ziel_der_arbeit}
nochmals auf, dann kann hierdurch die nächste wichtige Anforderung abgeleitet werden.
Die Erweiterung soll gut und einfach über ein User Interface (UI) bedient werden
können. Außerdem ist eine stabile Anwendung gefragt, die sich gut in die
Kernanwendung von 3D Slicer einfügt. Die Extension selber soll neben einer Einzelbildbearbeitung
auch einen Batch-Prozess ermöglichen. So können Beispielsweise Parameter an
einem Bild erprobt werden und diese anschließend in einen Batch-Prozess für
viele Bilder überführt werden. Außerdem soll es möglich sein, verschiedenen
Schwellwertverfahren, die in der anatomischen Segmentierung vorgesehen sind,
auch in der Extention auszuwählen. Ein wichtiger Softwaretechnischer Anspruch an
die Extension ist die Erweiterbarkeit. Es soll ohne große Mühen möglich sein, ein
weiteres Verfahren zu integrieren, ohne das große Anpassungen an der UI oder der
Erweiterung selbst, unternommen werden müssen. Für ein solides Verständnis
dieser Software soll es selbstverständlich eine Dokumentation mit
Benutzerhandbuch geben. Zudem wird großer Wert auf die Qualitätssicherung gelegt,
weshalb eine Reihe von Unit-Tests (Tests für einzelne Programmeinheiten)
vorgesehen ist.

Um die Anforderungen an die Software besser zu verstehen und zu strukturieren, ist
neben der Sammlung technischer Spezifikationen auch ein solides Verständnis für die
zugrunde liegende Domäne essenziell. Die Abbildung \ref{fig:3d_slicer_domäne}
veranschaulicht dies durch ein UML-Domänenmodell (Unified Modeling Language), das
einen visuellen Überblick über die verschiedenen Teile der Software bietet. Dazu
sind auch einige der Anforderungen erkennbar.

\begin{figure}[h]
	\centering
	\includegraphics[width=0.8\textwidth]{img/domaenenmodell.jpg}
	\caption{UML-Domänenmodell des gesamten Softwaresystems}
	\label{fig:3d_slicer_domäne}
\end{figure}

Diese doch breite Palette an Anforderungen lässt sich unmöglich auf einmal bearbeiten.
Auch durch eine visuelle Darstellung kann dies nicht vereinfacht werden. Hierzu sieht
diese Arbeit eine Aufteilung in Teilprobleme vor. Der nächste Abschnitt blickt
auf die herausgearbeiteten Anforderungen in diesem Kapitel und leitet daraus
Teilprobleme ab.
% ---------------------------------------------------------------------------------------

Hier das Herrangehen an die Anforderungen beschreiben und noch in der
Anforderungsanalyse erwähnen, dass es zu anforderungen an die software kommt,
ganz am anfang.

\section{Zerlegung in Teilprobleme}
\label{sec_zerlegung_in_teilprobleme} Durch die Aufteilung des Gesamtsystems in
mehrere kleine Teilaufgaben wird die Software für den Entwicklungsprozess
übersichtlicher. Die einzelnen Domänen können so schneller und besser verstanden
werden. Es gibt viele Möglichkeiten ein Softwaresystem in kleine Teile
aufzuteilen, sodass es am Ende auf den konkreten Anwendungsfall ankommt. Diese
Arbeit sieht folgenden Teilaufgaben für das Gesamtsystem vor:

\begin{itemize}
	\item \textbf{Architekturdesign:} Mithilfe von UML Diagrammen soll die
		Architektur dieses Systems abgebildet werden und sukzessive immer
		detaillierter beschrieben werden. Es soll dann verglichen werden, welche
		Softwarepatterns für dieses System infrage kommen. Durch die Bearbeitung dieses
		Teilproblems kann die Anforderung an eine flexible Architektur erfüllt werden.
		Anschließend kann mit der Entwicklung des UI-Designs begonnen werden.

	\item \textbf{UI Design:} Es soll ein Design erstellt werden, dass sich an erfolgreichen
		und etablierten 3D Slicer Extensions orientiert. Jedoch sollen die Wünsche
		des Endnutzers auch nicht zu kurz kommen. Für eine Visualisierung des Designs
		bedient sich diese Arbeit der Wireframes.

	\item \textbf{Pseudo-Extension:} Bevor der tatsächliche Algorithmus
		eingebunden werden kann, ist es wichtig eine funktionierende Erweiterung zu haben,
		die noch keine konkrete Aufgabe hat, aber funktioniert und in Slicer eingebunden
		werden kann.

	\item \textbf{Hilfsfunktionen:} Nachdem die Infrastruktur der Erweiterung
		steht und funktioniert, kann mit der Implementierung einiger Hilfsfunktionen
		begonnen werden. Hierbei handelt es sich um Methode, die nicht direkt etwas mit
		dem Verfahren zu tun haben, jedoch kleine Nebenaufgaben erfüllen und so
		unumgänglich sind. Als Beispiel sei hier das Laden von CT-Bildern in die Szene
		gedacht.

	\item \textbf{Kappselung Hoffmann:} Nachdem die leere Extension lauffähig ist
		und auch einige Hilfsfunktionen bereitstehen, kann mit der Paketerstellung des
		Hoffmann begonnen werden. Hier soll das Verfahren von einem Python Notebook
		in eine Bibliothek überführt werden, sodass dieses Verfahren in der Extention
		ausführbar ist. Die konkrete Art des Paketes ist noch nicht festgelegt.

	\item \textbf{Speicherung der Parameter:} Der Benutzer steuert das Verfahren
		über die Parameter in der UI. Für die Speicherung der Parametereinstellungen
		hat Slicer den Mechanismus ParameterNode entworfen. Diese wurde bereits in
		Abschnitt \ref{subsec:benutzerschnitstelle} erwähnt. Dieser Mechanismus ist nicht
		trivial, erhöht die Benutzerfreundlichkeit des Systems aber erheblich und
		soll demnach auch in diese Extention Anwendung finden.

	\item \textbf{Single Prozess:} Sobald alle notwendigen Vorbereitungen
		getroffen sind, kann der Algorithmus nun eingebettet werden. Hierzu
		betrachtet man isoliert den Single Prozess. Auch die UI wird erst nur so weit
		entwickel, wie es für den einfachen Prozess nötig ist. Hierbei wird auf das erstellte
		Paket für das Hoffmann Verfahren und die zuvor erstellen Hilfsfunktionen zurückgegriffen.

	\item \textbf{Batch Prozess:} Ist das einfache Verfahren fertig implementiert
		und funktioniert, so kann der Batch Prozess hinzukommen. Hier bedarf es
		einer zusätzlichen Arbeit in der UI, da der Benutzer über das Verwenden dieser
		Funktion gewarnt werden muss. Der Batch Prozess bedarf nämlich erheblicher
		Ressourcen. Hinzukommt die Implementierung einer Fortschrittsanzeige, sodass
		zu erkennen ist, dass ein Hintergrundprozess läuft.

	\item \textbf{Dokumentation und Benutzerhandbuch:} Abschließend ist eine
		ausführliche Dokumentation der Architektur erwünscht, sodass zukünftige Entwickler
		wissen, wo sie ansetzten müssen. Hinzu kommt ein Benutzerhandbuch für eine Verwendung
		der Erweiterung. Das Benutzerhandbuch und die Architekturdokumentation
		erfolgen in einer README.md innerhalb der Extension.

	\item \textbf{Tests:} An letzter Stelle sollen noch Softwaretests
		implementiert werde, um die Richtigkeit der Extension sicherzustellen. 3D
		Slicer sieht hier Unittests vor, die über den Developer Modus in Slicer direkt
		in der jeweiligen Extension ausgeführt werden können.
\end{itemize}

Die Ordnung dieser Punkte gibt eine grobe Orientierung bezüglich der Reihenfolge
während der Umsetzung an. Bei der Bearbeitung der einzelnen Teilaufgaben ist es
auch wichtig eine gute Recherche zum aktuellen Stand der Technik durchzuführen.
% ---------------------------------------------------------------------------------------

\section{Recherche zum Stand der Kunst}
Es ist sehr ungünstig, wenn sich zu Ende eines Projektes herausstellt, dass
Lösungen, in die erhebliche Ressourcen investiert wurden, bereits veröffentlicht
sind. Um dies zu vermeiden, ist eine Umfassende Literaturrechre nötig, welche den
aktuellen Stand der Technik repräsentieren soll. Hierbei wurde auf diverse
Literatur zurückgegriffen. Eine ganz wesentliche Quelle liefert die offizielle Dokumentation
von 3D Slicer. Diese beinhaltete gute Anhaltspunkte und bewährte
Implementierungsbeispiele. Des Weiteren werden bereits existierende Erweiterungen
heruntergeladen und näher betrachtet. Für jeden Teilproblem, was im vorherigen
Kapitel definiert wurde, wird dann ein potentieller Lösungsansatz recherchiert.

\section{Werkzeuge}
TODO