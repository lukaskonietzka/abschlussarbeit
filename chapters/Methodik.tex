\chapter{Methodik}
\label{chap:methodik} Dieses Kapitel beschreibt das methodische Vorgehen, das
zur Beantwortung der Forschungsfrage gewählt wurde, um aussagekräftige und reproduzierbare
Ergebnisse zu erzielen. Eine nachvollziehbare Methodik ist essenziell, um die Ergebnisse
sowohl evaluierbar als auch für zukünftige Arbeiten nutzbar zu machen. Das
Hauptziel dieser Arbeit ist die Entwicklung einer stabilen und voll
funktionsfähigen Erweiterung für die Software 3D Slicer, die in der Klinik eingesetzt
werden kann. Zu Beginn wurde demnach eine umfassende Anforderungsanalyse
durchgeführt, um die spezifischen Anforderungen der Domäne zu erfassen und die Ausgangssituation
zu klären. Darauf aufbauend folgte eine detaillierte Literaturrecherche, um den aktuellen
Stand der Technik zu untersuchen und bestehende Lösungen zu identifizieren. Da
das Ziel dieser Arbeit die Entwicklung einer vollständigen Softwarelösung ist, wurde
das Problem anschließend in Teilaufgaben zerlegt. Dies ermöglicht eine gezielte
Bearbeitung einzelner Komponenten und erleichtert die iterative Entwicklung. Falls
für bestimmte Teilbereiche keine passenden Lösungsansätze aus der Literatur ableitbar
waren, wurden darauf basierend eigene methodische Ansätze erarbeitet.

Neben der praktischen Anwendung der entwickelten Erweiterung bietet diese Arbeit
auch wissenschaftlichen Mehrwert. Daher wird im folgenden Abschnitt die gewählte
Methodik detailliert begründet und deren Vorteile herausgearbeitet.
% ---------------------------------------------------------------------------------------

\section{Forschungsdesign}
Das Forschungsdesign dieser Arbeit folgt einem praktischen Entwicklungsansatz mit
einem Fokus auf softwaretechnische Methoden. Zum Erreichen der Ziele stützt sich
diese Arbeit so am Entwicklungsprozess und dokumentiert diesen. Dabei lässt sich
der gesamte Zeitraum dieser Arbeit in drei Phasen aufteilen, die jeweils einem
unterschiedlichen Zweck diene. Diese drei Phasen sollen auch eine grobe Orientierung
bezüglich der Reihenfolge während der Bearbeitung geben.

\pagebreak

\begin{description}
	\item[\textbf{Analysephase}] Diese erste Phase ist bei fast allen Softwareprojekten
		die wichtigste Phase und gleichzeitig aber die, die meist zu kurz kommt. Innerhalb
		der Analysephase werden also alle Anforderungen an die Software gesammelt.
		Diese basieren zum großen Teil auf der Literaturrecherche. Außerdem werden bestehende
		Lösungen analysiert und so die Kernfunktionalität herausgefiltert.

	\item[\textbf{Entwicklungsphase}] Die Entwicklungsphase bildet den größten Teil.
		Hier findet die konkrete Umsetzung statt. Hierzu wird das System in mehrere Subsysteme
		unterteilt. Dies ermöglicht eine isolierte Betrachtung. Während der Entwicklung
		wird ein Phototypenansatz verfolgt.

	\item[\textbf{Evaluationsphase}] Die letzte Phase dieser Arbeit beschäftigt sich
		ausschließlich mit der Evaluation der Ergebnisse. Hier soll eine Antwort auf
		die in \ref{chap:fragestellung} formulierten Fragestellungen gefunden werden.
\end{description}

Durch diese Unterteilung ist eine gutes strukturelles vorgehen Möglich um mittels
einer praktischen Umsetzungsmethodik zu einem guten Ergebnis zu kommen. Die
nächsten Kapitel blicken nun in die einzelnen Phasen, beginnend mit einer Anforderungsanalyse.
% ---------------------------------------------------------------------------------------

\section{Anforderungsanalyse}
\label{sec:anforderungsanalyse} Nach genauerem Betrachten der Fragestellung aus Kapitel
\ref{chap:fragestellung} und den Zielen aus \ref{sec:ziel_der_arbeit} können bereits
einige Anforderungen abgeleitet werden, die für die Erweiterung gelten sollen. Neben
diesen Anforderungen wurden auch die Klinik für Zahnerhaltung mit in diesen Prozess
eingebunden. Hierzu wurde innerhalb eines Meetings mit dem verantwortlichen Arzt,
Dr. Elias Walter, ein Anforderungskatalog ausgearbeitet \citep[vgl.][]{walter2025}.
Diese Anforderungen waren vor allem zu Beginn der Entwicklung sehr wichtig um einen
ersten Anhaltspunkt zu gewinnen. Im Laufe des Entwicklungsprozesses wurden
Statusberichte eingeplant, die ein Reagieren auf Anforderungsänderungen ermöglichen
sollen.

In erster Linie wird klar, dass im Rahme dieser vorliegenden Arbeit eine
Extension für die Plattform 3D Slicer entwickelt werden soll. Die Kernfunktionalität
soll dabei die anatomische Segmentierung bilden, wie sie in Kapitel
\ref{sec:verwwandte_arbeit} beschrieben wurde. Greift man das Ziel dieser Arbeit
aus der Einleitung \ref{sec:ziel_der_arbeit} nochmals auf, dann kann hierdurch
die nächste wichtige Anforderung abgeleitet werden. Die Erweiterung soll gut und
einfach über ein User Interface (UI) bedient werden können. Außerdem ist eine
stabile Anwendung gefragt, die sich gut in die Kernanwendung von 3D Slicer einfügt.
\citet[]{walter2025} machte im Interview deutlich, das ie Extension neben einer
Einzelbildbearbeitung auch einen Batch-Prozess ermöglichen so. So können Beispielsweise
Parameter an einem Bild erprobt werden und diese anschließend in einen Batch-Prozess
für viele Bilder überführt werden. Außerdem soll es möglich sein, verschiedenen Schwellwertverfahren,
die in der anatomischen Segmentierung vorgesehen sind, auch in der Extention auszuwählen.
Ein wichtiger Softwaretechnischer Anspruch an die Extension ist die
Erweiterbarkeit. Es soll ohne große Mühen möglich sein, ein weiteres Verfahren
zu integrieren, ohne das große Anpassungen an der UI oder der Erweiterung selbst,
unternommen werden müssen. Für ein solides Verständnis dieser Software soll es selbstverständlich
eine Dokumentation mit Benutzerhandbuch geben. Zudem wird großer Wert auf die
Qualitätssicherung gelegt, weshalb eine Reihe von Unit-Tests (Tests für einzelne
Programmeinheiten) vorgesehen ist. Um die Anforderungen an die Software besser
zu verstehen und zu strukturieren, ist neben der Sammlung technischer Spezifikationen
auch ein solides Verständnis für die zugrunde liegende Domäne essenziell. Die
Abbildung \ref{fig:3d_slicer_domäne} veranschaulicht dies durch ein UML-Domänenmodell
(Unified Modeling Language), das einen visuellen Überblick über die verschiedenen
Teile der Software bietet. Dazu sind auch einige der Anforderungen erkennbar
\citep[vgl.][]{walter2025}.

\begin{figure}[h]
	\centering
	\includegraphics[width=0.75\textwidth]{img/domaenenmodell.jpg}
	\caption{UML-Domänenmodell des gesamten Softwaresystems}
	\label{fig:3d_slicer_domäne}
\end{figure}

Durch diese breite Palette an Anforderungen ergeben sich verschiedene Aufgaben
für die Implementierung. Bevor jedoch mit der konkreten Umsetzung begonnen werden
kann, ist ein noch wichtigerer Schritt erforderlich: die Recherche. Sie dient
dazu, den aktuellen Stand der Technik zu erfassen und geeignete Lösungsansätze zu
identifizieren.
% ---------------------------------------------------------------------------------------

\section{Recherche zum Stand der Kunst}
Es wäre äußerst ungünstig, erst am Ende eines Projekts festzustellen, dass bereits
veröffentlichte Lösungen existieren, in die erhebliche Ressourcen investiert
wurden. Um dies zu vermeiden, ist eine umfassende Literaturrecherche essenziell,
die den aktuellen Stand der Technik abbildet. Dabei wird auf Fachliteratur sowie
domänenspezifische Quellen zurückgegriffen, um alle relevanten Aspekte abzudecken.

Für diese Arbeit spielt eine Quelle eine besonders wichtige Rolle: die
offizielle Dokumentation von \citet{slicer2024}. Sie bietet wertvolle
Anhaltspunkte für die Implementierung und hilft dabei, die technischen Gegebenheiten
von 3D Slicer zu verstehen. Zudem enthält sie Best-Practice-Ansätze, die bei der
Entwicklung berücksichtigt wurden. 3D Slicer stellt außerdem einen Developer
Guide zur Verfügung, der Teil der offiziellen Dokumentation ist und den Einstieg
in das Framework erleichtert. Ein weiterer zentraler Referenzpunkt ist der 3D
Slicer Extension Index, in dem bereits entwickelte Erweiterungen einsehbar sind.
Ein konkretes Beispiel ist das Modul \textit{Airway Segmentation}, dessen
Analyse dazu beiträgt, bewährte Konventionen für die Entwicklung der eigenen Erweiterung
abzuleiten.

Neben einer konkreten Implementierungshilfe dient die Literaturrecherche auch dazu,
ein fundiertes Verständnis für die Domäne der medizinischen Bildverarbeitung und
deren zugrunde liegende Strukturen zu entwickeln. Mithilfe verschiedener domänenspezifischer
Publikationen kann ein tieferes Wissen über diesen Fachbereich gewonnen werden.
Besonders relevant sind hierbei die verschiedenen Verfahren für die Verarbeitung
der Micro CT Aufnahmen. Konkret handelt es sich hier um die unterschiedlichen Algorithmen
zur Filterung und Segmentierung von Micro CT Bildern in der Zahnmedizin.

Darüber hinaus ermöglicht die Recherche einen Blick auf alternative Plattformen zur
Bildverarbeitung, wie beispielsweise die weit verbreitete Software ITK-SNAP. Ein
kurzer Vergleich ergab jedoch, dass diese Lösung aufgrund ihrer Struktur in
diesem speziellen Fall nicht mit 3D Slicer konkurrieren kann.

Die Recherche bietet somit einen ersten fundierten Überblick über mögliche
Lösungen für die einzelnen Anforderungen. Um nun detaillierter auf die Umsetzung
einzugehen, nimmt das nächste Kapitel eine Unterteilung der Gesamtheit der
Anforderungen in kleinere Teilsysteme vor.
% ---------------------------------------------------------------------------------------

\section{Zerlegung in Teilprobleme}
\label{sec_zerlegung_in_teilprobleme} Durch die Aufteilung des Gesamtsystems in
mehrere kleine Teilaufgaben wird die Software für den Entwicklungsprozess
übersichtlicher. Die einzelnen Domänen können so schneller und besser verstanden
werden. Es gibt viele Möglichkeiten ein Softwaresystem in kleine Teile
aufzuteilen, sodass es am Ende auf den konkreten Anwendungsfall ankommt. Diese
Arbeit sieht folgenden Teilaufgaben für das Gesamtsystem vor:

\begin{description}
	\item[\textbf{Architekturdesign:}] Mithilfe von UML Diagrammen soll die Architektur
		dieses Systems abgebildet werden und sukzessive immer detaillierter beschrieben
		werden. Es soll dann verglichen werden, welche Entwurfsmuster für dieses System
		infrage kommen. Durch die Bearbeitung dieses Teilproblems kann die
		Anforderung an eine flexible Architektur erfüllt werden. Anschließend kann mit
		der Entwicklung des UI-Designs begonnen werden.

	\item[\textbf{UI Design:}] Es soll ein Design erstellt werden, dass sich an
		erfolgreichen und etablierten 3D Slicer Extensions orientiert. Jedoch sollen
		die Wünsche des Endnutzers auch nicht zu kurz kommen. Für eine
		Visualisierung des Designs bedient sich diese Arbeit der Wireframes.

	\item[\textbf{Pseudo-Extension:}] Befor der tatsächliche Algorithmus eingebunden
		werden kann, ist es wichtig eine funktionierende Erweiterung zu haben, die noch
		keine konkrete Aufgabe hat, aber funktioniert und in Slicer eingebunden
		werden kann.

	\item[\textbf{Kappselung Hoffmann:}] Nachdem die leere Extension lauffähig ist
		und auch einige Hilfsfunktionen bereitstehen, kann mit der Paketerstellung
		des Hoffmann begonnen werden. Hier soll das Verfahren von einem Python Notebook
		in eine Bibliothek überführt werden, sodass dieses Verfahren in der
		Extention ausführbar ist. Die konkrete Art des Paketes ist noch nicht festgelegt.

	\item[\textbf{Parameter:}] Der Benutzer steuert das Verfahren über die Parameter
		in der UI. Für die Speicherung der Parametereinstellungen hat Slicer den
		Mechanismus ParameterNode entworfen. Diese wurde bereits in Abschnitt \ref{subsec:benutzerschnitstelle}
		erwähnt. Dieser Mechanismus ist nicht trivial, erhöht die Benutzerfreundlichkeit
		des Systems aber erheblich und soll demnach auch in diese Extention Anwendung
		finden.

	\item[\textbf{Single Prozess:}] Sobald alle notwendigen Vorbereitungen getroffen
		sind, kann der Algorithmus nun eingebettet werden. Hierzu betrachtet man isoliert
		den Single Prozess. Auch die UI wird erst nur so weit entwickel, wie es für den
		einfachen Prozess nötig ist. Hierbei wird auf das erstellte Paket für das
		Hoffmann Verfahren und die zuvor erstellen Hilfsfunktionen zurückgegriffen.

	\item[\textbf{Batch Prozess:}] Ist das einfache Verfahren fertig implementiert
		und funktioniert, so kann der Batch Prozess hinzukommen. Hier bedarf es einer
		zusätzlichen Arbeit in der UI, da der Benutzer über das Verwenden dieser
		Funktion gewarnt werden muss. Der Batch Prozess bedarf nämlich erheblicher Ressourcen.
		Hinzukommt die Implementierung einer Fortschrittsanzeige, sodass zu erkennen
		ist, dass ein Hintergrundprozess läuft.

	\item[\textbf{Dokumentation und Benutzerhandbuch:}] Abschließend ist eine ausführliche
		Dokumentation der Architektur erwünscht, sodass zukünftige Entwickler wissen,
		wo sie ansetzten müssen. Hinzu kommt ein Benutzerhandbuch für eine
		Verwendung der Erweiterung. Das Benutzerhandbuch und die Architekturdokumentation
		erfolgen in einer README.md innerhalb der Extension.

	\item[\textbf{Tests:}] An letzter Stelle sollen noch Softwaretests implementiert
		werde, um die Richtigkeit der Extension sicherzustellen. 3D Slicer sieht hier
		Unittests vor, die über den Developer Modus in Slicer direkt in der
		jeweiligen Extension ausgeführt werden können.
\end{description}

Die Ordnung dieser Punkte gibt eine grobe Orientierung bezüglich der Reihenfolge
während der Umsetzung an. Damit eine Umsetzung überhaupt realisiert werden kann,
sind unterschiedliche Werkzeuge und Mittel notwendig. Diese sollen im nächsten
Kapitel kurz erläutert werden.
% ---------------------------------------------------------------------------------------

\section{Entwicklungsumgebung}
Da bereits ein Framework feststeht, mit dem gearbeitet werden soll, ist keine
weitere Forschung nötig, um die richtige Programmiersprache auszuwählen. Jedoch
gibt es eine kleine Auswahl zu treffen. 3D Slicer unterscheidet zwischen zwei Arten
von Modulen, die CLI-Module (Commend Line Interface), welche in der Sprache C++ geschrieben
werden und die Scripted Moduls, die eine Python Implementierung verlangen. Da die
anatomische Segmentierung ohnehin in einem Python Notebook bereitliegt, fiel die
Wahl hier auf die Scripted Moduls. So kann auch die breite Palette der Python Pakete
genutzte werden. Für eine detalierte Beschreibung des Frameworks selber sei an
dieser Stelle auf das Kapitel \ref{sec:bildbearbeitung} verwiesen, indem das Framework
und alle zugehörigen Eigenheiten noch genauer beschrieben wurden. Um den
Entwicklungsprozess etwas zu vereinfachen, wurde während der Entwicklung auf ein
Modul von Slicer zurückgegriffen, das speziell für Entwickler entworfen wurde. Die
Abbildung \ref{fig:entwicklungsumgebung} verdeutlicht dieses Tool.

\begin{figure}[h]
	\centering
	\includegraphics[width=0.6\textwidth]{img/Entwicklungsumgebung.png}
	\caption{Umgenbung während der Entwicklung mit 3D Slicer und PyCharm}
	\label{fig:entwicklungsumgebung}
\end{figure}

Mit den Debugging Tools lässt sich eine gewohnte Umgebung reproduzieren, in der der
Quellcode Schritt für Schritt analysiert werden kann. Speziell bei der
Fehlersuche ist dieses Toll eine sehr gute Unterstützung. Die Abbildung beschreibt
weiter, das als Umgebung für das Erstellen des Programmcodes die Software
Pycharm verwendet wird. Pycharm ist eine Lösung der Firma Jetbrains, für das
Erstellen von Python Quellcode. Dieses Tool bietet eine breite Palette an Funktionalitäten,
die das Erstellen von Software vereinfachen und kann als \textit{State of the
Art} bezeichnet werden.

Neben der eigentlichen Umgebung und den Entwicklerwerkzeugen steht zur Entwicklung
auch ein bereits erstelltes Python Paket zur Verfügung, das von Herrn Prof. Rösch
speziell für die Klinik für Zahnerhaltung an der LMU in München erstellt wurde.
Dieses Tool beinhaltet diverse Funktionalität für das Verarbeiten von
medizinischen Bilddaten. Speziell für die Micro CT Aufnahmen der Klinik.

Nachdem die Anforderungen, die Recherche, die konkreten Aufgaben und die verfügbaren
Werkzeuge erläutert wurden, bleibt noch die Evaluation der Arbeite. Das Kapitel Forschungsevaluation
erläutert die Methodik, mit dem das Erreichen des Forschungsziels messbar
gemacht werden kann.

% ---------------------------------------------------------------------------------------

\section{Forschungsevaluation}
Die Evaluation kann grob in zwei Teile unterteilt werden. Der erste Teil ist der
wohl wichtigste und beschäftigt sich mit dem Testen der Anwendung durch die
Benutzer.
% ---------------------------------------------------------------------------------------