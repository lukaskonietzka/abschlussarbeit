\chapter{Methodik}
\label{chap:methodik} Dieses Kapitel beschreibt das methodische Vorgehen, das
zur Beantwortung der Forschungsfrage gewählt wurde, um aussagekräftige und reproduzierbare
Ergebnisse zu erzielen. Eine nachvollziehbare Methodik ist essenziell, um die Ergebnisse
sowohl evaluierbar als auch für zukünftige Arbeiten nutzbar zu machen. Das
Hauptziel dieser Arbeit ist die Entwicklung einer Erweiterung für die Software
3D Slicer, die in der Klinik eingesetzt werden kann. Dabei geht es nicht
ausschließlich um die Entwicklung einer Software, sondern auch um die Art der Entwicklung
und die Ideen hinter der Implementierung. Wie können Mikro-\ac{CT}-Bilder für die
Zahnmedizin effizient und interaktiv aufbereitet werden? Zu Beginn wurde demnach
eine umfassende Anforderungsanalyse durchgeführt, um die spezifischen Anforderungen
der Domäne zu erfassen und die Ausgangssituation zu klären. Darauf aufbauend
folgte eine detaillierte Recherche, um den aktuellen Stand der Technik zu untersuchen
und bestehende Lösungen für eine Implementierung zu identifizieren. Da das Ziel
dieser Arbeit die Entwicklung einer vollständigen Softwarelösung ist, wurde das Problem
anschließend in Teilaufgaben zerlegt. Dies ermöglicht eine gezielte Bearbeitung
einzelner Komponenten und erleichtert die iterative Entwicklung.
% ---------------------------------------------------------------------------------------

\section{Forschungsdesign}
Das Forschungsdesign dieser Arbeit folgt einem praktischen Entwicklungsansatz
mit einem Fokus auf softwaretechnische Methoden. Zum Erreichen der Ziele stützt sich
diese Arbeit so am Entwicklungsprozess ab und dokumentiert diesen. Während des
gesamten Entwicklungsprozesses wurde stets die Kommunikation zum Fachpersonal
gepflegt. So konnte schnell beurteilt werden, ob die Entwicklung in die richtige
Richtung geht. Im Falle dieser Arbeit übernahm diese Position Dr. Elias Walter, welcher
als praktizierender Arzt an der Klinik tätig ist. Durch das Aufeinandertreffen
zweier Fachexperten, kann so der optimale Leistungsumfang bestimmt werden. Des
Weiteren lässt sich der gesamte Zeitraum dieser Arbeit in drei Phasen aufteilen,
die jeweils einem unterschiedlichen Zweck dienen. Diese drei Phasen sollen auch
eine Orientierung bezüglich der Reihenfolge während der Bearbeitung geben.
\pagebreak

\begin{description}
	\item[Analysephase,] diese erste Phase ist bei fast allen Softwareprojekten die
		wichtigste Phase und gleichzeitig die, die meist zu kurz kommt. Innerhalb der
		Analysephase werden also alle Anforderungen an die Software gesammelt. Diese
		basieren zum großen Teil auf der Recherche. Außerdem werden bestehende Lösungen
		analysiert und so die Kernfunktionalität herausgefiltert.

	\item[Entwicklungsphase,] die Entwicklungsphase bildet den größten Teil. Hier
		findet die konkrete Umsetzung statt. Hierzu wird das System in mehrere Subsysteme
		unterteilt, was eine isolierte Betrachtung ermöglicht. Während der Entwicklung
		wird ein prototypischer Ansatz verfolgt.

	\item[Evaluationsphase,] die letzte Phase dieser Arbeit beschäftigt sich ausschließlich
		mit der Evaluation der Ergebnisse. Hier soll eine Antwort auf die in Kapitel
		\ref{sec:ziel_der_arbeit} formulierten Fragestellungen gefunden werden.
\end{description}

Durch diese Unterteilung ist ein gutes strukturelles vorgehen Möglich um mittels
einer praktischen Umsetzungsmethodik zu einem guten Ergebnis zu kommen. Die
restlichen Abschnitte in diesem Kapitel bilden die eben genannte Analysephase.
% ---------------------------------------------------------------------------------------

\section{Anforderungsanalyse}
\label{sec:anforderungsanalyse} Nach genauerem Betrachten der Fragestellung aus
Kapitel \ref{sec:ziel_der_arbeit} können bereits erste Anforderungen abgeleitet
werden. Neben diesen Anforderungen wurden auch die Klinik für Zahnerhaltung mit in
diesen Prozess eingebunden, um so mehr relevante Anforderungen zu definieren. Hierzu
wurde innerhalb einer Besprechung mit dem verantwortlichen Arzt, Dr. Elias Walter,
ein Anforderungskatalog ausgearbeitet \citep[vgl.][]{walter2025}. Das Protokoll
dieser Besprechung ist dem Anhang zu entnehmen. Diese Anforderungen waren vor allem
zu Beginn der Entwicklung sehr wichtig um einen ersten Anhaltspunkt zu gewinnen.
Im Laufe des Entwicklungsprozesses wurden Statusberichte eingeplant, die ein Reagieren
auf Anforderungsänderungen ermöglichen.

Um die Anforderungen an die Software besser zu verstehen und zu strukturieren,
ist neben der Sammlung technischer Spezifikationen auch ein solides Verständnis
für die zugrunde liegende Domäne essenziell. Die Abbildung \ref{fig:3d_slicer_domäne}
veranschaulicht dies durch ein Domänenmodell. Mittels dieser Grafik können die
gestellten Anforderungen visuell dargestellt werden.

\begin{figure}[h]
	\centering
	\includegraphics[width=0.9\textwidth]{img/domaene.png}
	\caption{Domänenmodell des gesamten Softwaresystems}
	\label{fig:3d_slicer_domäne}
\end{figure}

In erster Linie wird klar, dass im Rahmen dieser vorliegenden Arbeit eine
Erweiterung für die Plattform 3D Slicer entwickelt werden soll. Die Kernfunktionalität
soll dabei die anatomische Segmentierung bilden, wie sie in Kapitel
\ref{chap:theoretische_grundlagen} beschrieben wurde. Neben dieser Kernfunktionalität
soll auch eine Vor- und Nachbereitung der Bilder möglich sein. Dieser Prozess der
Vorverarbeitung, Bearbeitung und Nachbearbeitung soll an einem Stück ausgeführt
werden können. Die Erweiterung soll gut und einfach über eine \ac{UI} bedient
werden können. Außerdem ist eine stabile Anwendung gefragt, die sich gut in die
Kernanwendung von 3D Slicer einfügt. \citet[]{walter2025} machte im Interview
deutlich, dass die Erweiterung neben einer Einzelbildbearbeitung auch einen Batch-Prozess
ermöglichen soll. So können beispielsweise Parameter an einem Bild erprobt
werden und diese anschließend in einen Batch-Prozess auf viele Bilder überführt
werden. Außerdem soll es möglich sein, verschiedenen Schwellwertverfahren, die in
der anatomischen Segmentierung vorgesehen sind, auch in der Erweiterung
auszuwählen. Neben dem Ergebnis selber wurden auch Anforderungen an die Art der Abspeicherung
gesetzt. Es ist wichtig, dass es in der Slicer-Szene nicht zu einer Überfüllung von
Daten kommt. So soll nach jeder Einzelausführung die Szene wieder geleert werden,
sodass immer nur das aktuell segmentierte Bild in der Anwendung liegt. Neben den
segmentierten Daten sollen auch die medialen Flächen für jedes Segment
bereitgestellt werden. Ähnlich wie die Segmentierung selbst sollen diese mit jeder
neuen Generierung aktualisiert werden.

Ein wichtiger Softwaretechnischer Anspruch an das \ac{SEM} ist die
Erweiterbarkeit. Es soll ohne große Mühen möglich sein, ein weiteres Verfahren
zu integrieren, ohne das große Anpassungen an der \ac{UI} oder der Erweiterung
selbst unternommen werden müssen. Für ein solides Verständnis dieser Software soll
es selbstverständlich eine Dokumentation mit Benutzerhandbuch geben. Zudem wird
großer Wert auf die Qualitätssicherung gelegt, weshalb eine Reihe von Unit-Tests
(Tests für einzelne Programmeinheiten) vorgesehen ist.

Durch diese breite Palette an Anforderungen ergeben sich verschiedene Aufgaben für
die Implementierung. Bevor jedoch mit der konkreten Umsetzung begonnen werden
kann, ist ein noch wichtigerer Schritt erforderlich: die Recherche. Sie dient dazu,
den aktuellen Stand der Technik zu erfassen und geeignete Lösungsansätze für die
Implementierung zu identifizieren.
% ---------------------------------------------------------------------------------------

\section{Literatur zur Implementierung}
Es wäre äußerst ungünstig, erst am Ende eines Projekts festzustellen, dass
bereits veröffentlichte Lösungen existieren, in die erhebliche Ressourcen investiert
wurden. Um dies zu vermeiden, ist eine gute Recherche essenziell, die den
aktuellen Stand der Technik abbildet.

Für diese Arbeit spielt eine Quelle eine besonders wichtige Rolle: die offizielle
Dokumentation von 3D Slicer. Sie bietet wertvolle Anhaltspunkte für die
Implementierung und hilft dabei, die technischen Gegebenheiten von 3D Slicer zu verstehen.
Zudem enthält sie \textit{Best-Practice-Ansätze}, die bei der Entwicklung
berücksichtigt wurden. 3D Slicer stellt außerdem einen \textit{Developer Guide}
zur Verfügung, der Teil der offiziellen Dokumentation ist und den Einstieg in das
Framework erleichtert. Ein weiterer zentraler Referenzpunkt ist der \textit{3D
Slicer Extension Index}, in dem bereits entwickelte Erweiterungen einsehbar sind.
Ein konkretes Beispiel ist das Modul \textit{Airway Segmentation}, dessen Analyse
dazu beiträgt, bewährte Konventionen für die Entwicklung der eigenen Erweiterung
abzuleiten. So kommt es, dass beispielsweise für die Gestaltung einer
Benutzerschnittstelle bereits Lösungen existieren, die sich bewährten und so übernommen
werden können. Da dieses Modul mit der Absicht entwickelt wird, später im \textit{Extension
Manager} veröffentlicht zu werden, sind einige dieser Richtlinien im Leitfaden obligatorisch.
Welche dies sind, ist der Dokumentation von Slicer zu entnehmen.

Neben einer konkreten Implementierungshilfe dient die Recherche zur Implementierung
auch dazu, ein fundiertes Verständnis für die Domäne der medizinischen
Bildverarbeitung und deren zugrunde liegende Strukturen zu entwickeln. Mithilfe verschiedener
domänenspezifischer Publikationen, kann ein tieferes Wissen über diesen
Fachbereich gewonnen werden. Besonders relevant sind hierbei die verschiedenen Verfahren
für die Verarbeitung der Mikro-\ac{CT}-Aufnahmen. Konkret handelt es sich hier um
die unterschiedlichen Algorithmen zur Filterung und Segmentierung von Mikro-\ac{CT}-Bildern
in der Zahnmedizin.

Um nun detaillierter auf die Umsetzung einzugehen, nimmt das nächste Kapitel eine
Unterteilung der Gesamtheit der Anforderungen in kleinere Teilsysteme vor.
% ---------------------------------------------------------------------------------------

\section{Zerlegung in Teilprobleme}
\label{sec_zerlegung_in_teilprobleme} Durch die Aufteilung des Gesamtsystems in mehrere
kleine Teilaufgaben wird die Software für die Entwicklungsphase übersichtlicher.
Die einzelnen Domänen können so schneller und besser verstanden werden. Es gibt viele
Möglichkeiten ein Softwaresystem in kleine Teile aufzuteilen, sodass es am Ende
auf den konkreten Anwendungsfall ankommt. Diese Arbeit sieht folgenden Teilaufgaben
für das Gesamtsystem vor:

\begin{description}
	\item[Architektur- und \ac{UI}-Design,] die Systemarchitektur wird mithilfe von
		Klassendiagrammen modelliert und schrittweise verfeinert. Anschließend wird
		ein geeignetes Entwurfsmuster ausgewählt, das die Struktur und Modularität des
		Systems unterstützt. Durch die Bearbeitung dieses Teilproblems kann die
		Anforderung an eine flexible Architektur gut erfüllt werden.

	\item[Kapselung der anatomischen Segmentierung,] das bereits bestehende Segmentierungsverfahren
		muss in das Modul integriert werden. Hier soll das Verfahren von einem
		IPython-Notebook in eine Python-Bibliothek überführt werden, sodass dieses Verfahren
		in der Erweiterung ausführbar ist.

	\item[Parameter Node,] der Benutzer steuert das Verfahren über die Parameter in
		der \ac{UI}. Für die Speicherung der Parametereinstellungen hat Slicer den Mechanismus
		\textit{ParameterNode} entworfen. Dieser Mechanismus ist nicht trivial,
		erhöht aber die Benutzerfreundlichkeit des Systems erheblich und soll
		demnach auch in diese Erweiterung Anwendung finden.

	\item[Einzelbildbearbeitung,] sobald alle notwendigen Vorbereitungen getroffen
		sind, kann der Algorithmus nun eingebettet werden. Hierzu betrachtet man isoliert
		die Einzelbildbearbeitung. Auch die \ac{UI} wird erst nur so weit entwickelt,
		wie es für den einfachen Prozess nötig ist. Hierbei wird auf das erstellte Paket
		der anatomischen Segmentierung zurückgegriffen.

	\item[Batch-Prozess,] ist die Einzelbildbearbeitung fertig implementiert und funktioniert,
		so kann der Batch-Prozess hinzukommen. Hier geht es federführend um das richtige
		und organisierte Abspeichern der erstellten Bilder im Dateisystem.

	\item[Ausführungsmodus,] während der Ausführung des Algorithmus soll das Modul
		in einen Aufführungszustand wechseln. Hierbei ist wichtig, dass die \ac{UI}
		in dieser Zeit gesperrt bleibt.

	\item[Dokumentation und Test,] abschließend ist eine Dokumentation der Architektur
		erwünscht, sodass zukünftige Entwickler wissen, wo sie ansetzten müssen.
		Hinzu kommt ein Benutzerhandbuch für eine Verwendung der Erweiterung. Das Benutzerhandbuch
		und die Architekturdokumentation erfolgen in einer README.md innerhalb der Erweiterung.
		An letzter Stelle sollen noch Softwaretests implementiert werde, um die Richtigkeit
		der Erweiterung sicherzustellen. Hierfür sind Unittests vorgesehen.
\end{description}

Die Ordnung dieser Punkte gibt eine grobe Orientierung bezüglich der Reihenfolge
während der Umsetzung an. Bevor es im nächsten Kapitel um die konkreten Ergebnisse
diese Arbeit geht, soll hier abschließend noch ein Blick auf die Art der
Evaluation geworfen werden. Diese stellt sicher, dass die Ergebnisse am Ende
bewertbar sind.
% ---------------------------------------------------------------------------------------

\section{Forschungsevaluation}
Die Evaluation kann grob in zwei Teile unterteilt werden. Der erste Teil ist der
wohl Wichtigste und beschäftigt sich mit dem Testen der Anwendung durch die Benutzer.
Hier kann also die Benutzerfreundlichkeit und die \ac{UI} der Erweiterung gut analysiert
werden. Diese Art von Test involviert mehrere Außenstehende und sorgt damit für
eine objektive Bewertungsgrundlage. Verbesserungen, die durch die Tests
hervorkommen, werden gesammelt und als möglicher Ausblick zur Verfügung gestellt.
Wichtig für die Benutzbarkeit der Software ist auch das Benutzerhandbuch. Dies ist
auch Teil der Ergebnisse und muss mittels Benutzertests evaluiert werden. Neben
dem konkreten Feedback der Anwender sollen auch verschiedenen
Anwendungsszenarien diskutiert werden.

Der zweite Teil der Evaluation soll prüfen ob der softwaretechnische Ansatz
erfolgreich umgesetzt wurde. Um dies gewährleisten zu können, müssen zusätzlich
zur Funktionalität auch Softwaretests bereitgestellt werden. Wie eine der Teilaufgaben
aus Kapitel \ref{sec_zerlegung_in_teilprobleme} bereits zeigt, handelt es sich
hierbei um Unittests. Außerdem ist für diesen Teil auch die technische Dokumentation
notwendig. Abschießend soll die Performance des Systems noch gemessen und
analysiert werden.

Die in diesem Kapitel beschriebenen methodischen Schritte bildeten die Grundlage
für die Entwicklung der Erweiterung. Nachdem die Analysephase abgeschlossen ist,
folgen nun die Phasen Entwicklung und Evaluation in Form der Kapitel Ergebnisse
und Evaluation. Hierzu wird das erstellte \ac{SEM} detailliert vorgestellt und
die Funktionen näher beschrieben.
% ---------------------------------------------------------------------------------------