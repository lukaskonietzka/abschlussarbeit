\chapter{Methodik}
\label{chap:methodik} Wie das vorherige Kapitel bereits eingeleitet hat, soll es
hier um das \textit{Wie} gehen. Es wird also aufgezeigt, welches methodische Herangehen
an die Fragestellung verfolgt wurden, um ein aussagekräftiges Ergebnis zu
erzielen. Dabei wurde mit einer umfangreichen Anforderungsanalyse gestartet,
welche die Domäne und die Ausgangslage klären soll. Sind diese Konzepte klar, so
wird die endgültige Problemstellung in mehrere kleine Teilaufgaben zerlegt. Für jede
dieser Teilaufgaben wurde dann eine Recherche zum Stand der Technik durchgeführt,
um bereits existierenden Lösungen ausfindig zu machen. Sofern es für eine Teilaufgabe
noch keine Lösung gibt, werden anschließend konkrete Lösungsansätzen für die
einzelnen Teilprobleme erarbeitet. Sollte für eine Aufgabe mehrere Ansätze existieren,
so werden diese im letzten Abschnitt miteinander verglichen und ein passender
Ansatz gewählt.

Bei der Durchführung dieser Schritte zum Erreichen eines Ergebnisses, soll der in
der Softwareentwicklung allgemeine bekannte Ansatz \textit{make it run, make it right,
make it fast} verfolgt werden. Dieser beschreibt, dass zunächst dafür gesorgt
werden soll, dass ein Problem überhaupt gelöst wird. Im Anschluss soll es so
umgebaut werden, dass eine Wartbarkeit und Erweiterbarkeit entsteht. Erst an
letzter Stelle steht die Performance. QUELLE

\section{Anforderungsanalyse}
\label{sec:anforderungsanalyse} Nach genauerem Betrachten der Fragestellung aus
Kapitel \ref{chap:fragestellung} wird klar, dass im Rahmen dieser vorliegenden Arbeit
eine Extension für die Plattform 3D Slicer entwickelt werden soll. Diese
Erweiterung beinhaltet das Segmentierungsverfahren nach Hoffmann \citep[vgl.][]{hoffmann2020},
wie es in Kapitel \ref{sec:verwwandte_arbeit} beschrieben wurde. Das Verfahren segmentiert
Micro-CT Aufnahmen der Zahnklinik in München und wird zu Forschung eingesetzt.
Da Ärzte keine Softwareentwickler sind, ist es wichtig, dass das Verfahren eine
UI erhält die eingängig und übersichtlich ist. Außerdem ist eine stabile Anwendung
gefragt, die sich gut in die Kernanwendung von 3D Slicer einfügt. Für einen Überblick
über die wichtigsten Eigenschaften von 3D Slicer sei auf das Kapitel \ref{sec:3d_slicer}
verwiesen.

Die Extension selber soll neben einer Einzelbildbearbeitung auch einen Batch-Prozess
ermöglichen. So können Beispielsweise Parameter an einem Bild erprobt werden und
diese anschließend in eine Batchprozess für viele Bilder überführt werden. Außerdem
soll es möglich sein, verschiedenen Segmentierungsverfahren, die in Hoffmann vorgesehen
sind, auch in der Extention auszuwählen.

Ein wichtiger Softwaretechnischer Anspruch an die Extension ist die
Erweiterbarkeit. Es soll ohne große Mühen möglich sein, ein weiteres Verfahren
zu integrieren, ohne das große Anpassungen an der UI oder der Erweiterung selbst,
unternommen werden müssen. Für die Architektur und die Benutzung dieser Software
soll es selbstverständlich eine Dokumentation mit Benutzerhandbuch geben. Auch
das Testen der Software soll natürlich nicht zu kurz kommen, weswegen eine Reihe
an Unit-Tests (Das Testen einer konkreten Programmeinheit) vorgesehen ist.

\begin{figure}[h]
	\centering
	\includegraphics[width=0.8\textwidth]{img/domaenenmodell.jpg}
	\caption{UML-Domänenmodell des gesamten Softwaresystems}
	\label{fig:3d_slicer_domäne}
\end{figure}

Diese doch breite Palette an Anforderungen lässt sich unmöglich auf einmal bearbeiten.
Hierzu sieht diese Arbeit eine Aufteilung in Teilprobleme vor. Der nächste Abschnitt
blickt auf die herausgearbeiteten Anforderungen in diesem Kapitel und leitet daraus
Teilprobleme ab.

\section{Zerlegung in Teilprobleme}
\label{sec_zerlegung_in_teilprobleme} Durch die Aufteilung des Gesamtsystems in mehrere
kleine Teilaufgaben, wird die Software für den Entwicklungsprozess
übersichtlicher. Die einzelnen Domänen können so schneller und besser verstanden
werden. Es gibt viele Möglichkeiten ein Softwaresystem in kleine Teile
aufzuteilen, sodass es am Ende auf den konkreten Anwendungsfall ankommt. Diese
Arbeit sieht folgenden Teilaufgaben für das Gesamtsystem vor:

\begin{itemize}
	\item \textbf{Architekturdesign:} Mit Hilfe von UML Diagrammen soll die
		Architektur dieses Systems abgebilde werden und suksesive immer detalierte
		beschrieben werden. Es soll dann verglichen werden, welche Softwarepatterns
		für dieses System in Frage kommen, Durch Bearbeitung dieses Teilproblems, kann der
        Anforderung der flexiblen Architektur gerecht weden. Ist dies geschehen, so kann mit einem UI
		Design begonnen werden.

	\item \textbf{UI Design:} Es soll eine Design erstellt werden, dass sich an erfolgreichen
		und etablierten 3D Slicer Extentions orientiert. Jedoch sollen die Wünsche
		des Endnutzers auch nicht zu kurz kommen. Für eine visualsierung des Designs
		bedient sich diese Arbeit der Wireframes.

	\item \textbf{Pseude-Extension:} Bevor der tatsächliche Algorithmus
		eingebunden werden kann, ist es wichtig eine funktionierende Erweiterung zu haben,
		die erstmal nichts sinvolles tut, aber funktioniert und in Slicer eingebunden
		werden kann.

	\item \textbf{Hoffmann als Python Paket:} Nachdem die leere Extension
		lauffähig ist, kann mit der Paketerstellung des Hoffmann begonnen werden.
		Hier soll das Verfahren von einem Python Notebook in eine Python Wheel Datei
		überführt werden, sodass eine abkapselung gegeben ist.

	\item \textbf{Speicheurng der Parameter:} Für die Speicherung der
		Parametereinstellungen hat Slicer den Mechanismus des ParameterNode
		entworfen. Diese wurde bereits in Abschnitt \ref{subsec:benutzerschnitstelle}
		erwähnt. Diese Mechnismus ist nicht trivial, erhöht die Benutzerfreundlichkeit
		des Systems aber erheblich.

	\item \textbf{Single Prozess:} Sobald alle notwendigen Vorbereitungen
		getorffen sind, kann der Algorithmus nun eingebettet werden. Hierzu
		betrachtet man isoliert den Single Prozess. Auch Die Ui wird erst nur so weit
		gebaut, wie es für diesen Prozess nötig ist.

	\item \textbf{Batch Prozess:} Ist das einfache Verfahren fertig implementiert
		und funktioniert, so kann der Batch Prozess hinzukommen. Hier muss bedarf
        es einer zusätzlichen Arbeit in der UI, da der Benutzer über das Verwenden
        dieser Funktion gewarnt werden muss. Der Batch Prozess bedard nämlich
        erheblicher Ressourcen.

	\item \textbf{Dokumentation und Benutzerhandbuch:} Abschließend ist eine
		ausführliche Dokumentation der Architektur erwünscht, sodass zukünftige Entwickler
		wissen, wo sie ansetzten müssen. Hinzu kommt eine Benutzerhandbuch für eine Verwendung
		der Erweiterung.

	\item \textbf{Test:} An letzter Stelle sollen noch Softwaretests implementiert
		werde, um die Richtigkeit der Extension sicherzustellen. 3D Slicer sieht
		hier Unittests vor.
\end{itemize}

Die Ordnung dieser Punkte gibt eine grobe Orientierung bezüglich der Reihenfolge
während der Umsetzung an. Bei der Bearbeitung der einzelnen Teilaufgaben ist es
auch wichtig eine gute Recherche zum aktuellen Stand der Technik durchzuführen. Es
ist sehr ungünstig, wenn sich zu Ende eines Projektes herraustellt, dass Lösungen,
in die erhebliche Ressourcen investiert wurden, bereits veröffentlicht sind.

\section{Recherche zum Stand der Kunst}
\label{sec:recherche} Bei der Recherche ist es wichtig

\section{Erarbeiten von Lösungsansätzen}
\label{sec:lösungsansätze} hier geht es um Brainstorming

\textbf{Architekturdesign}

\textbf{UI Design}

\textbf{Pseude Extension}

\textbf{Hoffmann als Python Paket}

\textbf{Single Prozess}

\textbf{Batch Prozess}

\textbf{Analysen}

\section{Auswahl von Lösungsansätzen}