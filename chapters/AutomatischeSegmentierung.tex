\chapter{Automatische Segmentierung mittels 3D Slicer}
\label{sec:3d_slicer} 3D Slicer ist eine Open-Source-Plattform, die speziell für
die Verarbeitung von Bilddaten im medizinischen Kontext eingesetzt wird. Dabei wird
sie von einer aktiven Community regelmäßig gewartet und weiterentwickelt \citep[vgl.][]{slicer2024},
\citep[vgl.][S.~1325]{fedorov2012slicer}. Für Slicer gibt es offiziell keine Nutzungsbeschränkung.
Jedoch sei auch gesagt, dass 3D Slicer nicht für die klinische Nutzung zugelassen
ist. \citet[S.~1331]{fedorov2012slicer} machen deutlich, dass 3D Slicer
ausschließlich für die Forschung gedacht ist. Um einen ersten Überblick über die
Komponenten von Slicer zu erlangen, soll die Abbildung
\ref{fig:3d_slicer_oekosystem} betrachtet werden.

\begin{figure}[h]
	\centering
	\includegraphics[width=1\textwidth]{img/3d_slicer_overview.jpg}
	\caption{3D Slicer Ökosystem nach \citet[S.~1326]{fedorov2012slicer}}
	\label{fig:3d_slicer_oekosystem}
\end{figure}

\citet[S.~1326]{fedorov2012slicer} teilt mit der Abbildung
\ref{fig:3d_slicer_oekosystem} die Plattform in drei Schichten auf. Auf der Obersten
wird klar, dass 3D Slicer aus der Kernanwendung und den installierbaren Modulen
besteht. Neben den bereits vorhandenen Modulen können von externen Entwicklern Module
über die Slicer Erweiterung entwickelt und bereitgestellt werden. Um eine
Weiterentwicklung möglich zu machen hat Slicer eine Reihe von Abhängigkeiten, die
jedoch portabel gehalten werden. Auf der untersten Schicht sind die
Plattformspezifischen Anforderungen zu sehen, die Slicer erfüllen soll. So kommt
es, dass das 3D Slicer Ökosystem sich durch einige Kriterien auszeichnet, die es
besonders attraktiv für die Bearbeitung von medizinischen Bilddaten machen. Zu den
wichtigsten Vorteilen gehört die Tatsache, dass die Software kostenfrei
verfügbar ist. Darüber hinaus bietet sie eine umfassende Plug-in-Infrastruktur,
die über den sogenannten Extension Manager zugänglich ist. Dies ermöglicht eine einfache
Erweiterung der Funktionalitäten nach Bedarf. Ein weiteres herausragendes
Merkmal ist die Möglichkeit, Skripte direkt in der integrierten Python-Konsole auszuführen,
was eine flexible und effiziente Automatisierung von Prozessen ermöglicht. Schließlich
ist 3D Slicer besonders für seine Vielseitigkeit bekannt, da es medizinische
Bilddaten aus sämtlichen Bereichen der Medizin – von Kopf bis Fuß – verarbeiten
kann.

3D Slicer hat für alle diese Punkte jeweils eine Lösung entwickelt, wobei der erste
Punkt durch die Open-Source-Philosophie schon gegeben ist. Die folgenden
Abschnitte decken diese Lösungen ab und bilden so eine erste Grundlage für die
Entwicklung mit 3D Slicer.
% ---------------------------------------------------------------------------------------

\section{Plug-in-Infrastruktur}
Der wohl bedeutendste Punkt ist die Plug-in-Infrastruktur, welche Slicer von sich
aus mitbringt. Um dieses Konzept genauer zu beleuchten, teilt man die Plattform am
besten in zwei Teile auf, die Kernanwendung und die Module, welcher jeder Anwender
personalisiert installieren oder deinstallieren kann. Diese Module werden als
\textit{Slicer lodabel module} bezeichnet \citep[vgl.][S.~1332]{fedorov2012slicer}.
Slicer realisiert die Struktur durch den \textit{Extension Manager}, welcher durchaus
vergleichbar ist mit einer Art App-Store. Über diesen können bequem und mit
wenig Klicks die gewünschten Erweiterungen in das Kernsystem installiert werden.
Neben der Möglichkeit Module zu installieren bietet Slicer noch die Möglichkeit eigenen
Module zu bauen und Sie im \textit{Extension Manager} zu veröffentlichen. Diese werden
als \ac{SEM} bezeichnet. Abbildung \ref{fig:3d_slicer_extension_index} soll diesen
Vorgang verdeutlichen \citep[vgl.][]{slicer2024}.

\begin{figure}[h]
	\centering
	\includegraphics[width=0.7 \textwidth]{img/slicer_extention_index.png}
	\caption{Funktionsweise der Plug-in-Infrastruktur von 3D Slicer nach \citet{extensionsIndex2024}}
	\label{fig:3d_slicer_extension_index}
\end{figure}

Slicer realisiert dies, indem die Plattform über ein zusätzliches Repository
verfügt, dass sich \href{https://github.com/Slicer/ExtensionsIndex?tab=readme-ov-file}{\textit{ExtensionIndex}}
nennt. Dieses öffentliche Repository ist eine Auflistung aller \ac{SEM}. Die
Auflistung erfolgt durch eine Reihe an \ac{JSON} Dateien, die auf die
Repositorien der einzelnen Entwickler verweisen. Dieser \href{https://github.com/Slicer/ExtensionsIndex?tab=readme-ov-file}{\textit{ExtensionIndex}}
ist über die Slicer Factory an den Extention Server und damit auch an den Extention
Manager angebunden. Die Slicer Factory ist ein System, das aus einem Slicer Extention
Repository ein lauffähiges Build erstellt, welches in den Extention Katalog
eingebunden werden kann. Ist eine Erweiterung in dem Erweiterungskatalog gelistet,
so sorgt der \textit{Extension Manager} dafür, dass die von der Slicer Factory
erstellt Build-Datei installiert werden kann.

Während die Module von Slicer gebaut werden, werden parallel auch immer die Tests
für die Kernanwendung ausgeführt. Diese sind über ein Dashboard einsehbar. So wird
sichergestellt, dass keines der Module einen Fehler im Kernsystem verursacht. Außerdem
ist so für alle Benutzer von Slicer transparent einsehbar, in welchem Zustand
sich die Software gerade befindet. Die Kernanwendung von 3D Slicer folgt einem Entwurfsmuster,
dass sich \ac{MVC} nennt. Bei der Erstellung einer \ac{SEM} soll dieser Ansatz ebenfalls
gepflegt werden. Eine High Level Betrachtung der Softwarearchitektur von 3D
Slicer bietet \cite[S.~1332]{fedorov2012slicer} mit der Abbildung
\ref{fig:3d_slicer_architektur}.

\begin{figure}[h]
	\centering
	\includegraphics[width=0.7\textwidth]{img/3d_slicer_architektur.jpg}
	\caption{3D Slicer High Level Architektur nach \citet[S.~1332]{fedorov2012slicer}}
	\label{fig:3d_slicer_architektur}
\end{figure}

Das Zusammenspiel zwischen \ac{MRML}, \ac{GUI} und der Logik bilden das MVC-Pattern
in der Kernanwendung. Das identische Pattern spiegelt sich auch in den einzelnen
Modulen von Slicer wieder. So wird sichergestellt, dass ein Softwareentwicklungsparadigma
eingehalten wird, was sich \textit{separation of concerns} nennt. Die Kapselung
von zusammengehöriger Logik. Bei der Erstellung einer eigenen Erweiterung ist die
Idee, dass nur die Logik implementiert werden muss und die komplexe Architektur
von Slicer erstmal nicht relevant ist.

Jedoch bietet sich in Slicer nicht nur die Möglichkeit eigene Erweiterungen zu
erstellen, es lässt sich hierfür auch die integrierte Python-Konsole nutzen.
% ---------------------------------------------------------------------------------------

\section{Python-Umgebung}
\label{subsec:pythob_umgebung} 3D Slicer bringt eine integrierte Python-Konsole mit,
über die mit der Datenstruktur interagiert werden kann. So ist es möglich,
Python- Skripte direkt in der Konsole auszuführen. Um dies zu realisieren, bringt
Slicer mit der Installation im jeweiligen Betriebssystem eine eigene Python-Umgebung
mit. Dieses sieht wie folgt aus.

\begin{center}
	\texttt{./Slicer/bin/PythonSlicer}
\end{center}

Diese Python-Umgebung verfügt über alle notwendigen Abhängigkeiten und Pakete.
Bei der Entwicklung eines \ac{SEM} kann dann auf das \ac{PyPi} in der
integrierten Python-Umgebung zurückgegriffen werden. So kommt es, dass für eine
Entwicklung mit Slicer. keine eigene Python-Umgebung auf der lokalen Maschine installiert
sein muss. Slicer bringt hier alles mit.

Für den letzten charakteristischen Punkt von Slicer aus Kapitel
\ref{sec:3d_slicer} führt der nächste Abschnitt in die durchaus komplexe
Datenstruktur \ac{MRML} ein, die bei einer Entwicklung mit Slicer unausweichlich
zu berücksichtigen ist.
% ---------------------------------------------------------------------------------------

\section{MRML-Datenstruktur}
\label{subsec:mrml_datenstruktur} Die \ac{MRML}, gesprochen \textit{"Murlm"} ist
ein Datenmodell, das dafür entwickelt wurde, alle möglichen Bilddaten zu visualisieren
und zu speichern, die für einen klinischen Zweck Einsatz finden \citep[vgl.][]{slicer2024}.
Laut \citet{slicer2024} wurde die \ac{MRML}-Datenstruktur völlig unabhängig von
der Slicer Kernanwendung entwickelt. Dies ermöglicht ein Portieren der Datenstruktur
auf andere Softwareapplikationen. Da Slicer die einzig große Plattform ist, die diese
Datenstruktur nutzt, wird der Quellcode für \ac{MRML} im Repository von 3D Slicer
gewartet und weiterentwickelt, so \citet{slicer2024}. Durch den Artikel von
\citet[S.~1331]{fedorov2012slicer} wird klar, dass \ac{MRML} mehr ist also nur eine
Datenstruktur. Sie beschreiben \ac{MRML} als Szenenorganisator von Bilder,
Annotationen, Layouts und Anwendungsdaten. \citet[S.~1327]{fedorov2012slicer} beschreiben
die \ac{MRML}-Datenstruktur als Schlüsselkomponenten innerhalb von 3D Slicer. Dies
ist auf die Softwarearchitektur von Slicer zurückzuführen, die in Abbildung
\ref{fig:3d_slicer_architektur} beschrieben wurde. Die Kernanwendung von Slicer arbeitet
wie bereits beschrieben nach dem \ac{MVC}-Pattern. \ac{MRML} übernimmt hier den
Teil des \textit{Models (M)} und bildet damit den Grundstein der Anwendung \citep[vgl.][S.~1332]{fedorov2012slicer}.
\citet{slicer2024} und der Artikel von \citet[S.~1327]{fedorov2012slicer} beschreibt
\ac{MRML} als \ac{XML}-Format. Wird also eine \ac{MRML}-Szene gespeichert, so
folgt eine Speicherung als \ac{MRML}-Datei und damit unter der Haube als \ac{XML}-Datei.
Dabei wird laut \citet{slicer2024} nur eine Referenz auf das Bild gespeichert. Die
zu bearbeitende Aufnahme selbst wird nicht innerhalb einer \ac{MRML}-Datei
abgespeichert. \ac{MRML} zeichnet sich vor allem dadurch aus, dass es eine Vielzahl
an Dateiformaten akzeptiert. Alle Formate, die für einen klinischen Zweck verarbeitet
werden, können durch \ac{MRML} unterstützt werden. Um dies zu gewährleisten, ist
die \ac{MRML}-Szene in sogenannte Knoten (engl.: nodes) aufgeteilt. Die Basistypen
für einen Knoten folgen der \citet{slicer2024} und sind in der folgenden
Aufzählung zu sehen.

\begin{minipage}{0.45\textwidth}
	\begin{itemize}
		\item Data node

		\item Display node

		\item Storage node

		\item View node
	\end{itemize}
\end{minipage}
\hfill
\begin{minipage}{0.45\textwidth}
	\begin{itemize}
		\item Plot node

		\item Subject hierarchy node

		\item Sequence node

		\item Parameter node
	\end{itemize}
\end{minipage}

Wird also ein Bild in eine \ac{MRML}-Szene geladen, so speichert Slicer die unterschiedlichen
Eigenschaften eines Bildes in unterschiedlichen Knoten. So werden Beispielsweise
Basiseigenschaften einer Probe im \textit{Data node} gespeichert, wo hingegen
ein \textit{Storage node} beschreibt wie ein Bild auf der Festplatte gespeichert
wird. In \textit{Display node} werden die Eigenschaften zur Darstellung eines Bildes
hinterlegt. Der Hintergrund für die Speicherung von Probendaten in
unterschiedlichen Knoten ist, dass beispielsweise dasselbe Bild in unterschiedlichen
Formaten vorliegt oder ein und dasselbe Bild auf zwei unterschiedliche Arten visualisiert
werden soll. So kann sich beispielsweise eine Struktur wie in Abbildung
\ref{fig:3d_slicer_class} ergeben.

\begin{figure}[h]
	\centering
	\includegraphics[width=1\textwidth]{img/slicer_class_index.jpg}
	\caption{Verkettung der einzelnen Knoten in der MRML-Datenstruktur nach
	\cite{slicer2024}}
	\label{fig:3d_slicer_class}
\end{figure}

Die Informationen in einem Bild werden also über diese Typen aufgeteilt und je nach
Sinn abgespeichert. Möchte man demnach auf die bestimmte Informationen in einer
Probe zugreifen. So kann diese Information über den Aufruf bestimmter Methoden erfolgen

\begin{lstlisting}[
	language={python},
	caption={Auslesen der Informationen aus den verschiedenen Knoten},
	label={lst:_auslehsen_nodes}]
# data node - vtkMRMLVolumeNode
currentVolume.GetImageData()
# storage node - vtkMRMLStorableNode
currentVolume.GetStorageNode()
# display node - vtkMRMLDisplayableNode
currentVolume.GetDisplayNode()
\end{lstlisting}

Wie die Kommentare in Listing \ref{lst:_auslehsen_nodes} bereits zeigen, gibt es
noch eine Besonderheit von \ac{MRML}. Damit eine Verwaltung aller Dateiformate
möglich ist, bedient sich \ac{MRML} einiger Tools, die sich bereits etabliert haben.
Die Wichtigsten sind hierbei \ac{VTK} und \ac{ITK} \citep[vgl.][K.~1.1]{vtk2006},
\citep[vgl.][K.~1.1]{itkguide2015}. Die beiden Tools sind echte Riesen in ihrer
Branche. \ac{MRML} nutzt diese, um einige Dateiformate zu lesen und zu schreiben.
Durch das Betrachten der \ac{MRML}-Szene wird klar, dass Slicer hierdurch viele
Möglichkeiten bietet. Speziell für die effiziente Speicherung der Proben in einer
Szene durch die unterschiedlichen Knotentypen. Ein besonderer Knoten, der gleichzeitig
auch die Brücke zu der interaktiven Benutzerschnittstelle von Slicer baut, ist
der \textit{Parameter node}. Warum dieser eine zentrale Rolle spielt und wie
Slicer die Schnittstelle grundsätzlich gestaltet, soll in Kapitel \ref{subsec:benutzerschnitstelle}
Benutzerschnittstelle diskutiert werden.

Mit dem Ende dieses Kapitels wurden alle wichtigen theoretischen Grundlagn beaprochen
die notwendig sind um die anatomische Segmentierung über ein 3D Slicer Modul bereitzuetllen.
Bevor die konkrete Methodik thematisiert wird, die für die Entwicklung des Moduls
angewendet wird, sollen im nächsten Kapitel kurz die Rahmenbedingungen gestzten
werden, innerhalb deren entwickelt werden soll.
% ---------------------------------------------------------------------------------------