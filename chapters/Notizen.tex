\section{Literaturrecherche}
\label{sec:recherche} Für die Recherche zu den verschiedenen Teilaufgaben ist
die Dokumentation der Open Source Plattform 3D Slicer eine wichtige Ressource
\cite[vgl.]{[}]{slicer2024}. Diese zeigt bereits etablierte Verfahren und einen \textit{Best
Practise} Ansatz. Auch das \citet{extensionsIndex2024} Repository ist eine
wichtige Quelle, da so ein Einblick in andere Lösungen möglich ist. So kommt es,
das nach ausführlicher Recherche zu den Teilaufgaben UI-Design, Pseudo-Extension,
Kapselung Hoffmann, Speicherung Parameter, Dokumentation und Test bereits Lösungen
existieren. Bei diesen Lösungen handelt es sich jedoch nicht um konkrete
Ergebnisse, sondern vielmehr um einen Leitfaden zur Lösung der Teilaufgabe. Die
Recherche hat demnach ergeben, dass diese Teilprobleme im Kontext der 3D Slicer Umgebung,
nicht das erste Mal zutage treten und Lösungswege existieren.

\begin{minipage}{0.40\textwidth}
	Für ein \textbf{UI Design} wird sehr empfohlen, bereits etablierte 3D Slicer Extensions
	als Orientierung zu nutzen. Eine sehr gute Orientierung bietet das Modul
	\textit{Transforms}, das in Abbildung \ref{fig:module_example} zu sehen ist. Zu
	Erkennen ist, dass die UI mittels Collapsible Buttons in verschiedenen Gruppen
	unterteilt wird. Ohne in die verschiedenen Gruppen hineinzublicken, lässt sich
	gut abschätzen, welche Parameter wo zu erwarten sind. Dies ermöglicht dem
	Benutzer ein isoliertes Betrachten der unterschiedlichen Funktionen in diesem
	Modul und so eine gute Benutzerfreundlichkeit.
\end{minipage}
\hfill
\begin{minipage}{0.50\textwidth}
	\centering
	\includegraphics[scale=0.50]{img/modul_example.jpg}
	\captionof{figure}{Das Modul Transforms als Beispiel einer etablierten UI für eine Slicer Extension | Screenshot aus 3D Slicer}
	\label{fig:module_example}
\end{minipage}

Für die Extension, welche in der vorliegenden Arbeit erstellt werden soll, wird genau
dieser Ansatz gepflegt und somit eine gute Benutzbarkeit des Moduls gewährleistet.
Die speziellen wünsche der konkreten Benutzergruppe sollen jedoch nicht zu kurz
kommen.

Für das Erstellen einer \textbf{ersten funktionierenden Extension} bietet Slicer
eine sehr gute Hilfe. 3D Slicer hat hierfür ein eigenes Modul entwickelt, das sich
\textit{Extention Wizard} nennt. Dieses Modul gibt eine gute Einführung in die Entwicklung
mit Slicer. Hiermit lässt sich mittels Leitfaden eine erste Demo Extension
erstellen, die sich bereits gut in 3D Slicer einfügt. Diese Lösung könnte als
Abstraktionsschicht betrachtet werden, da durch dieses Modul im ersten Schritt nahe
zu keine Kenntnisse über die Kernanwendung von Slicer nötig sind. Der
Extentionwizard ist wie folgt zu finden:

\texttt{Modules -> Developer Tools -> Extension Wizard}

Für \textbf{die Kapselung} einer bestimmten Einheit von Code, sieht Slicer eine
Bibliothek innerhalb der Extention vor, so beschreibt es die \citet{slicer2024}.
Das Listing \ref{lst:3d_slicer_projektverzeichnis} zeigt, wo eine Bibliothek
innerhalb einer Extention einzuordnen ist, hier als \texttt{MyLib} zu sehen. Innerhalb
dieses Ordners können sich weitere Module befinden, die als einfache Python-Files
abgelegt werden. Zu beachten ist, dass in jeder Bibliothek eine \texttt{\_\_init\_\_.py}
hinzugefügt wird, sodass dieser Ordner auch entsprechend verwendet werden kann.

\begin{lstlisting}[
    language={python},
    caption={Prinzipeller Aufbau eines Projektes für eine Slicer Extension nach Slicer (2024)},
    label={lst:3d_slicer_projektverzeichnis}]
|-- CMakeLists.txt
|-- MyLib
|   |-- __init__.py
|   |-- cool_maths.py
|   |-- utils.py
|-- MyExtension.py
\end{lstlisting}

Für die Teilaufgabe zur \textbf{Speicherung der Parameter} nutzt Slicer eine
Technik, das durchaus als eines der Kernfunktionen beschrieben werden kann. Die
Rede ist hier von der Klasse \texttt{ParameterNodeWrapper}. Dieser wurde bereits
zu einem früheren Zeitpunkt in dieser Arbeit beschrieben. Für die Funktion dieser
Lösung sei auf den Abschnitt \ref{subsec:benutzerschnitstelle} Benutzerschnittstelle
verwiesen.

Sowohl das \textbf{Benutzerhandbuch}, also auch die technische Dokumentation einer
Slicer Extension wird immer in der \texttt{README.md} des jeweiligen Repository
hinterlegt. In der Extension selber sieht Slicer keine umfangreiche Dokumentation
vor. Es wird lediglich via Link auf die Dokumentation im Repository verwiesen
und eine kurze Einführung gegeben. Auch hier gibt es wieder etablierte Designs,
die als Vorlage verwenden werden können. Die \citet{slicer2024} erwähnt hier
unter anderem das Module
\href{https://github.com/lassoan/SlicerSegmentMesher}{SegmentMesher} als gutes
Beispiel.

Für die letzte Teilaufgabe, das \textbf{Testen}, gibt die Dokumentation von 3D Slicer
ebenfalls eine konkrete Struktur vor. Ist also nicht nötig eigenen Verfahren zu
entwickeln. Für die Softwaretests stell Slicer eine eigene Klasse innerhalb der Slicer
Bibliothek bereit, die alle wichtigen Funktionen enthält, um konkrete Testfälle zu
schreiben. Die Klasse trägt den Namen \textsl{ScriptedLoadableModuleTest} und kann
als Elternklasse für einen Testfall verwendet werden.

\begin{lstlisting}[
    language={python},
    caption={Aufbau einer Testklasse zum ausführen von Unittest nach \citet{slicer2024}},
    label={lst:3d_slicer_test_class}]
class MyExtensionTest(ScriptedLoadableModuleTest)
    def setUp(self):
      # do something befor a test
    def runTest(self):
      # execute your test case here
    def testMyExtension1(self):
      # write your test case here
\end{lstlisting}

Im Listing \ref{lst:3d_slicer_test_class} ist zu sehen, dass sich die Klasse in drei
Teil aufteilt, die Methode \texttt{setUp()}, die als Vorbedingung dient, die Methode
\texttt{runTest()}, die über die UI getriggert werden kann, um die Test zu
starten und die konkreten Textfälle, die in selbst definierten Methoden geschrieben
werden können. Als Beispiel sie hier die Methode \texttt{testMyExtension1()} gezeigt.

Im Laufe dieses Kapitels wurde klar, dass es für einige der Teilaufgaben bereits
Lösungen oder Lösungsansätze gibt, die zu etwas weniger Entwicklungsarbeit führen.
Jedoch trifft dies nicht auf alle Punkte zu, sodass sich der nächste Abschnitt mit
der Erarbeitung von konkreten Lösungsansätzen für die noch nicht behandelten Teilaufgaben
beschäftigt.
% ---------------------------------------------------------------------------------------

\section{Erarbeiten von Lösungsansätzen}
\label{sec:lösungsansätze} hier geht es um Brainstorming

\textbf{Architekturdesign}

\textbf{Single Prozess}

\textbf{Batch Prozess}

% ---------------------------------------------------------------------------------------
\subsection{Benutzerschnittstelle}
\label{subsec:benutzerschnitstelle} Für das Erstellen einer \ac{UI}, die für
eine Slicer Erweiterung notwendig ist, nutzt 3D Slicer den Qt-Designer \citep[vgl.][]{qt2024}.
Die Integration des Qt-Designers als Applikation in eine andere Applikation funktioniert
aufgrund der Plattformintegrität, die der Designer mitbringt \citep[vgl.][]{qt2024}.
Diese bietet so die Möglichkeit die benötigten Widgets über eine interaktive
Benutzerschnittstelle zu bauen. Für diese \ac{UI}-Vorrichtung gibt es einen
Gegenspieler im Quelltext des Programmes, welcher als \textit{ParameterNode}
bekannt ist. Der \textit{ParameterNode} ist laut \citet{slicer2024} eine leichte
Variante eines \ac{MRML}-Knoten um Parametereinstellungen zu speichern. Durch das
Zusammenspiel zwischen \ac{UI} und \textit{ParameterNode} wird die \ac{UI}
automatisch aktualisiert, wenn sich das Programm ändert \citep[vgl.][]{slicer2024}.

\begin{minipage}{0.35\textwidth}
	Das Erstellen der Verknüpfung zwischen \ac{UI}-Widget und \textit{ParameterNode}
	erfolgt über die dynamische Eigenschaft \texttt{SlicerParameterName}, die direkt
	in der Komponentenansicht im Qt-Designer einstellbar ist. Die Abbildung
	\ref{fig:qt_designer} soll diesen Vorgang verdeutlichen. Dabei ist es wichtig,
	dass genau diese Eigenschaft auch verwendet wird. Diese Verknüpfung lässt sich
	laut \citet{slicer2024} auch via Programmcode setzten.
\end{minipage}
\hfill
\begin{minipage}{0.55\textwidth}
	\centering
	\includegraphics[width=0.5\textwidth]{img/qt_designer.png}
	\captionof{figure}{Komponentenansicht im Qt-Designer nach der \citet{slicer2024}}
	\label{fig:qt_designer}
\end{minipage}

Über das Objekt \texttt{widget} kann die Eigenschaft einer Komponente gesetzt
werden, ohne dass sie im Designer berührt werden muss. Das nachfolgende Beispiel
zeigt dies genauer.
\begin{center}
	\texttt{widget.setProperty('SlicerParameterName', 'parameterName')}
\end{center}
Mit dem Ende dieses Abschnittes wurden alle wichtigen Bestandteile von 3D Slicer
abgedeckt und diskutiert, sowie alle weiteren Domänen eingeführt. So bleibt nun die
Frage nach dem Sinn dieser Arbeit. Das Kapitel \ref{chap:fragestellung} soll
hier Klarheit liefern und die konkrete Fragestellung ausarbeiten.
% ---------------------------------------------------------------------------------------

\chapter{Einleitung}
\label{chap:einleitung} Die \ac{CT} hat die Medizintechnik revolutioniert und
ist bis heute eines der wichtigsten Methoden für die Bildanalyse. Sie ist eine der
führenden Erweiterungen der klassischen Röntgentechnik. Für die Entwicklung
dieser Technologie wurden Godfrey Newbold Hounsfield und Allan McLeod Cormack im
Jahre 1979 mit dem Nobelpreis für Medizin ausgezeichnet \citep[.vgl][S.~12]{handels2000}.

\begin{minipage}{0.45\textwidth}
	Die Computertomografie wird in den verschiedensten Bereichen und im wahrsten Sinne
	des Wortes von Kopf bis Fuß eingesetzt. So kommt es, dass auch im Dentalbereich
	\ac{CT}-Aufnahmen von größter Wichtigkeit sind. Abbildung \ref{fig:ct_aufnahme_eines_zahns}
	zeigt eine solche \ac{CT}-Aufnahmen. Eine konkrete Anwendung in diesem Kontext
	ist die Zahnkaries Forschung der Poliklinik für Zahnerhaltung und Parodontologie
	der \ac{LMU}.
\end{minipage}
\hfill
\begin{minipage}{0.45\textwidth}
	\centering
	\includegraphics[scale=0.2, width=\textwidth]{img/micro_ct_orginal.jpg}
	\captionof{figure}{CT-Aufnahme eines Zahns nach \citet{heck2024}} \label{fig:ct_aufnahme_eines_zahns}
\end{minipage}

Die vorliegende Arbeit soll genau diese Forschung unterstützen. In welchem Umfang
und zu welchem Grund ist in den folgenden Abschnitten beschrieben.
% ---------------------------------------------------------------------------------------

\section{Ziel der Arbeit}
\label{sec:ziel_der_arbeit} Diese Arbeit beschreibt eine Technik, mit der \ac{3D}
Mikro-\ac{CT}-Bilder zur Untersuchung zahnmedizinischen Strukturen automatisch
mittels der Software 3D Slicer segmentiert und analysiert werden können. Was genau
unter eine Segmentierung verstanden wird, darüber informiert das Kapitel
\ref{subsec:segmentierung} Segmentierung. Die algorithmische Formulierung einer konkreten
Segmentierung ist bereits vorhanden und prototypisch implementiert. Dieser
Algorithmus hat jedoch Schwachstellen. So muss beispielsweise das Verfahren umständlich
über ein IPython Notebook im Terminal ausgeführt werden, was die
Benutzerfreundlichkeit deutlich beeinträchtigt. Ziel dieser Arbeit ist es in erster
Linie das bereits existierende Verfahren in der Klinik für Zahnerhaltung zu analysieren
und für die Mitarbeiter der Klinik benutzbar zu machen. Dabei soll auf
etablierte und vertraute Lösungen zurückgegriffen werden.

Es stellt sich nun die Frage, zu welchem Zweck eine automatische und interaktive
Segmentierung überhaupt notwendig ist. Für die Zahnklinik an der LMU in München
gibt es hierfür viele Gründe. Über den wichtigsten gibt das nächste Kapitel Aufschluss.
% ---------------------------------------------------------------------------------------

\section{Relevanz der Arbeit}
\label{sec:relevanz_der_arbeit} Der wohl relevanteste Punkt wurde bereits im vorherigen
Kapitel \ref{sec:ziel_der_arbeit} diskutiert, Zahnärzte sind keine
Softwareentwickler, sondern reine Anwender von Software. Darüber hinaus verfolgt
die Klinik für Zahnerhaltung und Parodontologie der \ac{LMU} einen sehr interessanten
Forschungsansatz, welche eine Segmentbetrachtung der \ac{CT}s rechtfertigt.

Über viele Jahre hinweg wurden in der Zahnklinik sehr viel Bilddaten von Zähnen
gesammelt. Hierbei wurden Aufnahmen der unterschiedlichsten Arten gemacht.
Darunter fallen zum Beispiel einfache Bilddateien, Infrarotbilder und die für diese
Arbeit so relevanten dreidimensionalen Mikro-CT-Aufnahmen. Dieser große Schatz
an Bildmaterial soll verwendet werden, um in ferner Zukunft ein neuronales Netzwerk
zu trainieren, welches statistische Aussagen über das Verhalten von Karies
treffen kann. Jedoch gibt es hier ein Problem, bei dem das Ergebnis dieser
Arbeit unterstützen kann. Karies auf \ac{CT}-Bildern zu lokalisieren ist nicht
trivial. Er ist ohne weitere Bearbeitung des Bildes nur sehr schwer auf eine Stelle
einzugrenzen. So kommt es vor, dass drei verschiedene Ärzte auf demselben Mikro-\ac{CT}-Bild
drei unterschiedliche Stellen mit Karies identifizieren. Eine Segmentierung des dreidimensionalen
\ac{CT}s kann hier Wunder wirken. Durch die Aufteilung des Mikro-\ac{CT}s in
seine zwei Zahnhauptsubstanzen, kann eine sehr gute visuelle Darstellung des Zahnes
gewährleistet werden. Für Ärzte bietet diese Darstellung einen sehr großen
Mehrwert \citep[vgl.][S.~1]{walter2025projekt}.

Mit dieser klaren und eindeutigen Identifizierung von Karies, sind die
Ergebnisse, die ein neuronales Netzer generieren würde viel genauer und brauchbarer.
Konkret wird mit einer automatischen Segmentierung ein \textit{Ground Trueth} gewonnen,
der eine eindeutige Basiswahrheit liefert. Hierbei sei gesagt das diese Anwendung
nur eine von vielen Möglichkeiten ist. Konkrete Daten über die Ausbreitung einer
Krankheit im menschlichen Körper zu besitzen kann in den verschiedensten Fällen
und Institutionen von größtem Nutzen sein. So zeigen es auch \citet[S.~207]{de20083d}
in ihrem Paper.

Anhand dieser Argumente wird deutlich, dass eine automatische Segmentierung durchaus
einen Mehrwert für Ärzte bilden kann. Nicht zuletzt auch durch die enorme
Zeiteinsparung. Für eine automatische Segmentierung von Mikro-\ac{CT}-Bildern
gibt es einige Softwarelösungen am Markt, die alle eine gute Optionen sind. Aus
diesem Grund soll im folgenden Kapitel ein mögliches Framework diskutiert werden.
% ---------------------------------------------------------------------------------------

\section{Fokus der Arbeit}
\label{sec:fokus_der-arbeit} Dieser Arbeit setzt den Fokus auf die Open-Sorce-Plattform
3D Slicer, da diese ohnehin bereits eine breite Anwendung in der Zahnklinik in München
findet. Durch die Modul- und Plug-in-Infrastruktur dieser Plattform kann die Software
auch anderen Institutionen bereitgestellt werden. Hierzu muss diese einfach als
\textit{3D Slicer Extension} bereitgestellt werden. 3D Slicer bietet einen
\textit{Extension Manager}, der ähnlich wie ein App Store betrachtet werden kann.
So bleibt die vorerst konkret entwickelte Software nicht nur einer Einrichtung vorbehalten.
Eine tiefere Einführung in die Open-Source-Plattform bietet der Abschnitt
\ref{sec:3d_slicer}. Das weitere Optimieren des bereits bestehenden Verfahrens wird
in dieser Arbeit nicht thematisiert. Es werden lediglich Anpassungen vorgenommen,
sodass eine Benutzerschnittstelle verwendet werden kann.

Mit diesem Umfang, der Motivation und dem gesetzten Fokus, ergibt sich für diese
Arbeit eine konkrete Struktur, die einen hohen Detailgrad aufweist. Um einen ersten
Überblick zu gewähren, sei diese Struktur hier kurz erläutert.
% ---------------------------------------------------------------------------------------

\section{Aufbau der Arbeit}
\label{sec:aufbau_der_arbeit} Die Arbeit ist in sieben Kapitel unterteilt. Nach der
Einführung in Kapitel \ref{chap:einleitung}, in der die Relevanz und der Fokus
beschrieben werden, werden in Kapitel \ref{chap:theoretische_grundlagen} die theoretischen
und technischen Grundlagen behandelt, welche zum Verstehen der Ergebnisse
essenziell sind. Als Ergebnis der theoretischen Grundlagen bildet das Kapitel \ref{chap:fragestellung}
eine konkrete Forschungsfrage. Während sich Kapitel \ref{chap:methodik} darum
kümmert mit welchen Methodiken und Lösungsansätzen an die Forschungsfrage
herangegangen wird, erläutert das Kapitel \ref{chap:ergebnisse} welche die konkreten
Ergebnisse der Arbeit sind. In Kapitel \ref{chap:diskussion} erfolgt eine
kritische Diskussion der Resultate. Das abschließende Kapitel \ref{chap:schlussfolgerung}
fasst die wichtigsten Erkenntnisse zusammen und gibt einen Ausblick auf zukünftige
Forschungsfragen.

Die theoretischen Grundlagen, die wie beschrieben direkt nach der Einleitung
folgen, sind zentral für das Verstehen der Fragestellung und der späteren Ergebnisse
der Arbeit.
% ---------------------------------------------------------------------------------------

\section{Anatomische Segmentierung von Mikro-CT-Bildern}
\label{sec:verwwandte_arbeit} Wie bereits in der Einleitung dieser Arbeit klar wurde
verfügt die Poliklinik für Zahnerhaltung und Parodontologie des \ac{LMU}-Klinikums
München über einen breiten Schatz an Bilddaten. Im Rahmen einer Bachelorarbeit an
der Hochschule für angewandte Wissenschaften in Augsburg unterstützte Herr Hofmann
die Verarbeitung dieser Bilddaten mit Methoden der 3D-Bildverarbeitung. Konkret
sollte diese Arbeit die Kariesklassifizierung unterstützen. Hierzu entwickelte er
ein Verfahren, das auf Basis adaptiver Schwellwertverfahren die Zahnsubstanzen
Schmelz und Dentin aus dem Originalbild herauslöst. Konkret kann diese Segmentierung
mit verschiedenen Verfahren durchgeführt werden. Man spricht hier von einer
anatomischen Segmentierung der Zahnkrone.

\begin{minipage}{0.40\textwidth}
	Durch die Segmentbetrachtung der beiden Zahnhauptteile Schmelz und Dentin konnte
	\citet[S.~41]{hoffmann2020} eine gute Hilfe für die Befundung kariöse Stellen
	liefern. Ein Ergebnis aus der Arbeit von Hofmann sei in Abbildung \ref{fig:ergebnis_hoffmann}
	gezeigt. \citet[S.~53]{hoffmann2020} entwickelte hierfür ein prototypisches Verfahren,
	mit dem es gelang ca. 250 Datensätze der Zahnklinik automatisch aufzubereiten.
\end{minipage}
\hfill
\begin{minipage}{0.50\textwidth}
	\centering
	\includegraphics[width=0.7\textwidth]{img/ergebnis_hoffmann_2.jpg}
	\captionof{figure}{Reproduzierte Ergebnisansicht der anatomischen Segmentierung}
	\label{fig:ergebnis_hoffmann}
\end{minipage}

Die anatomische Segmentierung des Zahnes umfasst eine ganze Reihe algorithmischer
Schritte, sodass sich eine Pipeline an Verarbeitungsschritten ergibt. Die Abbildung
\ref{fig:anatomische_segmentierung} zeigt den groben Ablauf des Verfahrens.
Kleinere Zwischenschritte wurden nicht berücksichtigt.

\begin{figure}[h]
	\centering
	\includegraphics[width=0.8\textwidth]{img/anatomischeSegmentierung.png}
	\caption{Algorithmische Formulierung der anatomischen Segmentierung nach
	\citet{hoffmann2020}}
	\label{fig:anatomische_segmentierung}
\end{figure}

\citet[S.~55]{hoffmann2020} beschreibt, dass dieses Verfahren bis zu einem gewissen
Fortschritt des Karies durchgeführt werden konnte, da der Algorithmus
diesbezüglich seine Grenzen hat. Außerdem ist das Verfahren für die ordinalen \ac{ISQ}-Bilder
erstellt worden, deren Daten im Format \ac{16Int} vorliegen. Für die spätere Darstellung
der Ergebnisse kann eine überlappende Ansicht in einer Visualisierungssoftware verwendet
werden. So ergibt sich die Situation, dass der Algorithmus ein gutes Ergebnis liefert,
jedoch nicht benutzerfreundlich zu bedienen ist. Für das Starten und visualisieren
des Verfahrens sind aufwendige Befehle über das Terminal zu tippen \citep[vgl.][S.~53]{hoffmann2020}.
Genau an dieser Stelle soll die vorliegende Arbeit anknüpfen und das Verfahren der
anatomischen Segmentierung so interaktiv und benutzerfreundlich gestalten.

Für eine interaktive Verarbeitung von 3D Bilddaten bieten sich einige Möglichkeiten.
Die wohl beste Lösung liefert 3D Slicer. Warum die Wahl auf diese Plattform fiel
und welche Vorteile daraus entstehen wird im folgenden Abschnitt \ref{sec:3d_slicer}
erläutert.
% ---------------------------------------------------------------------------------------

TODO:

- Kurzfassung überarbeiten : DONE

- Diskussion überarbeiten, mehr auf die Forschungsfrage eingehen

- Das Kapitel theoretische Grundlagen umbauen

- Benutzerschnittstelle kommt in das Kapitel Ergebnisse : DONE

- Überleitung von MRML-Datenstruktur auf Das Kapitel Methodik : DONE

- passt das Kapitel Forschungsevaluation zu den tatsächlichen Evaluationen

Diese Seiten muss Sara nochmal lesen: