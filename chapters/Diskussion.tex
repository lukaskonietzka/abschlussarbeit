\chapter{Diskussion und Fazit}
\label{chap:diskussion} Betrachtet man gegen Ende dieser vorliegenden Arbeit nochmals
die Forschungsfrage aus Kapitel \ref{chap:fragestellung} so können noch weitere Punkte
hinzugefügt werden. Grundsätzlich lässt sich sagen, dass die Integration der anatomischen
Segmentierung aus Kapitel \ref{sec:verwwandte_arbeit} gut in das Ökosystem von Slicer
integriert werden konnte. Somit ergibt sich die Situation, dass das zuvor aufwendig
zu bedienenden Verfahren nach dieser Arbeit mit nur wenigen Klicks ein Ergebnis liefert.
Bei der Integration dieses Verfahrens mussten keine Abstriche in Bezug auf die
Qualität des Ergebnisses gemacht werden. Die Benutzerfreundlichkeit der Software
konnte so um ein vielfaches verbessert werden. Es steigt auch der Benutzerkreis
der Anwendung. Der Entwicklungsprozess selbst kann ebenfalls als durchwegs positiv
betrachtet werden. Durch die sehr gute Dokumentation von 3D Slicer konnten
schnell erste Ergebnisse erzielt werden, die noch nicht final wahren, jedoch in
die richtige Richtung gingen. Es lässt sich so sagen, dass einfache Algorithmen
ohne viel zusätzliche Abhängigkeiten, mit etwas Vorwissen schnell in einem Slicer
Modul integriert werden können. Es ist auch nicht zwingend notwendig, ein Modul über
den \textit{Extension Manager} zu veröffentlichen. Soll das Modul nur wenigen ausgewählten
Personen zur Verfügung stehen, so kann es auch lokal in die Slicer Anwendung
eingebunden werden. Betrachtet man die Erweiterbarkeit der Software, so kann
auch diese als weites gehend erfolgreich angesehen werde. Durch die Kapselung des
zugrunde liegenden Verfahrens lässt sich dieses einfach durch ein anderes austauschen.
Auch das Hinzufügen neuer Funktionalität lässt sich problemlos realisieren. Es sei
jedoch gesagt, das zu viele unterschiedliche Funktionen zur Überladung des
Moduls führen könnte. Zu empfehlen ist dann eine Aufteilung in Submodule. Bei
der technischen Umsetzung dieser Ansprüche konnten jedoch nicht alle Prinzipien
einer sauberen Softwareentwicklung gewährleistet werden. Diese Entscheidung war jedoch
aktiv und hat auch positive Auswirkungen auf das Gesamtsystem. Abschließend
lässt sich also sagen, dass diese vorliegende Arbeit der zu Beginn gestellten Forschungsfrage
gerecht werden konnte und alle Anforderungen erfüllte. Betrachtet man jedoch die
Ziele dieser Arbeit aus Kapitel \ref{sec:ziel_der_arbeit} so stellt man fest, dass
der Tooth Analyser noch lange nicht den Reifegrad erreicht hat, der alle
Anforderungen abdeckt. Um dies endgültig zu gewährleisten, sind unter anderem
weitere Benutzertests und gegebenenfalls Fehlerbehebungen notwendig.
% ---------------------------------------------------------------------------------------