\chapter{Diskussion}
\label{chap:diskussion} Am Ende dieser Arbeit lohnt es sich, erneut einen Blick
auf die in Kapitel \ref{sec:ziel_der_arbeit} formulierte Forschungsfrage zu
werfen und noch einige Aspekte hinzuzufügen:

\textit{Wie kann eine benutzerfreundliche Schnittstelle innerhalb von 3D Slicer
entwickelt werden, die nicht nur das Verfahren der anatomischen Segmentierung effizient
integriert und den Zugang für Anwender vereinfacht, sondern auch eine
strukturierte und flexible Umgebung zur Verarbeitung von Mikro-CT-Aufnahmen
bietet?}

Die Ergebnisse dieser Arbeit zeigen, dass die Integration der anatomischen Segmentierung
erfolgreich in das Ökosystem von 3D Slicer eingebunden werden konnte. Das zuvor
komplexe und manuell aufwendige Verfahren wurde durch die entwickelte
Benutzeroberfläche erheblich vereinfacht. Anstelle einer umständlichen Terminal-Nutzung
kann die Segmentierung nun mit wenigen Klicks gestartet werden, was insbesondere
die praktische Anwendung in der Zahnmedizin erheblich erleichtert.

Ein zentraler Aspekt der Forschungsfrage betraf die Effizienz der Integration. Hier
konnte gezeigt werden, dass das entwickelte Modul zuverlässige Ergebnisse
liefert, ohne dabei die Qualität der anatomischen Segmentierung zu beeinträchtigen.
Durch die enge Verzahnung mit 3D Slicer bleibt das Modul kompatibel mit
bestehenden Workflows, sodass eine nahtlose Nutzung innerhalb der Klinik ermöglicht
wird. Mit dem Tooth Analyser entstand somit eine durchdachte Struktur für die
Verarbeitung von Mikro-CT-Aufnahmen, die einen bedeutenden Fortschritt für die zahnmedizinische
Forschung darstellt. Der Luxus einer interaktiven Schnittstelle ist dabei nur
die Spitze des Eisbergs. Des weitern konnte beobachtet werden, dass es neben der
Auführung einer ganzen Bearbeitungspipelin auch sinnvoll ist einzelne Schritte
in der Pipeline isoliert ausführen zu können. Auch ein Batch-Prozess, der eine
ganze Sammlung an Bilder auf einmal verarbeiten kann stellte sich als sehr
praktikabel herraus.

Neben der verbesserten Benutzerfreundlichkeit wurde auch die Effizienz des
Workflows gesteigert. Eine manuelle Segmentierung eines einzelnen Zahns nimmt durchschnittlich
rund 20 Minuten in Anspruch. Durch den Tooth Analyser kann dieser Prozess
vollständig automatisiert im Hintergrund ablaufen, wodurch wertvolle Zeit eingespart
wird. Laut Dr. Elias Walter reduziert sich der aktive Arbeitsaufwand dadurch erheblich,
da die Segmentierung nicht mehr manuell durchgeführt werden muss.

Ein Punkt, der zu Beginn der Entwicklung nicht feststand, war die optionale
Generierung der medialen Flächen sowie die automatische Erkennung bereits
gefilterter Bilder. Beide Funktionen sind das Ergebnis der Performancemessungen und
tragen zur Effizienzsteigerung der Anwendung bei. Die optionale Generierung
medialer Flächen ermöglicht es, die Berechnung gezielt an- oder auszuschalten, je
nach den Anforderungen des jeweiligen Anwendungsfalls. Dadurch wird verhindert, dass
unnötige Rechenoperationen ausgeführt werden, wenn diese für die Analyse nicht
erforderlich sind. Die automatische Erkennung bereits gefilterter Bilder sorgt dafür,
dass doppelte Verarbeitungsschritte vermieden werden. Anstatt ein Bild erneut durch
die gesamte Pipeline zu schicken, erkennt das System, ob eine vorherige Filterung
bereits erfolgt ist, und kann das Bild direkt für die nächste Verarbeitungsstufe
nutzen.

Auch die Erweiterbarkeit des Systems wurde berücksichtigt. Dank der modularen Architektur
kann das zugrunde liegende Segmentierungsverfahren ohne größere Anpassungen ausgetauscht
oder durch weitere Funktionalitäten ergänzt werden. Dies gewährleistet
langfristige Flexibilität und Anpassungsfähigkeit an zukünftige Entwicklungen. Gleichzeitig
zeigte sich jedoch auch, dass eine übermäßige anzahl an Funktionen im Modul zu
einer Überladung der \ac{UI} führen kann. Um dies in der Zukunft zu vermeiden, kann
die 3D Slicher-Erweiterung auf mehrer Module aufgeteilt werden, die einen
logischen und inhaltlichen Zusammenhang haben.

Zur objektiven Evaluation wurden Benutzertests mit mehreren Zahnärzten der
Klinik durchgeführt. Über einen Zeitraum von drei Wochen integrierten sie den Tooth
Analyser in ihren Forschungsalltag. Dabei gelang nicht nur die Segmentierung
einzelner Zahnkronen, sondern auch vollständiger Zähne mit komplexen Wurzeln.
Besonders bemerkenswert ist, dass die Software während der gesamten Entwicklung ausschließlich
mit CT-Aufnahmen von Zahnkronen getestet wurde. Dennoch konnte sie ohne
zusätzliche Anpassungen auch diese erweiterte Anwendungsform erfolgreich
bewältigen. Dies unterstreicht die Flexibilität des Systems und stellt einen bedeutenden
Erfolg dar. Grundsätlzich lässt sich auch zu den Benutzertest sagen, dass diese mit
einer der wichtigsten Testgrundlagen sind.

In einem weiteren Test wurde die Skalierbarkeit der Anwendung überprüft. Hierbei
wurden 103 Mikro-CT-Aufnahmen in einem Batch-Prozess über einen leistungsstarken
Server der LMU verarbeitet. Die Bearbeitungszeit pro Bild lag bei etwa neun Minuten,
sodass eine Gesamtzeit von etwa 15 Stunden und 27 Minuten prognostiziert wurde –
ein Wert, der sich in der Praxis bestätigte. Entscheidend war jedoch, dass alle Bilder
erfolgreich anatomisch segmentiert werden konnten.

Trotz der vielen positiven Ergebnisse zeigen sich auch einige Einschränkungen.
Die wesenstlichen wurden bereits im Kapitel Limitation erwähnt. So kommt es, das
die anatomische Segmentierung speziell für Mikro-CT-Aufnahmen im ISQ-Format
entwickelt wurde, wodurch bestimmte Parameter und Schwellwerte fest im Code hinterlegt
wurden. Dies kann die Flexibilität bei der Nutzung mit anderen Bildformaten
einschränken. Eine detaillierte Analyse der Segmentierungsergebnisse zeigte zudem
eine unerwartete Herausforderung im Bereich der Auffüllung. Zu Beginn der
Segmentierung werden die Zahnbestandteile, einschließlich der Pulpa, korrekt
erkannt. Während der anschließenden Auffüllung des Schmelzsegments kommt es jedoch
zu einer fehlerhaften Überschreibung bereits segmentierter Strukturen. Diese
wohl größte Limitierung des Tooth Aanalyser führt auf eine weitere, welche die Vorverarbeitung
von Bilder betrifft.

die Implementierung einer Vorverarbeitung der Bilder konnte bislang nicht umgesetzt
werden. Zwar ist diese Funktionalität bereits konzeptionell vorgesehen, jedoch bedarf
es noch weiterer Arbeit, um eine vollumfängliche Implementierung zu ermöglichen.
der Grund hierfür ist, dass innerhalb der Vorverarbeitung eine Komprimierung der
Bilder geplant war. Diese Komprimierung zieht jedoch auch einen Formatwechsel von
\ac{16Int} auf \ac{8UInt} nachsich. Da diese Bilder ohnehin nicht bearbeitet
weden können, wurde hierfür keine implementierung bereitgestellt.

Zusammenfassend lässt sich sagen, dass die in dieser Arbeit entwickelte Lösung
die Forschungsfrage in wesentlichen Punkten beantworten konnte: Die Segmentierung
wurde effizient in 3D Slicer integriert, die Benutzerfreundlichkeit erheblich
verbessert und die Softwarearchitektur flexibel gestaltet. Darüber hinaus wurde ein
strukturiertes Modell für die effiziente und interaktive Verarbeitung von Mikro-CT-Aufnahmen
entwickelt. Nichtsdestotrotz ist das System noch nicht in seinem finalen
Entwicklungsstadium. Insbesondere weitere Benutzertests und mögliche Anpassungen
der Segmentierungsparameter könnten zu einer noch höheren Präzision und Benutzerfreundlichkeit
beitragen. Auch die in Kapitel \ref{sec:relevanz_der_arbeit} diskutierte
Relevanz der Arbeit zeigt, dass der Tooth Analyser zwar bereits einen großen Fortschritt
darstellt, jedoch noch Optimierungen erforderlich sind, um eine langfristig etablierte
und klinisch einsetzbare Lösung zu erreichen.
% ---------------------------------------------------------------------------------------