\chapter{Diskussion und Fazit}
\label{chap:diskussion} Am Ende dieser Arbeit lohnt es sich, erneut einen Blick
auf die in Kapitel \ref{sec:ziel_der_arbeit} formulierte Forschungsfrage zu
werfen und noch einige Aspekte hinzuzufügen:

\textit{Wie kann eine benutzerfreundliche Schnittstelle innerhalb 3D Slicer
entwickelt werden, die das Verfahren der anatomischen Segmentierung effizient integriert,
den Zugang für Anwender vereinfacht und zugleich eine flexible Erweiterbarkeit
für zukünftige Funktionalitäten gewährleistet?}

Die Ergebnisse dieser Arbeit zeigen, dass die Integration der anatomischen Segmentierung
erfolgreich in das Ökosystem von 3D Slicer eingebunden werden konnte. Das zuvor
aufwendig zu bedienende Verfahren wurde durch die entwickelte Benutzeroberfläche
erheblich vereinfacht. Anstelle einer manuellen Ausführung über das Terminal lässt
sich die Segmentierung nun mit wenigen Klicks starten, was die Anwendung
besonders für Praktiker in der Zahnmedizin zugänglicher macht.

Ein zentraler Aspekt der Forschungsfrage betraf die Effizienz der Integration. Hier
konnte gezeigt werden, dass das entwickelte Modul die gewünschten Ergebnisse
liefert, ohne dabei die Qualität der anatomischen Segmentierung zu beeinträchtigen.
Durch die enge Verzahnung mit 3D Slicer bleibt das Modul kompatibel mit
bestehenden Workflows, wodurch eine nahtlose Nutzung innerhalb der Klinik ermöglicht
wird. Mit dem Tooth Analyser entstand so eine Struktur die ein gutes Vorgehen
bei der Verarbietung von Mikro-CT-Aufnhamen beschreibt, das für die gesamte Zahnmedizin
eine Bereicherung bietet. Der Luxus einer interaktiven Schnittstelle ist hier
nur die Spize des Eisberges.

Auch die Erweiterbarkeit des Systems wurde berücksichtigt. Dank der modularen Architektur
kann das zugrunde liegende Segmentierungsverfahren ohne größere Anpassungen ausgetauscht
oder durch weitere Funktionalitäten ergänzt werden. Dies stellt sicher, dass das
System auch in Zukunft flexibel bleibt. Es wurde auch deutlich, dass eine zu große
Funktionsvielfalt zu einer Überladung des Moduls führen könnte. Hier bietet sich
eine mögliche Lösung in der Aufteilung in Submodule.

Während der Entwicklung erwiesen sich die umfangreiche Dokumentation und die
bestehende Infrastruktur von 3D Slicer als große Vorteile. Die Implementierung einfacher
Algorithmen gestaltete sich dadurch effizient, und erste funktionale Prototypen
konnten zügig erstellt werden. Auch zeigte sich, dass eine Veröffentlichung über
den \textit{Extension Manager} nicht zwingend erforderlich ist – für bestimmte Nutzergruppen
kann das Modul problemlos lokal eingebunden werden.

Dennoch gab es Herausforderungen. Nicht alle Prinzipien einer idealen
Softwarearchitektur konnten konsequent umgesetzt werden. Diese Entscheidungen wurden
jedoch bewusst getroffen, um die Entwicklung pragmatisch und
anwendungsorientiert zu gestalten. Auch eine Vorverarbeitung der Bilder konnte nicht
implementiert werden. Zwar wurde diese Funktionalität vorgesehen und muss nur
noch befüllt werden, aber ein konkrete Ausführung ist zu dieser Zeit nicht mögliche.

Auch die Anforderungen aus Kapitel \ref{sec:anforderungsanalyse} sollen hier noch
einmal von kritischer Seite betrachtet werden. Neben der Frage, ob alle Anforderungen
erfüllt wurden, stellt sich hier auch die Frage, ob die Anforderungen auch passend
gewählt wurden. Wie die ersten Anwendertests zeigten, liefert der Tooth Analyser
einen guten Mehrwert im Forschungsaltag, sodass keine der Anforderungen als
misslungen bezeichnet werden kann. Auch die Ergenisdarsstellung in der Slicer-Szene
erlang in der ersten Testreihe ein gutes Feedback.

Zusammenfassend lässt sich sagen, dass die in dieser Arbeit entwickelte Lösung
die Forschungsfrage in wesentlichen Punkten beantworten konnte: Die Segmentierung
wurde effizient in 3D Slicer integriert, die Benutzerfreundlichkeit erheblich
verbessert und eine Erweiterbarkeit der Software gewährleistet. Darüber hinaus konnte
ein Modell ausgearbeitet werden, nach dem diverse Mikro-CT-Aufnahmen effizient
bearbeitet werden können. Allerdings zeigt ein Blick auf die in Kapitel \ref{sec:relevanz_der_arbeit}
diskutierte Relevanz der Arbeit, dass der Tooth Analyser noch nicht den
endgültigen Reifegrad erreicht hat. Um das System weiter zu optimieren, sind
zusätzliche Benutzertests und potenzielle Fehlerbehebungen erforderlich. Erst mit
diesen weiteren Schritten kann das Ziel einer langfristig etablierten und klinisch
einsetzbaren Lösung vollständig erreicht werden.
% ---------------------------------------------------------------------------------------