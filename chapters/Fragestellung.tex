\chapter{Forschungsfrage}
\label{chap:fragestellung} Dieses Kapitel beleuchtet die zentrale Frage, welche
mit Hilfe der Ergebnisse dieser Arbeit beantwortet werden soll. Dabei kann die Fragestellung
in erster Linie als Ergebnis der theoretischen Grundlagen Kapitel \ref{chap:theoretische_grundlagen}
interpretiert werden.

Die vorliegende Arbeit befasst sich mit der Erstellung einer Erweiterung für die
Plattform 3D Slicer, die ein spezifisches Segmentierungsverfahren auf Basis der
anatomischen Segmentierung integriert. Die Grundlagen für diese Methode bildet das
Kapitel \ref{sec:verwwandte_arbeit}. Die anatomische Segmentierung nutzt
bestehende Segmentierungsmethoden und wurde speziell für die Anforderungen der
Zahnklinik in München entwickelt. Sie ist prototypisch implementiert und zeigt in
ihrer Funktionalität vielversprechende Ergebnisse. Allerdings hat das aktuelle
Verfahren eine wesentliche Einschränkung: Es muss über das Terminal ausgeführt werden.
Dies erschwert die Anwendung in der klinischen Praxis und reduziert die
Benutzerfreundlichkeit erheblich. Als interaktive Lösung bietet sich die Software
3D Slicer an, da sie bereits in der Zahnklinik eingesetzt wird und über eine
flexible Infrastruktur für die Integration von Erweiterungen verfügt. Diese Eigenschaften
machen Slicer zu einer idealen Plattform, um die anatomische Segmentierung der
Zahnkronen in einer benutzerfreundlichen Form bereitzustellen.

Die zentrale Frage, die so aus den Grundlagen abgeleitet werden kann, teilt sich
in drei einzelen Fragen auf, die hier zu sehen sind.

\begin{description}
	\item[1. Fragestellung] Kann das Verfahren der anatomischen Segmentierung als Extension
		in Slicer implementiert werden?

	\item[2. Fragestellung] Ist es möglich, die Integration so zu gestalten, dass der
		zugrunde liegende Algorithmus problemlos austauschbar ist oder zusätzliche Algorithmen
		eingebaut werden können?

	\item[3. Fragestellung] Welche Erfahrungen lassen sich bei der Entwicklung einer
		Slicer-Extension sammeln, und welche Herausforderungen treten dabei auf?
\end{description}

Im folgenden Kapitel wird dargestellt, wie diese Fragestellung methodisch bearbeitet
wurde. Es wird gezeigt, wie das Problem strukturiert in Teilaufgaben zerlegt wurde
und welche Schritte zur Lösung unternommen wurden.
% ---------------------------------------------------------------------------------------