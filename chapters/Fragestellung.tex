\chapter{Fragestellung}
\label{chap:fragestellung} Dieses Kapitel beleuchten die zentrale Frage, welche
mit Hilfe der Ergebnisse dieser Arbeit beantwortet werden sollen. Dabei kann die
Fragestellung in erster Linie als Ergebnis der theoretischen Grundlagen Kapitel \ref{chap:theoretische_grundlagen}
interpretiert werden.

Die vorliegende Arbeit befasst sich mit der Erstellung einer Slicer-Extension,
die ein spezifisches Segmentierungsverfahren auf Basis der Methode nach Hoffmann
integriert. Die Grundlagen für diese Methode bilden die theoretischen Grundlagen.

Die Methode nach Hoffmann nutzt bestehende Segmentierungsmethoden und wurde speziell
für die Anforderungen der Zahnklinik in München entwickelt. Sie ist prototypisch
implementiert und zeigt in ihrer Funktionalität vielversprechende Ergebnisse. Allerdings
hat das aktuelle Verfahren eine wesentliche Einschränkung: Es muss über das
Terminal ausgeführt werden. Dies erschwert die Anwendung in der klinischen Praxis
und reduziert die Benutzerfreundlichkeit erheblich.

Als interaktive Lösung bietet sich die Software 3D Slicer an, da sie bereits in
der Zahnklinik eingesetzt wird und über eine flexible Infrastruktur für die
Integration von Erweiterungen verfügt. Diese Eigenschaften machen Slicer zu einer
idealen Plattform, um die Methode nach Hoffmann in einer benutzerfreundlichen
Form bereitzustellen.

Die zentrale Frage, die so aus den Grundlagen abgeleitet werden kann, ist:

\begin{enumerate}
	\item Kann das Verfahren nach Hoffmann als Extension in Slicer implementiert werden?

	\item Ist es möglich, die Integration so zu gestalten, dass der zugrunde liegende
		Algorithmus problemlos austauschbar ist oder zusätzliche Algorithmen eingebaut
		werden können?

	\item Welche Erfahrungen lassen sich bei der Entwicklung einer Slicer-Extension
		sammeln, und welche Herausforderungen treten dabei auf?
\end{enumerate}

Im folgenden Kapitel wird dargestellt, wie diese Fragestellung methodisch
bearbeitet wurde. Es wird gezeigt, wie das Problem strukturiert in Teilaufgaben
zerlegt wurde und welche Schritte zur Lösung unternommen wurden.
% ---------------------------------------------------------------------------------------