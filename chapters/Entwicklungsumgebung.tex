\chapter{Entwicklungsumgebung}
\label{chap:entwicklungsumgebung} Da bereits ein Framework feststeht, mit dem
gearbeitet werden soll, ist keine weitere Forschung nötig, um die richtige
Programmiersprache auszuwählen. Jedoch gibt es eine kleine Auswahl zu treffen.
3D Slicer unterscheidet zwischen zwei Arten von Modulen, die \ac{CLI}-Module, welche
in der Sprache C++ geschrieben werden und die \textit{Scripted Moduls}, die eine
Python Implementierung verlangen. Da die anatomische Segmentierung ohnehin in
einem IPython-Notebook bereitliegt, fiel die Wahl hier auf die Scripted Moduls.
So kann auch die breite Palette der Python-Pakete genutzte werden. Für eine detaillierte
Beschreibung des Frameworks selber sei an dieser Stelle auf das Kapitel \ref{sec:bildbearbeitung}f
verwiesen, indem das Framework und alle zugehörigen Eigenheiten noch genauer beschrieben
wurden. Um den Entwicklungsprozess etwas zu vereinfachen, wurde während der Entwicklung
auf ein Modul von Slicer zurückgegriffen, das speziell für Entwickler entworfen
wurde. Die Abbildung \ref{fig:entwicklungsumgebung} verdeutlicht dieses Tool.

\begin{figure}[h]
	\centering
	\includegraphics[width=0.6\textwidth]{img/Entwicklungsumgebung.png}
	\caption{Umgebung während der Entwicklung mit 3D Slicer und PyCharm}
	\label{fig:entwicklungsumgebung}
\end{figure}

Mit dem Debugging Tool lässt sich eine gewohnte Umgebung reproduzieren, in der der
Quellcode Schritt für Schritt analysiert werden kann \citep[vgl.][]{slicerdebuggingtools}.
Speziell bei der Fehlersuche ist dieses Tool eine sehr gute Unterstützung. Die
Abbildung beschreibt weiter, das als Umgebung für das Erstellen des Programmcodes
die Software Pycharm verwendet wird. Pycharm ist eine Lösung der Firma Jetbrains,
für das Erstellen von Python-Quellcode. Dieses Tool bietet eine breite Palette
an Funktionalitäten, die das Erstellen von Software vereinfachen und als \textit{State
of the Art} bezeichnet werden kann \citep[vgl.][]{jetbrains2024}.

Für die Interaktion mit der Slicer Kernanwendung stellt der Python-Interpreter das
Paket \texttt{slicer} zur Verfügung. Hierdurch lassen sich diverse Mechanismen steuern.
Wichtig ist hierbei, dass das Paket \texttt{slicer} nicht auf \ac{PyPi} zu finden
ist. Es ist nur in der Python umgebung vorhanden, die mit der Slicer Installation
einherget. Neben dem Paket \texttt{slicer} sind noch viele weitere Pakete verfügbar,
die sich in der medizinischen Bildbearbeitung durchgesetzt haben. Für eine komplette
Auflistung aller Python-Pakete sei an dieser Stelle auf die Dokumentation von Slicer
verwiesen \citep[vgl.][]{slicer2024}.

Damit während der Entwicklung auch Teilbereiche der Software bereits getestet
werden können, werden Testdaten benötigt. Bei diesen Testdaten handelt es sich
um originale Mikro-\ac{CT}-Aufnahmen als \ac{ISO}. Diese wurden auf einem Server
an der \ac{LMU} in München bereitgestellt. Der Zugiff auf den Server erfolgt über
den \textit{x2goclient}. Heruntergeladen wurden die Daten über eine \ac{SSH}
Verbindung. heruntergeladen werden. Bei den Mikro-CT-Bildern handelt es sich
ausschließlich um Aufnahmen der Zahnkrone. Mit dem Zugriff auf den Server an der
\ac{LMU} konnten auch diverse Pythonumgebungen zum Verarbeiten von Daten genutzt
werden. Dies war in erster Linie für ein Nachvollziehen und Verstehen der anatomischen
Segmentierung hilfreich.

Neben der eigentlichen Umgebung von Slicer und den Entwicklerwerkzeugen steht zur
Entwicklung auch ein Python Paket zur Verfügung, das von Herrn Prof. Rösch speziell
für die Klinik an der \ac{LMU} erstellt wurde. Dieses Tool beinhaltet diverse Funktionalität
für das Verarbeiten von medizinischen Bilddaten. Speziell für die Mikro-\ac{CT}-Aufnahmen
der Klinik.

Nachdem die Entwicklungsumgebung definiert und eingerichtet wurde, kann nun auf
die konkrete Methodik eingegangen werden, die zur Umsetzung dieser Schnittstelle
verwendet wird. Dabei werden sowohl die Anforderungen an die Erweiterung
beschrieben, also auch die konzeptionellen Überlegungen detailliert beschrieben.

% ---------------------------------------------------------------------------------------