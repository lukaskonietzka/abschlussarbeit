\chapter{Schlussfolgerung und Ausblick}
\label{chap:schlussfolgerung} Der Tooth Analyser bietet mit dem aktuellen Stand bereits
einen großen Mehrwert für die Ärzte in der Klinik für Zahnerhaltung. So lässt
sich sagen, dass diese vorliegende Arbeit durchaus von Erfolg gekrönt ist. Durch
das Kapitel \ref{sec:limitierungen} wird aber auch klar, dass noch viel Potenzial
im Tooth Analyser steckt. Das meiste Potenzial ist hierbei in den Limitierungen
zu finden. Eine hervorragende Ergänzung dieser Arbeit würde eine Vorverarbeitung
der Bilder bieten. Es soll also in Zukunft möglich sein, die CT-Aufnahmen in einer
Vorverarbeitung zu komprimieren und sie dann mittels eines Verfahren zu bearbeiten.
Dies schließt auch ein verarbeitung aller Formate ein. Ein schöner Nebeneffekt dieser
Ergänzung ist, das durch ein Komprimierung der Bilder die Laufzeit sinkt. Der verlorene
Detailgrad dieser Komrimierung ist dabei nicht störend. Somit lässt sich sagen, das
durch das Akzeptieren unterschiedlicher Formate der größte Merwehrt für den Tooth
Analyser gewonnen werden kann. Des weiteren lässt der Bereich Analysen im Tooth
Analyser noch viel Spielraum. Hier gibt es diverse Möglichkeiten, das Modul noch
mit weiteren Funktionen zu bestücken. Eine gute Idee ist hier eine Integration des
Python Pakets \textit{radiomics}. Damit lassen sich Radiomics- Merkmale aus
bilder Extrahierne und isoliert analysieren. Den letzten Ausblick der hier gegeben
werden soll bezieht sich wieder auf das Verfahren der anatomischen Segmentiernug.
In der aktuellen Version nimmt das Verfahren nur eine Aufteilung in Schmelz und
Dentin vor. Der Teil der Pulpa wird dem Hintergrund zugeordnet und nicht beachtet.
Die zusätzliche Segmentierung der Pulpa würde die Arbeit ebenfalls hervoragend
ergänzen. Jedoch sei auch gesagt, das dies die wohl herrausfordernste Aufgabe
dier hier erwähnten Ausbilcke ist. Die Pulpa hebt sich nur sehr leicht aus dem Hintergrund
hervor und ist demnach schwer zu segmentieren.

Ordnet man die hier genannten Punkte nach Ihrere Wichtigkeit ein, so lässt sich
sagen, das durch das Erweitern des Moduls auf mehrere Bildformate der größte Mehrwert
für den Tooth Analyser gewonnen werden kann. Jedoch sei auch gesagt, dass die die
übrigen Punkte ebenfalls einen großen Mehrwert für den Tooth Analyser und damit für
die praktizierenden Ärzte am Klinikum für Zahnerhaltung in München bietet.
% ---------------------------------------------------------------------------------------