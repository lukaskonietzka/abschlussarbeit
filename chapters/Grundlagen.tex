\chapter{Theoretische Grundlagen}
\label{chap:theoretische_grundlagen} Dieses Kapitel führt in die theoretischen Grundlagen
ein, die in dieser Arbeit benötigt werden. Den ersten Teil bilden die domänenspezifischen
Grundlagen \ref{sec:domänenspezifisch}, welche genauer darauf eingehen, welchen Inhalt
die zu bearbeitenden Bilder bieten und wie dieser zu verstehen ist. Abschnitt
\ref{sec:technologisch} geht hierbei auf verschiedenen Technologien ein, die eine
wichtige Rolle spielen. Der Abschnitt \ref{sec:verwwandte_arbeit} geht auf die
Arbeit von \citet{hoffmann2020} ein und legte damit den Grundstein dieser Arbeit.
Die Abschnitte \ref{sec:3d_slicer} und \ref{sec:architektonisch} führen in
Softwareentwicklungsthemen ein, die zum Erstellen einer \textit{3D Slicer
Extention} wichtig sind.

\section{Domänenspezifisch}
\label{sec:domänenspezifisch} Wie bereits aus dem Kapitel \ref{chap:einleitung}
Einleitung klar wurde, handelt es sich bei den Micro-CT Bilder um Zahnbilder,
die aufgrund von Zahnkaries entfernt wurden. Um zu verstehen, wie eine CT-Aufnahme
eines Zahns aufgeteilt werden soll, ist es hilfreich zu verstehen, wie ein Zahn
aufgebaut ist.

\begin{minipage}{0.40\textwidth}
	Die Abbildung \ref{fig:aufbau_eines_zahnes} zeigt den groben Aufbau eines Zahnes
	nach \citet[Seite 17]{lehmann2012Zahnheilkunde}. Zu sehen ist, dass das Denit oder
	auch Zahnbein genannt, den Großteil eines Zahnes einnimmt. Im Bereich der Zahnkrone
	wird das Dentin von Zahnschmelz überzogen. Der Zahnschmelz ragt in die
	Mundhöhle und ist nach \cite[Seite 41]{lehmann2012Zahnheilkunde} das härteste
	Material im menschlichen Körper. In der Mitte des Zahnes befindet sich Weichgewebe,
	welches als Pulpa bezeichnet wird vgl. \citep[Seite ]{lehmann2012Zahnheilkunde}.
\end{minipage}
\hfill
\begin{minipage}{0.50\textwidth}
	\centering
	\includegraphics[scale=0.50]{img/aufbau_eines_zahns.jpg}
	\captionof{figure}{Aufbau eines Zahnes nach \citet{lehmann2012Zahnheilkunde}} \label{fig:aufbau_eines_zahnes}
\end{minipage}

Für die Bearbeitung von Micro-CT Aufnahmen sind die Bereich Schmelz Dentin und Pulpa
von besonderer Bedeutung. Betrachtet man eine CT wie es zu Beginn in der
Abbildung \ref{fig:ct_aufnahme_eines_zahns} gezeigt wurde, so bilden diese 3 Gewebearten
die unterschiedlichen Grauwerte in einem CT-Bild. \\ \textbf{Pulpa:} Die Pulpa unterscheidet
sich hierbei nur wenig vom Hintergrund, da sie als einzige der drei Hauptteile
eines Zahnes ein Weichgewebe ist und bei einer Röntgenaufnahme nicht absorbiert.
Geht man weiter von innen nach außen, so ist der nächste Zahnteil auf einem CT
das Zahnbein. \\ \textbf{Dentin:} Das Dentin ist laut \citet[Seite 41]{lehmann2012Zahnheilkunde}
eine Hartsubstanz, die dem Kieferknochen sehr nah steht. So kommt es, dass dieser
Teil schon deutlich besser auf einem CT zu erkennen ist. Den äußersten Teil in
der Mundhöhle bildet das Zahnschmelz. \\ \textbf{Schmelz:} Der Schmelz ist wie
bereits erwähnt, der härteste Teil im menschlichen Körper und aus diesem Grund auch
am hellsten auf dem CT zu erkennen. Die folgende Abbildung
\ref{fig:pulpa_dentin_schmelz} sollen durch Gegenüberstellung den Zusammenhang
zwischen einem CT-Bild und einer Zahnzeichnung verdeutlichen.

\begin{figure}[h]
	\centering
	\includegraphics[width=0.9\textwidth]{
		img/Bildschirmfoto 2024-11-22 um 15.13.24.jpg
	}
	\caption{Darstellung von Pulpa, Dentin und Schmelz auf einer CT-Aufnahme (link)
	und einer Zeichnung (rechts) nach \citet[Seite 29]{lehmann2012Zahnheilkunde}. }
	\label{fig:pulpa_dentin_schmelz}
\end{figure}

Mit diesem Domänenwissen kann ein Schritt weiter gegangen werden, sodass der
Fokus nun auf die CT-Bilder gesetzt wird. Das Kapitel \ref{sec:technologisch} Technologien
führt die Technologie der Computertomografie tiefer ein. Mit diesem
Domänenwissen kann ein Schritt weiter gegangen werden, sodass der Fokus nun auf die
CT-Bilder gesetzt wird. Das Kapitel \ref{sec:technologisch} Technologien führt
die Technologie der Computertomografie tiefer ein.£

\section{Technologisch}
\label{sec:technologisch} Dieser Abschnitt erläutert federführend die Technologie
der Micro-CT Bilder und deren weitere Bearbeitung. Hierbei soll mit der
Computertomografie selbst begonnen werden. Für die Einführung in die Bearbeitung
der Micro-CT Bilder bietet das Pipeline-Modell von \citet[Seite 50]{handels2000}
eine gute Richtlinie. Er beschreibt mit dieser Visualisierungs-Pipeline Schritte,
die bei der Bearbeitung und dreidimensionalen Darstellung von CT-Aufnahmen notwendig
sind (vgl. \citep[Seite 50]{handels2000}).

\subsection{Computertomografie}
\label{subsec:computertomografie} Die Erfindung der Computertomografie (CT) war
ein Quantensprung in der Geschichte der Medizin. Sie ist aus heutigen Diagnosen nicht
mehr wegzudenken. Ein Micro-CT Bild ist laut \citet[Abstract]{baird2017} ein
Menge hochauflösender Bilder, die wie ein Stapel zusammengelegt werden. Der
Aspekt Micro deutet dabei darauf hin, dass es eine miniaturisierte Ausführung eines
üblichen Kegelstrahl-CTs ist so \citet[Seite 340]{buzug2011}. Eine andere
Definition erläutert \citet{lehmann2013bildverarbeitung}. Er beschreibt die
Computertomografie als Projektionen einzelner Ebenen im Untersuchungsobjekt. Die
Technologie, mit der diese Bilderstapel aufgenommen werden, ist unter der Röntgentechnik
oder auch X-Ray bekannt. Die Röntgenstrahlung ist eine Form der
elektromagnetischen Strahlung, ähnlich wie da sichtbare Licht so das \citet{nib2024}.
Anders als das Licht haben die Röntgenstrahlen eine viel höhere Energie. Das
führt dazu, dass man mit dieser elektromagnetischen Strahlung viele Objekte durchdringen
kann. So auch Gewebeteile eines Zahnes vgl. \citep{nib2024}. Die Abbildung \ref{fig:spectrum}
zeigt diese elektromagnetische Spektrum.

\begin{figure}[h]
	\centering
	\includegraphics[width=0.9\textwidth]{img/spectrum.jpg}
	\caption{Einordnung der Röntgenstrahlung nach dem \citet{nib2024}}
	\label{fig:spectrum}
\end{figure}

Durchdringt ein solcher Röntgenstrahl ein Untersuchungsobjekt, werden die Details
aufgrund der Wechselwirkung mit Materie auf einer CT-Probe sichtbar. Die
bekannteste Wechselwirkung ist die Absorption. Mit der Steigerung der Atomzahl in
einem Material nimmt auch die Absorption eines Materials zu, sodass es leicht
ist verschiedenen Materialien in einer CT-Aufnahme zu unterscheiden \citep{nib2024}.

Für eine Micro-CT Aufnahme bedarf es spezieller Technik. Es gibt
unterschiedliche Firmen, welche die unterschiedlichsten Modelle anbieten. Im
Falle der Zahnklinik an der LMU in München handelt es sich um ein Micro-CT 40
der Firma \citet{scanco2024}. Dieses Gerät erstellt Aufnahmen mittels
Röntgenstrahlung und generiert mithilfe der Computertomografie ein
dreidimensionales Bild, welches im Format \texttt{.ISQ} abgelegt wird. Wie das Nächste
Kapitel beschreiben wird ist der Speicherumfang den solch ein Bild benötigt,
deutlich zu groß. Es bedarf einer Technik, mit der die Aufnahmen auf eine
handhabbare Größe schrumpfen.

\subsection{Datensätze}
\label{subsec:datensätze} Die rohen Datensätze, welche direkt aus dem Micr-CT Gerät
kommen, haben nach \citet{scanco2024} das Format \texttt{.ISQ}. Dieses Format fällt
speziell auf die Geräte der Firma SCANCO zurück. Wie das vorherige Kapitel
\ref{subsec:computertomografie} bereits eingeführt hat, ist dieser Dateityp für eine
weitere Bearbeitung nicht geeignet. Unter Anderm wegen ihrer Größe. \citet{RoeschKunzelmann2018}
haben hierfür eine Paket entwickelt. Diese convertiert ein \texttt{.ISQ} Format
in ein \texttt{.mhd} Format. Bei einer \texttt{.mhd} Datei handelt es sich im ein
Metafile, dass auf die eigentliche Dateiverweist. Folgender Ausschnitt zeigt die
verwendung das Pakets.

\texttt{python3 isq\_to\_mhd.py <quelle> <ziel>}

Diese Meta-Datei kann genutzt werden um interessante Infos über das Bild zu erlangen.
Wird dieser Kommand ausgeführt, so erstellt das Skript \texttt{isq\_to\_mhd} ein
Metafile, das detalierte Daten über die Datei enthält. Ein Ausschnitt dieses
Metafiles liefer das Listing \ref{lst:inhalt_mhd_datei}

\begin{lstlisting}[
	caption={Ausschnitt des Inhaltes einer MHD-Datei},
	label={lst:inhalt_mhd_datei}]
ObjectType = Image
NDims = 3
CenterOfRotation = 0 0 0
ElementSpacing = 0.02 0.02 0.02
DimSize = 1024 1024 517
ElementType = MET_SHORT
ElementDataFile = P01A-C0005278.ISQ
\end{lstlisting}

In der Datei sind Infos über die Ausprägung, Art und Größe der Datei zu finden.
Besonders interessant sind die Punkte \texttt{DimSize und ElementType}. Über
diese Parameter lässt sich die Größe eines Bildes berechenen. \citet[Seite 10-11]{burger2009}
erklärt, das ein Bild in Zellen aufgeteilt ist, welche Informationen enthalten.
Diese Zellen sind im zweidimensionalen Raum als Pixel bekannt. Betrachtet man jedoch
ein, wie im Falle der Zahnklinik an der LMU ein dreidimensonales Bild, so spricht
man nicht mehr von Pixeln sonder von Voxeln. Ein Voxel ist demnach das
dreidimensonale Äquivalent zu einem Pixel. \citet[Seite 10-11]{burger2009} beschreibt
weiter das jeder diese Zellen ein binäres Wort der Länge $2^{k}$ ist. Die Basis
2 ergibt sich durch das binäre Wort, wo hingengen für k gilt: $k \in \mathbb{N}$.
Um für den konkreten Fall aus Listing \ref{lst:inhalt_mhd_datei} das entsprechenden
$k$ zu ermittlen muss der \texttt{ElemtType} näher betrachtet werden. \texttt{MET\_SHORT}
steht hierbei für Signed short, was eine größe von 16 bit entspricht. Damit
ergibt sich für die länge $k$ ein Wert von 4. So gilt nach \citet[Seite 10-11]{burger2009}

\begin{align}
	\label{equ:größe_bestimmen}1024 \cdot 1024 \cdot 517    & = 542,113,792 \, \text{Voxel}\notag  \\
	542,113,792 \, \text{Voxel}\cdot 2 \, \text{Byte/Voxel} & = 1,084,227,584 \, \text{Byte}\notag \\
	1,084,227,584 / 1,000,000,000                           & = 1.0842 \, \text{GB}
\end{align}

Die erste Gleichung bestimmt die Gesamtzahl aller Voxel. Gleichung 2 Bestimmt die
Größe des Bildes in der einheit Byte. Die Letzte Zeile nimmt eine Umrechnung von
Byte nach Gigabyte (GB) vor.

Durch die Gleichungen in \ref{equ:größe_bestimmen} wird klar, das eine CT-Aufnahem
des Typs \texttt{ISQ} direkt nach seiner Aufnahme über einen GB groß ist. Laut \citet{poliklinikLMU}
ist dies ein zu Großes Format. Es stellt sich also die spannennde Frage, wie
solch eine Datei komprimiert werden kann, ohne das es Verluste in der Qualität gibt.

\subsection{Filter}

\subsection{Segmentierung}

\section{Verwandte Arbeit}
\label{sec:verwwandte_arbeit}

\section{3D Slicer}
\label{sec:3d_slicer}

\subsection{Extention Manager}

\subsection{Python Umgebung}

\subsection{MRML Datenstruktur}

\subsection{Qt-Designer}

\section{Architektonisch}
\label{sec:architektonisch}