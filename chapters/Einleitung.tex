\chapter{Einleitung}
\label{chap:einleitung} Die \ac{CT} hat die Medizintechnik revolutioniert und
ist bis heute eines der wichtigsten Methoden für die Bildanalyse. Sie ist eine der
führenden Erweiterungen der klassischen Röntgentechnik. Für die Entwicklung
dieser Technologie wurden Godfrey Newbold Hounsfield und Allan McLeod Cormack im
Jahre 1979 mit dem Nobelpreis für Medizin ausgezeichnet \citep[.vgl][S.~12]{handels2000}.

\begin{minipage}{0.45\textwidth}
	Die Computertomografie wird in den verschiedensten Bereichen und im wahrsten Sinne
	des Wortes von Kopf bis Fuß eingesetzt. So kommt es, dass auch im Dentalbereich
	\ac{CT} Aufnahmen von größter Wichtigkeit sind. Abbildung \ref{fig:ct_aufnahme_eines_zahns}
	zeigt eine solche \ac{CT}-Aufnahmen. Eine konkrete Anwendung in diesem Kontext
	ist die Zahnkaries Forschung der Poliklinik für Zahnerhaltung und Parodontologie
	der \ac{LMU}.
\end{minipage}
\hfill
\begin{minipage}{0.45\textwidth}
	\centering
	\includegraphics[scale=0.2, width=\textwidth]{img/micro_ct_orginal.jpg}
	\captionof{figure}{\ac{CT}-Aufnahme eines Zahns nach \citet{heck2024}} \label{fig:ct_aufnahme_eines_zahns}
\end{minipage}

Die vorliegende Arbeit soll genau diese Forschung unterstützen. In welchem Umfang
und zu welchem Grund ist in den folgenden Abschnitten beschrieben.
% ---------------------------------------------------------------------------------------

\section{Ziel der Arbeit}
\label{sec:ziel_der_arbeit} Diese Arbeit beschreibt eine Technik, mit der \ac{3D}
Mikro-\ac{CT}-Bilder zur Untersuchung zahnmedizinischen Strukturen automatisch
mittels der Software \textit{3D Slicer} segmentiert und analysiert werden können.
Was genau unter eine Segmentierung verstanden wird, darüber informiert das Kapitel
\ref{subsec:segmentierung} Segmentierung. Die algorithmische Formulierung einer
konkreten Segmentierung ist bereits vorhanden und prototypisch implementiert. Dieser
Algorithmus hat jedoch Schwachstellen. So muss beispielsweise das Verfahren
umständlich über ein ipython Notebook im Terminal ausgeführt werden, was die Benutzerfreundlichkeit
deutlich beeinträchtigt. Ziel dieser Arbeit ist es in erster Linie das bereits
existierende Verfahren in der Klinik für Zahnerhaltung zu analysieren und für
die Mitarbeiter der Klinik benutzbar zu machen. Dabei soll auf etablierte und vertraute
Lösungen zurückgegriffen werden.

Es stellt sich nun die Frage, zu welchem Zweck eine automatische und interaktive
Segmentierung überhaupt notwendig ist. Für die Zahnklinik an der LMU in München gibt
es hierfür viele Gründe. Über den wichtigsten gibt das nächste Kapitel
Aufschluss.
% ---------------------------------------------------------------------------------------

\section{Relevanz der Arbeit}
\label{sec:relevanz_der_arbeit} Der wohl relevanteste Punkt wurde bereits im
vorherigen Kapitel \ref{sec:ziel_der_arbeit} diskutiert, Zahnärzte sind keine Softwareentwickler
sondern reine Anwender von Software. Darüber hinaus verfolgt die Klinik für
Zahnerhaltung und Parodontologie der \ac{LMU} einen sehr interessanten Forschungsansatz,
welche eine Segmentbetrachtung der \ac{CT}s rechtfertigt.

Über viele Jahre hinweg wurden in der Zahnklinik sehr viel Bilddaten von Zähnen
gesammelt. Hierbei wurden Aufnahmen der unterschiedlichsten Arten gemacht.
Darunter fallen zum Beispiel einfache Bilddateien, Infrarotbilder und die für diese
Arbeit so relevanten dreidimensionalen Mikro-CT-Aufnahmen. Dieser große Schatz
an Bildmaterial soll verwendet werden, um in ferner Zukunft ein neuronales Netzwerk
zu trainieren, welches statistische Aussagen über das Verhalten von Karies
treffen kann. Jedoch gibt es hier ein Problem bei dem das Ergebnis dieser Arbeit
unterstützen kann. Karies auf \ac{CT}-Bildern zu lokalisieren ist nicht trivial.
Er ist ohne weitere Bearbeitung des Bildes nur sehr schwer auf eine Stelle
einzugrenzen. So kommt es vor, dass drei verschiedene Ärzte auf dem selben Mikro-\ac{CT}-Bild
drei unterschiedliche Stellen mit Karies identifizieren. Eine Segmentierung des
dreidimensionalen \ac{CT}s kann hier Wunder wirken. Durch die Aufteilung des Mikro-\ac{CT}s
in seine zwei Zahnhauptsubstanzen, kann eine sehr gute visuelle Darstellung des
Zahnes gewährleistet werden. Für Ärztes bietet diese Darstellung einen sehr großen
Mehrwert \citep[vgl.][S.~1]{walter2025projekt}.

Mit dieser klaren und eindeutigen Identifizierung von Karies, sind die Ergebnisse,
die ein neuronales Netzer generieren würde viel genauer und brauchbarer. Konkret
wird mit einer automatischen Segmentierung ein \textit{Ground Trueth} gewonnen, der
eine eindeutige Basiswahrheit liefert. Hierbei sei gesagt das diese Anwendung
nur eine von vielen Möglichkeiten ist. Konkrete Daten über die Ausbreitung einer
Krankheit im Menschlichen Körper zu besitzen kann in den verschiedensten Fällen und
Institutionen von größtem Nutzen sein. So zeigen es auch \citet{de20083d} in ihrem
Paper.

Anhand dieser Argumente wird deutlich, dass eine automatische Segmentierung
durchaus einen mehrwert für Ärzte bilden kann. Nicht zuletzt auch durch die enorme
Zeiteinsparung. Für eine automatische Segmentierung von Mikro-\ac{CT}-Bildern gibt
es einige Softwarelösungen am Markt, die alle eine gut Optionen sind. Aus diesem
Grund soll im folgenden Kapitel ein mögliches Framework diskutiert werden.
% ---------------------------------------------------------------------------------------

\section{Fokus der Arbeit}
\label{sec:fokus_der-arbeit} Dieser Arbeit setzt den Fokus auf die Open Sorce Plattform
3D Slicer, da diese ohnehin bereits eine breite Anwendung in der Zahnklinik in
München findet. Durch die Modul und Plug-In Infrastruktur dieser Plattform kann die
Software auch anderen Institutionen bereitgestellt werden. Hierzu muss diese
einfach als \textit{3D Slicer Extension} bereitgestellt werden. 3D Slicer bietet
einen \textit{Extension Manager}, der ähnlich wie ein App Store betrachtet
werden kann. So bleibt die vorerst konkret entwickelte Software nicht nur einer Einrichtung
vorbehalten. Eine tiefere Einführung in die Open Source Plattform bietet der
Abschnitt \ref{sec:3d_slicer}. Das weitere Optimieren des bereits bestehenden Verfahrens
wird in dieser Arbeit nicht weiter thematisiert. Es werden lediglich Anpassungen
vorgenommen, sodass eine Benutzerschnittstelle verwendet werden kann.

Mit diesem Umfang, der Motiavtion und dem gesetzten Fokus, ergibt sich für dies Arbeit
eine konkrete Struktur, die einen hohen Detailgrad aufweisen wird. Um einen Überblick
zu gewähren, sei diese Struktur hier kurz erläutert.
% ---------------------------------------------------------------------------------------

\section{Aufbau der Arbeit}
\label{sec:aufbau_der_arbeit} Die Arbeit ist in sieben Kapitel unterteilt. Nach der
Einführung in Kapitel \ref{chap:einleitung}, in der die Relevanz und der Fokus
beschrieben werden, werden in Kapitel \ref{chap:theoretische_grundlagen} die theoretischen
und technischen Grundlagen behandelt, welche zum Verstehen der Ergebnisse
essenziell sind. Als Ergebnis der theoretischen Grundlagen bildet das Kapitel \ref{chap:fragestellung}
eine konkrete Forschungsfrage. Während sich Kapitel \ref{chap:methodik} darum
kümmert mit welchen Methodiken und Lösungsansätzen an die Forschungsfrage
herrangegangen wird, erläutert das Kapitel \ref{chap:ergebnisse} welche die konkreten
Ergebnisse der Arbeit sind. In Kapitel \ref{chap:diskussion} erfolgt eine
kritische Diskussion der Resultate einschließich möglicher Limitationen. Das Abschließende
Kapitel \ref{chap:schlussfolgerung} fasst die wichtigsten Erkenntnisse zusammen und
gibt einen Ausblick auf zukünftige Forschungsfragen.

Die theoretischen Grundlagen die wie beschrieben direkt nach der Einleitung folgen,
sind zentral für das Verstehen der Fragestellung und der später folgenden
Ergebnisse der Arbeit.
% ---------------------------------------------------------------------------------------