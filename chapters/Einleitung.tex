\chapter{Einleitung}
\label{chap:einleitung} Die \ac{CT} hat die Medizintechnik revolutioniert und
zählt bis heute zu den wichtigsten Verfahren der Bildanalyse. Sie stellt eine entscheidende
Weiterentwicklung der klassischen Röntgentechnik dar. Für ihre bahnbrechende
Entwicklung wurden Godfrey Newbold Hounsfield und Allan McLeod Cormack im Jahr
1979 mit dem Nobelpreis für Medizin ausgezeichnet \citep[vgl.][S.~12]{handels2000}.
Dank dieser innovativen Technik lassen sich detaillierte Krankheitsanalysen
durchführen, die eine gezielte und individuell angepasste Behandlung ermöglichen.
Damit trägt die Computertomografie maßgeblich zur Optimierung der medizinischen
Versorgung bei und verbessert die Effektivität therapeutischer Maßnahmen \citep[vgl.][S.~207]{de20083d}.

Die Einsatzmöglichkeiten der Computertomografie sind vielfältig und werden im wahrsten
Sinne des Wortes von Kopf bis Fuß eingesetzt. So kommt es, dass sie auch in der Zahnmedizin
eine zentrale Rolle spielen. Die Abbildung \ref{fig:ct_aufnahme_eines_zahns}
zeigt eine solche Aufnahme, wie sie zu Forschungszwecken eingesetzt wird.

\begin{figure}[h]
	\centering
	\includegraphics[width=0.5\textwidth]{img/micro_ct_orginal.jpg}
	\caption{CT-Aufnahme einer Zahnkrone nach \citet{heck2024}}
	\label{fig:ct_aufnahme_eines_zahns}
\end{figure}

Mikro-\ac{CT}-Aufnahmen von Zähnen liefern hochauflösende Bilder der inneren Zahnstruktur
und sind damit eine essenzielle Grundlage für die zahnmedizinische Forschung.
Sie ermöglichen nicht nur eine präzisere Diagnostik, sondern auch weiterführende
wissenschaftliche Analysen, beispielsweise zur Untersuchung von Kariesverläufen
oder zur Entwicklung neuer Behandlungsmethoden. Eine konkrete Anwendung zeigt die
Forschung der Poliklinik für Zahnerhaltung und Parodontologie an der \ac{LMU}.
% ---------------------------------------------------------------------------------------

\section{Relevanz der Arbeit}
\label{sec:relevanz_der_arbeit} Da Mikro-\ac{CT} eines der relevantesten Forschungsgrundlagen
im Dentalbereich ist, sammelt die Klinik Bilddaten in großen Mengen. Dabei entstehen
Aufnahmen der unterschiedlichsten Art, darunter einfache Bilddateien, Infrarotbilder
und insbesondere die für diese Arbeit relevanten dreidimensionalen Mikro-\ac{CT}-Aufnahmen.
Dieser umfangreiche Datenschatz soll künftig genutzt werden, um ein neuronales
Netzwerk zu trainieren, das statistische Aussagen über das Verhalten von Karies treffen
kann \citep[vgl.][S.~1]{walter2025projekt}. Damit dieses Ziel erreicht werden kann,
müssen die Mikro-\ac{CT}-Bilder jedoch zunächst in eine geeignete Form gebracht werden.
Rohdaten allein sind für eine weiterführende Analyse nur bedingt nutzbar, da sie
komplexe Strukturen enthalten, die erst durch gezielte Verarbeitung sinnvoll
interpretiert werden können. Eine der zentralen Herausforderungen dabei ist die Segmentierung
– also die präzise Trennung der verschiedenen Gewebetypen wie Zahnschmelz und Dentin
\citep[vgl.][S.~359]{lehmann2013bildverarbeitung}. Erst durch diesen Schritt lassen
sich anatomisch aussagekräftige Modelle erstellen, die als Grundlage für
Diagnosen, Rekonstruktionen und weiterführende computergestützte Verfahren dienen.
Besonders für das Training neuronaler Netzwerke ist eine saubere Segmentierung
essenziell, da sie die Qualität und Aussagekraft der daraus gewonnenen Erkenntnisse
maßgeblich beeinflusst.
% ---------------------------------------------------------------------------------------

\section{Ziel der Arbeit}
\label{sec:ziel_der_arbeit} Das Erzeugen von segmentierter Daten aus einem Mikro-\ac{CT}
ist jedoch nicht trivial und verlangt komplexe Techniken der 3D-Bildbearbeitung.
Der aktuelle Stand der Kunst zeigt, dass es bereits ein Verfahren gibt, mit dem solche
Daten erzeugt werden können. Hierbei unterteilt das Verfahren ein gegebenes
Mikro-\ac{CT}-Bild in die zwei Gewebesubstanzen Dentin und Schmelz. Dieser
Vorgang kann als anatomische Segmentierung des Zahns bezeichnet werden. Die Ergebnisse
des Verfahrens zeigen vielversprechende Ergebnisse. Jedoch bringt es auch einige
Limitierungen mit, was die Benutzung deutlich einschränkt. Eine dieser
Limitierungen bezieht sich auf die Art der Ausführung. Zum aktuellen Stand muss das
Verfahren aufwendig über ein Terminal gestartet werden. Dies stellt für
praktizierende Ärzte, die letzten Endes die Anwendergruppe dieses Verfahrens sind,
eine besondere Herausforderung dar.

Die vorliegende Arbeit widmet sich genau dieser aktuellen Herausforderung. Ziel ist
es, eine Benutzerschnittstelle zu entwickeln, die den Anwender mit der anatomischen
Segmentierung verbindet und so die Verwendung des Verfahrens deutlich vereinfacht.
Da Mikro-\ac{CT}-Bilder weltweit eine zentrale Rolle in der zahnmedizinischen Forschung
spielt, bietet sich hierfür die Plattform 3D Slicer an. Diese etablierte Software
wird in zahlreichen Kliniken und Forschungseinrichtungen weltweit zur Verarbeitung
und Analyse medizinischer Bilddaten eingesetzt. Aufbauend auf dieser Grundlage
soll nun eine spezielle Benutzerschnittstelle innerhalb 3D Slicer entwickelt
werden, die auf der vorhandenen Modul- und Plug-in-Infrastruktur der Plattform
basiert. Diese Infrastruktur bietet die Möglichkeit, eigene Module mit
individuellen Algorithmen zu befüllen, eine Schnittstelle bereitzustellen und diese
nahtlos in das Kernsystem zu integrieren. 3D Slicer sieht hierfür die
Implementierung einer \ac{SEM} vor. Hierbei handelt es sich um einer Erweiterung
oder ein Plug-in für die 3D Slicer Kernanwendung. Durch die Bereitstellung von segmentierten
Daten in einer \ac{SEM} würde so nicht nur die der aktuellen Forschungsumgebung
einen Mehrwert erhalten, sondern die gesamte Zahnmedizin.

Bei der Umsetzung dieser Erweiterung stehen nicht nur die funktionalen Anforderungen
im Fokus, sondern auch softwaretechnische Qualitätskriterien. Eine möglichst generische
Architektur soll gewährleisten, dass das Modul flexibel bleibt und künftig um
weitere Funktionalitäten erweitert werden kann.

Mit diesen konkreten Zielsetzungen lässt sich nun eine Forschungsfrage ableiten,
die in dieser Arbeit im Fokus stehen soll.

\begin{center}
	\textit{Wie kann eine benutzerfreundliche Schnittstelle innerhalb 3D Slicer
	entwickelt werden, die das Verfahren der anatomischen Segmentierung effizient integriert,
	den Zugang für Anwender vereinfacht und zugleich eine flexible Erweiterbarkeit
	für zukünftige Funktionalitäten gewährleistet?}
\end{center}

Um das in dieser Arbeit zu entwickelnde Modul optimal in den bestehenden Prozess
der anatomischen Segmentierung zu integrieren, ist es essenziell, zunächst ein
grundlegendes Verständnis dieses Verfahrens zu erlangen. Daher wird im folgenden
Kapitel erläutert, wie die anatomische Segmentierung funktioniert, welche Techniken
und Algorithmen dabei zum Einsatz kommen und welche Herausforderungen sich daraus
ergeben. Dieses theoretische Fundament bildet so die Basis für die darauf
aufbauende Entwicklung einer 3D Slicer Erweiterung.
% ---------------------------------------------------------------------------------------