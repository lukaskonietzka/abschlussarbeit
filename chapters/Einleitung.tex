\chapter{Einleitung}
\label{chap:einleitung} Die Computertomografie (CT) hat die Medizintechnik revolutioniert
und ist bis heute eines der wichtigsten Methoden für die Bildanalyse. Sie ist
eine der führenden Erweiterungen der klassischen Röntgentechnik. Für die Entwicklung
dieser Technologie wurden Godfrey Newbold Hounsfield und Allan McLeod Cormack im
Jahre 1979 mit dem Nobelpreis für Medizin ausgezeichnet \citep[Seite12]{handels2000}.

Die Computertomografie wird in den verschiedensten Bereichen und im wahrsten Sinne
des Wortes von Kopf bis Fuß eingesetzt. So kommt es, dass auch im Dentalbereich CT
aufnahmen von größter Wichtigkeit sind. Eine konkrete Anwendung in diesem
Kontext ist die Zahnkaries Forschung der Poliklinik für Zahnerhaltung und
Parodontologie des LMU- Klinikums München. Die vorliegende Arbeit soll diese
Forschung unterstützen. In welchem Umfang dies geschehen soll, ist im folgenden
Kapitel thematisiert.

\section{Ziel der Arbeit}
\label{sec:ziel_der_arbeit} Diese Arbeit beschreibt eine Technik, mit der dreidimensionale
Micro-CT Bilder zur Untersuchung zahnmedizinischen Strukturen automatisch mittels
der Software \textit{3D Slicer} segmentiert und analysiert werden können. Die algorithmische
Formulierung einer konkreten Segmentierung ist bereits vorhanden und prototypisch
implementiert. Dieser Algorithmus muss jedoch umständlich über ein Kommando im
Terminal ausgeführt werden, was die Benutzerfreundlichkeit deutlich beeinträchtigt.
Ziel dieser Arbeit ist es nun das bereits implementiert Verfahren automatisiert
über ein interaktives User Interface (UI) zur Verfügung zu stellen. Dabei soll auf
etablierte und vertraute Lösungen zurückgegriffen werden.

Es stellt sich nun die Frage, zu welchem Zweck eine automatische und interaktive
Segmentierung überhaupt notwendig ist. Für die Zahnklinik an der LMU in München gibt
es hierfür viele Gründe. Über den wichtigsten gibt das nächste Kapitel
Aufschluss.

\section{Relevanz der Arbeit}
\label{sec:relevanz_der_arbeit} Der wohl relevanteste Punkt dieser Arbeit ist, dass
Ärzte reine Anwender und keine Entwickler von Software sind. Darüber hinaus
verfolgt die Parodontologie der LMU in München einen sehr interessanten
Forschungsansatz, die eine Segmentbetrachtung von Zahn-CTs unumgänglich macht.

Über viele Jahre hinweg wurden in der Zahnklinik sehr viel Bilddaten von Zähnen gesammelt,
die aufgrund von Zahnkaries entfernt wurden. Hierbei wurden Aufnahmen der unterschiedlichsten
Arten gemacht. Darunter fallen zum Beispiel einfache Bilddateien, Infrarotbilder
und die für diese Arbeit so relevanten dreidimensionalen Micro-CT Aufnahmen.
Dieser große Schatz an Bildmaterial soll verwendet werden, um in ferner Zukunft ein
neuronales Netzwerk zu trainieren, welches statistische Aussagen über das
Verhalten von Karies treffen kann. Jedoch gibt es hier ein Problem, bei dem das
Ergebnis dieser Arbeit Unterstützen kann. Karies auf CT-Bildern zu lokalisieren ist
nicht trivial. Er ist ohne weitere Bearbeitung des Bildes nur sehr schwer auf
eine Stelle einzugrenzen. So kommt es vor, dass auf einem CT drei
unterschiedlichen Ärzten drei unterschiedliche Stellen mit Karies identifizieren.
Eine Segmentierung des dreidimensionalen CTs kann hier Wunder wirken lassen.
Durch die Aufteilung des Micro-CTs in seine drei Zahnhartsubstanzen kann in das
innere der Zähne geblickt werden, was die lokalisierung kariöser Stellen deutlich
vereinfacht.

Mit dieser klaren und einduetigen identifizierung von Karies, sind die
Ergebnisse, die ein neuronales Netzer generieren würde viel genauer und brauchbarer.
Konkret wird mit einer automatischen Segmentierung ein \textit{Ground Trueth} gewonnen,
der eine eindeutige Basiswahrheit liefert.

Hierbe sei gesagt das diese Anwendung nur ein von vielen Möglichkeiten ist. Konkrete
Daten über die Ausbreitung einer Krankheit zu besitzten kann in den verschiedensten
Fällen und Institutionen von größtem Nutzen sein. Dieses Argument weist auf den
nächsten Teil dieser Arbeit hin.

\section{Fokus der Arbeit}
\label{sec:fokus_der-arbeit} Durch die integration der automatischen
Segmentierung in die Plattform \textit{3D Slicer}, steht dieser Algortihmus
durch das Modul-System auch anderen Institutionen zur Verfühgung. Hierzu setzt diese
Arbeit den Fokus klar auf die \textit{3D Slicer Extention}

\section{Aufbau der Arbeit}
\label{sec:aufbau_der_arbeit}