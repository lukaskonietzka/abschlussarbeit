%%% ---------------------------------------------------------------------------------
%%% Vorlage Abschlussarbeit (LaTeX)
%%%
%%% V1   03/2017, Stefan Etschberger (HSA)
%%% V1.1 04/2021, rnw-hack für biblatex-run
%%% V2   05/2021, Titelblatt und Erweiterungen: Stefan Jansen (HSA)
%%% V2.1 05/2021, Trennung von R-Support und einfachem LaTeX: Phillip Heidegger (HSA)
%%% V2.2 01/2024, Anpassung an THA-Layout
%%% V3   01/2024, I18n
%%% ---------------------------------------------------------------------------------
\documentclass[
	12pt,
	a4paper %
	,
	oneside % Fuer Veröffentlichung
	,
	titlepage,
	DIV=13,
	headinclude,
	footinclude=false %
	,
	cleardoublepage=empty %
	,
	parskip=half,
]{scrreprt}

\usepackage[utf8]{inputenc}
\usepackage[T1]{fontenc}
\usepackage[hidelinks]{hyperref}
\usepackage{xcolor}
\usepackage[printonlyused]{acronym}

\usepackage[
	authorName={Lukas Konietzka},
	authorEnrolmentNo={2122553},
	authorStreet={Sebastian-Kneipp-Gasse 6A},
	authorZip={86152},
	authorCity={Augsburg},
	authorEMail={lukas.konietzka@tha.de},
	authorPhone={+49\,172-2728-376},
	authorSignaturePlace={Augsburg},
	studyProgram={Informatik},
	thesisType={Bachelorarbeit},
	thesisTitle={Automatische Segmentierung von \\ Mikro-CT-Aufnahmen zur Untersuchung \\ zahnmedizinischer Strukturen},
	studyDegree={Bachelor of Science},
	faculty={{Fakultät für \\ Informatik}},
	topicAssignment={November 14, 2024},
	submissionDate={November 14, 2024},
	defenseDate={März 20, 2025},
	nonDisclosure={false},
	supervisor={Prof. Dr. Peter Rösch},
	supervisorDeputy={Prof. Dr. Gundolf Kiefer},
	language={de}
]{THA-Abschlussarbeit}

% Literaturdatenbank (.bib-Datei) aus Citavi o.ä.
\bibliography{Literatur_Abschlussarbeit}

\graphicspath{{Bilder/}}

\usepackage[
	format=plain,
	labelfont=bf,
	textfont=it,
	justification=raggedright,
	singlelinecheck=false
]{caption}
\DeclareCaptionLabelFormat{something}{#2.#1.}
\captionsetup[lstlisting]{labelformat=something}

\begin{document}
	% Sprachauswahl zum Umschalten innerhalb des Textes.
	% Alternativen: \thesisLanguage, ngerman, english
	\selectlanguage{\thesisLanguage}
	\pagenumbering{roman}
	\setcounter{page}{1}

	\THAtitlepage

	\let
	\cleardoublepage
	\relax

	%%% --------------------------------------------------
	%%% Kurzfassung
	%%% --------------------------------------------------
	\begin{abstract}
		\section*{Kurzfassung}
		Die vorliegende Arbeit befasst sich mit der Entwicklung einer Erweiterung zur
		Vereinfachung der anatomischen Segmentierung von Zähnen in 3D Slicer. Die
		anatomische Segmentierung ist ein wichtiger Prozess in der zahnmedizinischen
		Forschung, bei dem Mikro-CT-Bilder in ihre Hauptbestandteile, Dentin und
		Schmelz, unterteilt werden. Diese Segmentierung bildet die Grundlage für
		weiterführende Analysen, wie die 3D-Rekonstruktion von Zahnstrukturen oder das
		Training neuronaler Netze zur Karieserkennung. Der aktuelle Stand der
		Technik zeigt, dass bereits ein funktionierendes Verfahren zur anatomischen Segmentierung
		existiert, dessen Nutzung jedoch durch eine komplexe, terminal basierte Ausführung
		erschwert wird. Dies stellt insbesondere für Anwender in klinischen Praxen
		eine Hürde dar. Ziel dieser Arbeit ist es daher, eine benutzerfreundliche
		Lösung zu entwickeln, die die Funktionalität der anatomischen Segmentierung effizient
		in 3D Slicer integriert und deren Anwendung erleichtert. Im Laufe der
		Entwicklung entsteht so der \textit{Tooth Analyser}, der es möglich macht
		den Algorithmus der anatomischen Segmentierung mit nur wenigen Klicks über
		eine Benutzerschnittstelle zu starten und so die Benutzerfreundlichkeit
		erheblich zu steigern. Während der Entwicklung selbst entsteht nicht nur eine
		Software mit Funktionalität, es wird eine ganze Struktur beschrieben, mit der
		eine Verarbeitung von Mikro-CT-Bilder effizient, flexible und interaktiv
		durchgeführt werden kann. Neben dem Mehrwert, den die Anwendung für die
		praktizierenden Ärzte liefert, konnten auch viele Erfahrungen mit der Entwicklung
		in 3D Slicer gesammelt werden. So wird beispielsweise erläutert, wie die Plug-in-Infrastruktur
		der Plattform funktioniert und wie schließlich sichergestellt wird, das ein
		erstelltes Modul ein Softwareupdate von Slicer überlebt. Die detaillierte Evaluation
		des \textit{Tooth Analyser} wird zeigen, das die geforderte Integration eines
		bereits existierenden Verfahrens in die Plattform 3D Slicer durchaus als Erfolg
		betrachtet werden kann. Jedoch wird auch sichtbar, dass das Modul in mancher
		Hinsicht noch einschränkend ist und Wünsche offen lässt.
	\end{abstract}

	%%% --------------------------------------------------
	%%% Logo ToothAnalyser
	%%% --------------------------------------------------
	\clearpage
	\vfill
	\begin{figure}
		\centering
		\includegraphics[width=1\textwidth]{img/SlicerToothAnalyser.png}
	\end{figure}
	\vfill
	\clearpage

	%%% --------------------------------------------------
	%%% Inhaltsverzeichnis
	%%% --------------------------------------------------
	\tableofcontents

	%%% --------------------------------------------------
	%%% Verzeichnisse
	%%% --------------------------------------------------
	\listoffigures % Abbildungsverzeichnis
	\addcontentsline{toc}{chapter}{Abbildungsverzeichnis}

	%%% --------------------------------------------------
	%%% Abkürzungsverzeichnis
	%%% --------------------------------------------------

	\chapter*{Abkürzungsverzeichnis}
	\begin{acronym}
		[ITK-SNAP, BMBF] \acro{2D}{zweidimensionalen} \acro{3D}{dreidimensonale}
		\acro{8UInt}{8 bit unsigend integer} \acro{16Int}{16 bit sigend integer}
		\acro{CLI}{Kommandozeilenschnittstelle} \acro{CT}{Computertomografie} \acro{GB}{Gigabyte}
		\acro{GUI}{Grafische Benutzerschnittstelle} \acro{Html}{Hypertext Markup Language}
		\acro{ISQ}{Industrial Scan Quality} \acro{ITK}{Insight Toolkit} \acro{ITK-SNAP}{Insight Toolkit Snake Automatic Partitioning}
		\acro{JSON}{JavaScript Object Notation} \acro{LMU}{Ludwig-Maximilians-Universität München}
		\acro{MB}{Megabyte} \acro{MRT}{Magnetresonanztomografie} \acro{MHD}{Meta Header Data}
		\acro{MRML}{Medical Reality Modeling Language} \acro{MVC}{Model View Controller}
		\acro{NIfTI}{Neuroimaging Informatics Technology Initiative} \acro{NRRD}{Nearly Raw Raster Data}
		\acro{OCT}{optische Kohärenztomografie} \acro{PyPi}{Python-Paket-Index}
		\acro{SEM}{Slicer Erweiterungsmodul} \acro{SSH}{Secure Shell} \acro{THA}{Technische Hochschule Augsburg}
		\acro{UI}{Benutzerschnittstelle} \acro{UML}{Unified Modeling Language} \acro{UX}{Benutzererfahrung}
		\acro{VTK}{Visualization Toolkit} \acro{XML}{Extensible Markup Language}
		\acro{X-Ray}{Röntgenstrahlung}
	\end{acronym}
	\addcontentsline{toc}{chapter}{Abkürzungsverzeichnis}

	\renewcommand{\lstlistlistingname}{Quellcodeverzeichnis}
	\lstlistoflistings % Listings
	\addcontentsline{toc}{chapter}{Quellcodeverzeichnis}
	\listoftables % Tabellenverzeichnis
	\addcontentsline{toc}{chapter}{Tabellenverzeichnis}

	%%% --------------------------------------------------
	%%% Ab hier: Inhalt
	%%% --------------------------------------------------

	\cleardoubleoddpage
	\setcounter{page}{1}
	\pagenumbering{arabic}

	\chapter{Einleitung}
\label{chap:einleitung}

\cite{FOTACHE2015243} test
	\chapter{Anatomische Segmentierung}
\label{chap:theoretische_grundlagen} Die anatomische Segmentierung ist ein bestehendes
Verfahren, das bereits von der Klinik an der LMU eingesetzt wird, um segmentierte
Daten aus Mikro-\ac{CT}-Bildern gezielt zu gewinnen. Einen wichtigen Meilenstein
hierfür liefert Hofmann. Im Rahmen einer Bachelorarbeit an der Hochschule für
angewandte Wissenschaften in Augsburg unterstützte Herr Hofmann die
Kariesklassifizierung auf den Zahnkronen-\ac{CT}-Aufnahmen. Hierzu entwickelte
er ein Verfahren, das auf Basis von Schwellwertverfahren die Zahnsubstanzen Schmelz
und Dentin aus dem Originalbild herauslöst. Dabei wird die anatomische
Segmentierung als der Prozess verstanden, bei dem diese Gewebetypen gezielt voneinander
getrennt und visuell sowie rechnerisch unterscheidbar gemacht werden.

\begin{minipage}{0.40\textwidth}
	Durch die Segmentbetrachtung der beiden Gewebesubstanzen Schmelz und Dentin konnte
	\citet[S.~41]{hoffmann2020} eine gute Hilfe für die Befundung kariöse Stellen
	liefern. Ein Ergebnis aus der Arbeit von Hofmann sei in Abbildung \ref{fig:ergebnis_hoffmann}
	gezeigt. Hierfür entwickelte Hofmann ein prototypisches Verfahren innerhalb
	eines IPython Notebooks, mit dem es gelang ca. 250 Datensätze der Klinik
	automatisch aufzubereiten.
\end{minipage}
\hfill
\begin{minipage}{0.50\textwidth}
	\centering
	\includegraphics[width=0.7\textwidth]{img/ergebnis_hoffmann_2.jpg}
	\captionof{figure}{Reproduziertes Ergebnis der anatomischen Segmentierung} \label{fig:ergebnis_hoffmann}
\end{minipage}

Neben der Erstellung von segmentierten Daten stellt das Verfahren noch
sogenannte mediale Flächen zur Verfügung. Diese sind insbesondere für die Klassifizierung
von Karies notwendig. \citet[S.~42]{hoffmann2020} erklärt, das sich bei einer
Überlagert der medialen Flächen auf das Originale Bild auf einem Blick erkennen
lassen, ob die vorhandene Karies in diesen Schichten weiter als die Hälfte fortgeschritten
ist. Für die Erstellung der medial Flächen sind die segmentierte Daten eines
Zahnes zwingend erforderlich. Die Berechnung der medial Flächen erfolgte auf den
Segmenten mithilfe einer Distanztransformation und anschließender Kantenfindung \citep[vgl.][S.~42]{hoffmann2020}.

Die anatomische Segmentierung eines Zahnes – einschließlich der medialen Flächen
– erfolgt in mehreren algorithmischen Schritten, die in einer klar strukturierten
Pipeline ablaufen. Abbildung \ref{fig:anatomische_segmentierung} veranschaulicht
den groben Ablauf des Verfahrens, wobei kleinere Zwischenschritte ausgeklammert sind.
Der Prozess beginnt oben links und endet mit der Segmentierung der medialen
Flächen in der unteren rechten Ecke.

\begin{figure}[h]
	\centering
	\includegraphics[width=1\textwidth]{img/anatomischeSegmentierung.png}
	\caption{Algorithmische Formulierung der anatomischen Segmentierung nach
	\citet{hoffmann2020} (von links oben nach rechts unten)}
	\label{fig:anatomische_segmentierung}
\end{figure}

Wie \citet[S.~55]{hoffmann2020} beschreibt, kann dieses Verfahren bis zu einem gewissen
Fortschritt von Karies angewendet werden. Da Karies im Laufe der Zeit zur
Zersetzung des Zahns führt, entstehen dunkle Stellen im Schmelzbereich, die die
zusammenhängenden Grauwerte im Bild stören und somit die Segmentierung
erschweren. Eine weitere Herausforderung liegt in der Art der Bilddaten, für die
das Verfahren ursprünglich entwickelt wurde. Es basiert auf \ac{ISQ}-Bildern, die
im \ac{16Int}-Format vorliegen.

Die einzelnen Schritte in Abbildung \ref{fig:anatomische_segmentierung}
verdeutlichen, dass die Segmentierung stets einem spezifischen Muster folgt und das
Verfahren primär zwischen Dentin und Schmelz unterscheidet. Die Pulpa wird wie
bereits beschrieben in diesem Verfahren nicht berücksichtigt. Zudem wird ersichtlich,
dass sowohl Filter als auch Segmentierungstechniken essenziell für eine präzise
Trennung der Strukturen sind. Um ein besseres Verständnis für diesen Prozess zu gewinnen,
führt das folgende Kapitel in die anatomischen Strukturen eines Zahns ein und
zeigt, wie diese auf Mikro-\ac{CT}-Aufnahmen erkennbar sind. Dabei werden die grundlegenden
Techniken der Bildgewinnung und -bearbeitung näher erläutert.

\pagebreak
% ---------------------------------------------------------------------------------------

\section{Anatomische Zahnstrukturen in Mikro-CT-Bildern}
\label{sec:domänenspezifisch} Um nachvollziehen zu können, wie eine \ac{CT}-Aufnahme
technisch segmentiert und in ihre einzelnen Bestandteile zerlegt werden kann,
ist es zunächst essenziell, die anatomische Struktur des Zahns genau zu verstehen.
Dazu gehören die unterschiedlichen Gewebeschichten wie Zahnschmelz, Dentin und die
Pulpa, die sich nicht nur in ihrer physikalischen Zusammensetzung, sondern auch in
ihrer radiologischen Darstellung innerhalb der \ac{CT}-Bilder unterscheiden. Erst
mit diesem grundlegenden Verständnis lassen sich Methoden zur \ac{3D}-Bildbearbeitung
ableiten.

\begin{minipage}{0.40\textwidth}
	Die Abbildung \ref{fig:aufbau_eines_zahnes} zeigt den groben Aufbau eines Zahnes
	nach \citet[S.~17]{lehmann2012Zahnheilkunde}. Zu sehen ist, dass das Denit
	oder auch Zahnbein genannt den Großteil eines Zahnes einnimmt. Im Bereich der Zahnkrone
	wird das Dentin von Zahnschmelz überzogen. Der Zahnschmelz ragt in die
	Mundhöhle und ist nach \citet[S.~41]{lehmann2012Zahnheilkunde} das härteste Material
	im menschlichen Körper. In der Mitte des Zahnes befindet sich Weichgewebe, das
	als Pulpa bezeichnet wird vgl. \citep[vgl.][S.~15]{lehmann2012Zahnheilkunde}.
	Für die anatomische Segmentierung von Zahn-\ac{CT}-Aufnahmen spielen insbesondere
	die drei Hauptbestandteile des Zahns – Schmelz, Dentin und Pulpa – eine
	entscheidende Rolle.
\end{minipage}
\hfill
\begin{minipage}{0.50\textwidth}
	\centering
	\includegraphics[scale=0.50]{img/aufbau_eines_zahns.jpg}
	\captionof{figure}{Aufbau eines ganzen Zahnes nach \citet[S.~15]{lehmann2012Zahnheilkunde}}
	\label{fig:aufbau_eines_zahnes}
\end{minipage}

Diese Gewebearten weisen unterschiedliche Strukturen auf, die sich in den
verschiedenen Graustufen auf einem Mikro-\ac{CT}-Bild widerspiegeln. Die Pulpa, das
innere Weichgewebe des Zahns, unterscheidet sich dabei nur geringfügig vom
Hintergrund, da sie als einzige der drei Hauptstrukturen nur sehr wenig Röntgenstrahlen
absorbiert. Aufgrund dieser geringen Sichtbarkeit spielt die Pulpa in dieser
Arbeit eine untergeordnete Rolle und wird nicht in das Segmentierungsverfahren
einbezogen. Geht man von innen nach außen, folgt auf die Pulpa das Dentin, das auch
als Zahnbein bezeichnet wird. Laut \citet[S.~41]{lehmann2012Zahnheilkunde} handelt
es sich dabei um eine Hartsubstanz, die den Zahn im Kieferknochen hält. Aufgrund
seiner Dichte ist das Dentin in \ac{CT}-Aufnahmen bereits deutlich erkennbar. Die
äußerste Schicht des Zahns bildet der Zahnschmelz. Dieser stellt die härteste
Substanz im menschlichen Körper dar und erscheint auf den \ac{CT}-Bildern besonders
hell. Seine hohe Dichte sorgt für eine klare Abgrenzung zu den darunterliegenden
Schichten, was ihn zu einem wichtigen Orientierungspunkt für die Segmentierung macht
\citep[vgl.][S.~41]{lehmann2012Zahnheilkunde}. Mit der folgenden Abbildung \ref{fig:pulpa_dentin_schmelz}
werden die verschiedenen Gewebearten in einem Zahn den entsprechenden Graustufen
zugeordnet.

\begin{figure}[h]
	\centering
	\includegraphics[width=0.9\textwidth]{img/dentin_schmelz_pulpa.png}
	\caption{Darstellung von Pulpa, Dentin und Schmelz auf einem Zahnkronen-CT nach
	\citet{heck2024}}
	\label{fig:pulpa_dentin_schmelz}
\end{figure}

Zu Beachten ist hier, dass es sich bei der in Abbildung \ref{fig:pulpa_dentin_schmelz}
gezeigten \ac{CT}-Aufnahme um eine Zahnkrone handelt und nicht um den ganzen
Zahn. Zu sehen ist auch der fast identische Grauwert zwischen dem Hintergrund und
der Pulpa.

Mit diesem Wissen über die Anatomie eines Zahnes und die Zusammensetzung auf einem
Mikro-\ac{CT}, kann nun ein Schritt weiter gegangen werden, indem der Fokus auf
die \ac{CT}-Bilder gerichtet wird. Das folgende Kapitel bietet daher einen
Überblick über die Erstellung von Röntgenaufnahmen sowie die Berechnung einer
Mikro-\ac{CT}-Aufnahme. Durch die Bildgewinnung wir auch klar, mit welcher hohen
Auflösung gearbeitet wird und warum dies auch zu Hindernissen führt.
% ---------------------------------------------------------------------------------------

\section{Erstellung von Mikro-CT-Bildern}
\label{sec:technologisch} Die Erzeugung dreidimensionaler Bilddaten ist ein essenzieller
Bestandteil der modernen medizinischen Bildgebung und kann auf verschiedene Weise
erfolgen. Je nach Anwendungsgebiet kommen unterschiedliche Verfahren zum Einsatz,
darunter \ac{MRT}, \ac{OCT} und insbesondere die Computertomografie \citep[vgl.][S.~14]{handels2000}.
Eine spezielle Variante der Computertomografie ist das Mikro-\ac{CT}, das durch
seine kleinere Ausführung in der zahnmedizinischen Forschung und Diagnostik eine
zentrale Rolle spielt. Auch die Handhabung dieser erstellten Daten ist ein wichtiger
Bestandteil der Forschung.
% ---------------------------------------------------------------------------------------

\subsection{Computertomografie}
\label{subsec:computertomografie} Die Erfindung der Computertomografie (\ac{CT})
war ein Quantensprung in der Geschichte der Medizin. Sie ist aus heutigen Diagnosen
nicht mehr wegzudenken. Ein Mikro-\ac{CT}-Bild ist laut \citet[S.~1]{baird2017}
ein Menge hochauflösender Bilder, die wie ein Stapel zusammengelegt werden.
Vereinfacht gesagt, werden so \ac{2D} Bilder zu einem \ac{3D}-\ac{CT}
zusammengesetzt. Der Aspekt Mikro deutet dabei darauf hin, dass es eine
miniaturisierte Ausführung eines üblichen Kegelstrahl-\ac{CT}s ist, so \citet[S.~340]{buzug2011}.
Eine andere Definition erläutert \citet[S.~49]{lehmann2013bildverarbeitung}. Er beschreibt
die Computertomografie als Projektionen einzelner Ebenen im Untersuchungsobjekt.
Die Abbildung \ref{fig:spectrum} soll diesen Vorgang genauer beschreiben.

\begin{figure}[h]
	\centering
	\includegraphics[width=0.6\textwidth]{img/Funktion_CT.png}
	\caption{Funktionsweise eines Mikro-CT nach \citet[S.~16]{pult2021}}
	\label{fig:spectrum}
\end{figure}

Die Technologie, mit der diese Bilder aufgenommen werden, ist unter der Röntgentechnik
oder auch \ac{X-Ray} bekannt. Die Röntgenstrahlung ist eine Form der elektromagnetischen
Strahlung, ähnlich wie das sichtbare Licht, so das \citet[K.~1]{nib2024}. Anders
als das Licht haben die Röntgenstrahlen eine viel höhere Energie. Das führt dazu,
dass man mit dieser elektromagnetischen Strahlung viele Objekte durchdringt werden
können. So auch Gewebeteile eines Zahnes \citep[vgl.][K.~1]{nib2024}. Mit der
Steigerung der Atomzahl in einem Material, also der Dichte, nimmt auch die
Absorption eines Materials zu, sodass es leicht ist verschiedene Materialien in einer
\ac{CT}-Aufnahme zu unterscheiden \citep[vgl.][K.~1]{nib2024}. Um nun mittels
der Röntgenstrahlen und den einzelnen \ac{2D} Projektionen ein \ac{3D}-Micro-\ac{CT}
zu erstellen, werden 2D-Röntgenaufnahmen eines Objektes aus verschiedenen
Winkeln geschossen und in einem weiteren Schritt dann zu einem \ac{3D}-Modell rekonstruiert.
Das Ergebnis einer solchen Rekonstruktion ist dann eine Art Papierstapel. Dieser
besteht aus vielen 2D-Bilder, die aufgrund der Stapelung zu einem \ac{3D}-Bild werden.

Für die Erzeugung dieser Bilder bedarf spezieller Geräte, im Falle der
Zahnklinik an der \ac{LMU} handelt es sich um ein Mikro-\ac{CT}-Gerät der Firma
\citet{scanco2024}. Dieses Gerät erstellt Aufnahmen mittels Röntgenstrahlung und
generiert mithilfe der Computertomografie ein dreidimensionales Bild, welches im
Format \ac{ISQ} abgelegt wird. Das folgende Kapitel zeigt, dass Mikro-\ac{CT}-Bilder
mit dem Typ \ac{ISQ} zwar eine hohe Detailgenauigkeit bieten, jedoch auch einen erheblichen
Speicherbedarf mit sich bringen.
% ---------------------------------------------------------------------------------------

\subsection{Datenformate}
\label{subsec:datensätze} Die rohen Datensätze, welche direkt aus dem Mikro-\ac{CT}-Gerät
kommen, haben nach \citet{scanco2024} das Format \ac{ISQ}. Dieses Format fällt speziell
auf die Geräte der Firma SCANCO zurück. Wie das vorherige Kapitel
\ref{subsec:computertomografie} bereits eingeführt hat, ist dieser Dateityp für eine
weitere Bearbeitung nur bedingt geeignet. Unter anderem wegen ihrer Größe. \citet[S.~118-119]{RoeschKunzelmann2018}
haben ein Paket entwickelt, mit dem sich gu zeigen lässt, wie sich diese großen
Bilder zusammensetzten. Hierfür konvertiert das Paket ein \ac{ISQ}-Bild in ein
\ac{MHD} Format. Bei einer \ac{MHD}-Datei handelt es sich um ein Metafile, dass
auf die eigentliche Datei verweist. Wird dieses Paket demnach verwendet, so
erstellt das Skript \texttt{isq\_to\_mhd} ein Metafile, das detaillierte Daten
über die Datei enthält. Ein Ausschnitt dieses Metafiles liefer das Listing \ref{lst:inhalt_mhd_datei}.
Diese Metadatei kann genutzt werden, um interessante Informationen über das Bild
zu erlangen.

\begin{lstlisting}[
	caption={Ausschnitt des Inhaltes einer MHD-Datei},
	label={lst:inhalt_mhd_datei}]
ObjectType = Image
NDims = 3
CenterOfRotation = 0 0 0
ElementSpacing = 0.02 0.02 0.02
DimSize = 1024 1024 517
ElementType = MET_SHORT
ElementDataFile = P01A-C0005278.ISQ
\end{lstlisting}

In der Datei sind Informationen über die Ausprägung, Art und Größe der Datei zu finden.
Besonders interessant sind die Punkte \texttt{DimSize und ElementType}. Über
diese Parameter lässt sich die Größe eines Bildes berechnen. \citet[S.~10-11]{burger2009}
erklärt, dass ein Bild in Zellen aufgeteilt ist, welche Informationen enthalten.
Diese Zellen sind im zweidimensionalen Raum als Pixel bekannt. Betrachtet man
jedoch ein dreidimensionales Bild, so spricht man nicht mehr von einem Pixel,
sondern von einem Voxel. Ein Voxel ist demnach das dreidimensionale Äquivalent zu
einem Pixel. \citet[S.~10-11]{burger2009} beschreiben weiter das jeder diese Zellen
ein binäres Wort der Länge $2^{k}$ ist. Die Basis 2 ergibt sich durch das binäre
Wort, wo hingegen für $k$ gilt: $k \in \mathbb{N}$. Um für den konkreten Fall aus
Listing \ref{lst:inhalt_mhd_datei} das entsprechenden $k$ zu ermitteln, muss der
\texttt{ElementType} näher betrachtet werden. \texttt{MET\_SHORT} steht hierbei für
\textit{Signed short}, was eine Größe von 16 Bit entspricht. Damit ergibt sich
für die Länge $k$ ein Wert $k = 4$. So können nach \citet[S.~10-11]{burger2009}
folgende Gleichungen festgehalten werden, mit der die Größe eines Mikro-\ac{CT}-Bilds
erfasst werden können.

\begin{align}
	\label{equ:größe_bestimmen}1024 \cdot 1024 \cdot 517    & = 542,113,792 \, \text{Voxel}\notag  \\
	542,113,792 \, \text{Voxel}\cdot 2 \, \text{Byte/Voxel} & = 1,084,227,584 \, \text{Byte}\notag \\
	1,084,227,584 / 1,000,000,000                           & = 1.0842 \, \text{GB}
\end{align}

Die erste Gleichung bestimmt die Gesamtzahl aller Voxel in einem Bild. Gleichung
zwei ermittelt die Größe des Bildes in der Einheit Byte. Die letzte Zeile nimmt eine
Umrechnung von Byte nach \ac{GB} vor. Durch die Gleichungen in \ref{equ:größe_bestimmen}
wird klar, dass eine \ac{CT}-Aufnahme des Typs \ac{ISQ} direkt nach seiner
Aufnahme über einen \ac{GB} groß ist. Damit lässt sich zeigen, dass diese Form des
Mikro-\ac{CT} einen hohen Detailgrad aufweist. Es wird jedoch auch klar, dass
diese Größe an Bildern den alltäglichen Einsatz etwas erschweren, Da ein regelmäßiger
Umgang mit Ihnen erhebliche Speicherressourcen benötigt.

Nachdem dieses Kapitel erläutert hat, wie Mikro-\ac{CT}-Bilder erzeugt werden und
in welchem Datenformat sie vorliegen, beschäftigt sich das nächste Kapitel mit
der konkreten Bearbeitung der Mikro-\ac{CT}-Aufnahmen. Hierbei werden Konzepte
erläutert welche innerhalb der anatomischen Segmentierung zum Einsatz kommen. Dies
umfasst unter anderem die Rauschreduktion sowie die Segmentierung der relevanten
Strukturen. Daher werden wesentliche Methoden und Algorithmen zur Optimierung
und Analyse von Mikro-\ac{CT}-Aufnahmen vorgestellt, wobei insbesondere die
Filterung und die Segmentierung eine zentrale Rolle spielen.

\pagebreak
% ---------------------------------------------------------------------------------------

\section{Bearbeitung von Mikro-CT-Bildern}
\label{sec:bildbearbeitung} Nachdem ein \ac{CT} erzeugt wrude folgt die
Bearbeitung eines Bildes. Hierfür bietet das Pipeline-Modell von \citet[S.~50]{handels2000}
eine gute Richtlinie. Er beschreibt mit dieser Visualisierungs-Pipeline Schritte,
die bei der Bearbeitung von dreidimensionalen \ac{CT}-Aufnahmen notwendig sind \citep[vgl.][S.~50]{handels2000}.
Die ersten Schritte, \textit{Bildvorverarbeitung} und \textit{Segmentierung}, sind
von besonderem Interesse. Dieser Abschnitt orientiert sich an dieser
Unterteilung und nimmt sie sich als Vorbild. Daraus ergeben sich die Abschnitte \ref{subsec:filter}
Filter und \ref{subsec:segmentierung} Segmentierung, welche die Pipelineschritte
\textit{Bildvorverarbeitung} und \textit{Segmentierung} widerspiegeln sollen.
% ---------------------------------------------------------------------------------------

\subsection{Filterung}
\label{subsec:filter} \ac{CT}-Aufnahmen rauschen, dies ist ein Fakt und liegt in
der Natur einer Röntgenaufnahme. Dies beschreiben auch \citet[K.~3]{diwakar2018}
in ihrem Paper über \ac{CT}-Bildrauschen und Entrauschen. Dabei liegt die
Ursache des Rauschens nicht an einer Stelle, sondern ist auf viele Quellen zurückzuführen.
Eine gute Einteilung dieser Quellen liefern ebenfalls \citet[K.~3]{diwakar2018}.
Sie teilen die Rauschquellen auf in \textit{Random noise, Statistical noise,
Electronic noise} und \textit{roundoff noise}.

Unter dem Rauschen eines Bildes versteht man die Streuung der Pixelwerte im Bild.
Für eine Segmentierung des Bildes ist dieses Verhalten unerwünscht und führt zu schlechten
Ergebnissen \citep[vgl.][S.~51]{handels2000}. Die Bildvorverarbeitung oder auch Filter
genannt, hat die Aufgabe dieses Rauschen so gut wie möglich zu reduzieren. Hierzu
gibt es diverse Möglichkeiten.

\begin{minipage}{0.40\textwidth}
	Mit Blick auf die anatomische Segmentierung sind für diese Arbeit vor allem die
	lokalen Operatoren relevant. Die lokalen Operatoren sind charakteristisch für
	die Betrachtung der lokalen Nachbarschaft. Jeder Pixel betrachtet also seine Umgebung
	und führt auf Basis darauf eine Berechnung des jeweils betrachteten Pixels durch.
	In Abbildung \ref{fig:lokaler_operator_maske} ist der aktuellen Pixel der, mit
	der Position $P = (0/0)$ \citep[vgl.][S.~52]{handels2000}.
\end{minipage}
\hfill
\begin{minipage}{0.50\textwidth}
	\centering
	\includegraphics[width=0.60\textwidth]{img/lokaler_operator_maske.jpg}
	\captionof{figure}{Maske eines lokalen Operators nach \citet[S.~52]{handels2000}}
	\label{fig:lokaler_operator_maske}
\end{minipage}

Für die konkrete Betrachtung der Nachbarschaft eines Pixels empfiehlt \citet[S.~52]{handels2000}
eine Maske (Ausschnitt) heranzuziehen, die mit einer Matrix interpretiert werden
kann und die Nachbarschaft eines Pixels abbildet. Abbildung \ref{fig:lokaler_operator_maske}
zeigt eine solche Maske und soll das Verfahren so verdeutlichen. Der grau
hinterlegte Mittelpunkt $P = (0/0)$ ist das aktuell betrachtete Pixel. Die Felder
um die Mitte herum die Nachbarn. Es fällt jedoch auf, dass durch dieses Schema nicht
jede mögliche Ausprägung einer Maske infrage kommt. Um einen Mittelpunkt und
damit ein aktuelles Pixel betrachten zu können, bedarf es eines ungeraden Grades
für $M$. Diese Eingrenzung lässt sich in Gleichung \ref{equ:lokaler_operator}
generisch fassen.

\begin{align}
	\label{equ:lokaler_operator}M_{(2_m+1)x(2_m+1)} & = \begin{bmatrix}k_{11}&k_{12}&k_{13}\\ k_{21}&x&k_{23}\\ k_{31}&k_{32}&k_{33}\\\end{bmatrix} & m \in \mathbb{N}
\end{align}

Die Gleichung \ref{equ:lokaler_operator} beschreibt die mögliche Ausprägung eines
lokalen Operators als Matrix. Dabei sei: $m \in \mathbb{N}\wedge n \in \mathbb{N}$.
Die Variable $x$ beschreibt das aktuell betrachtete Pixel, während $k_{nn}$ die Nachbarn
illustrieren soll. Durch die Gleichung ist auch zu erkennen, dass die Maske des lokalen
Operators beliebig groß werden kann. Eine hohe Ordnung der Operatormatrix ist
jedoch nicht immer von Vorteil, sodass es letzten Endes auf den Anwendungsfall ankommt.

Mit der Technik der lokalen Operatoren können nun unterschiedliche Arten angewendet
werden. \citet[S.~54 - 55]{handels2000} unterscheidet hier in Glättungsfilter,
Mittelwertfilter, Medianfilter, Gaußfilter und Binomialfilter. Alle dieser Filter
bedienen sich einer Operatormaske, um auf Basis der Nachbarelemente einen
statistischen Wert für den Bildpunkt zu erhalten. Um einen genaueren Einblick in
alle Filter zu erlangen, sei an dieser Stelle auf \citet[S.~54 - 55]{handels2000}
verwiesen.

Wie zu Anfang dieses Kapitels beschrieben, ist eine Bildvorverarbeitung (Filterung)
für eine gute Segmentierung des Bildes unerlässlich. So kommt es das auch in der
Visualisierungspipeline nach \citet[S.~50]{handels2000} der zweite Schritt
bereits die Segmentierung einführt. Warum dies so ein wichtiger Bestandteil der Bildanalyse
ist und welche Methoden sich hier bieten, erläutert das folgende Kapitel.
% ---------------------------------------------------------------------------------------

\pagebreak

\subsection{Segmentierung}
\label{subsec:segmentierung} Die Bildsegmentierung oder auch Bildaufteilung genannt,
ist ein wichtiges Teilgebiet der Bildverarbeitung und beschäftigt sich mit der
Bildanalyse. Ihr Ziel ist es, detaillierte beschreibende Bilder aus dem
vorliegenden Originalbild zu berechnen. Dies kann im Falle eines \ac{CT}s der Zahnklinik
an der \ac{LMU} die hervorgehobene Darstellung der Zahnsubstanzen Schmelz und
Dentin sein. \citep[vgl.][S.~359]{lehmann2013bildverarbeitung}. Konkret teilt ein
Segmentierungsverfahren also ein Bild in Teilbereiche auf. Dabei sind die
Teilbereiche in sich bemerkenswert homogen. \citet[S.~1]{ramesh2021} beschreiben,
dass der Prozess der Segmentierung zur Gewinnung wichtiger Informationen dient wie
zum Beispiel die Zahnkaries Ausbreitung. So kommt es, dass \citet[S.~50]{handels2000}
in seiner Visualisierungspipeline die Segmentierung als zweiten Schritt und
damit als zentrales Problem darstellt. \citet[S.~95]{handels2000} und \citet[S.~360]{lehmann2013bildverarbeitung}
beschreiben beide, dass die Bildsegmentierung eines \ac{CT}s für eine gute und
eindeutige ärztliche Diagnose nicht mehr wegzudenken ist. Warum dem so ist,
verdeutlicht die Abbildung \ref{fig:interpretation_einer_ct_aufnahem}.

\begin{figure}[h]
	\centering
	\includegraphics[width=0.8\textwidth]{img/bild_interpretation.jpg}
	\caption{Interpretation einer CT-Aufnahme nach \citet[S.~360]{lehmann2013bildverarbeitung}}
	\label{fig:interpretation_einer_ct_aufnahem}
\end{figure}

Zu erkennen ist das originale Bild (Ausgangslage) und mögliche Interpretationsschritte
(Interpretation 1 und Interpretation 2). \citet[S.~360]{lehmann2013bildverarbeitung}
verdeutlichen mit dieser Abbildung \ref{fig:interpretation_einer_ct_aufnahem},
dass mittels Segmentierung die einzig mögliche Interpretation die Erste ist. Auch
wenn die zweite Interpretation die deutlich logischere ist, kann diese ohne
weitere Forschung nicht bewiesen werden, so \citet[S.~360]{lehmann2013bildverarbeitung}.
Außerdem ist zu erkennen, dass die Abbildung
\ref{fig:interpretation_einer_ct_aufnahem} die Definition einer Segmentierung
belegt. Die Erzeugung inhaltlich zusammengehöriger Regionen werden hier durch die
verschiedenen Formen visualisiert \citep[vgl.][S.~360]{lehmann2013bildverarbeitung}.

Um ein Bild zu segmentieren, gibt es unzählige Möglichkeiten. Für die Auswahl
eines Verfahrens spielt unter anderem der Anwendungsbereich eine wichtige Rolle.
Die Verfahren, die in dieser Arbeit von Wichtigkeit sind, sind die Schwellwertverfahren
\citep[vgl.][S.~361]{lehmann2013bildverarbeitung}.

\pagebreak

\textbf{Schwellwertverfahren} (engl.: thresholding) gehören zu den
Standardwerkzeugen einer Segmentierung, sodass diese die Basis vieler weiterer Verfahren
legen. Bei einer schwellwertbasierten Segmentierung werden die Pixel eines
Bildes anhand von Schwellwerten eingruppiert \citep[vgl.][S.~96]{handels2000}.
Die nachfolgende Gleichung \ref{equ:schwellwertverfahren} soll dies
verdeutlichen.

\begin{align}
	\label{equ:schwellwertverfahren}B(x, y, z) = \begin{cases}1,&\text{falls }t_{\text{unten}}\leq f(x, y, z) \leq t_{\text{oben}}, \\ 0,&\text{sonst}.\end{cases}
\end{align}

$B(x, y, z)$ beschreibt ein Pixel in einem dreidimensionalen Bild, demnach ein
Voxel. Liegen die Werte eines Voxels, also $f(x, y, z)$, innerhalb der beiden Schwellwerte
$t_{oben}$ und $t_{unten}$, dann wird eine 1 zugewiesen. Liegt der aktuell betrachtete
Voxel nicht zwischen den Schwellwerten, so wird eine 0 zugewiesen. Das Ergebnis einer
solchen primitiven Schwellwertsegmentierung ist ein binäres Bild, welches in
Abbildung \ref{fig:binäres_schwellwertverfahren} zu sehen ist.

\begin{figure}[h]
	\centering
	\includegraphics[width=0.8\textwidth]{img/beispiel_schwellwertverfahren.png}
	\caption{Ergebnis eines einfachen Schwellwertverfahrens nach \citet{heck2024} und
	\citet{hoffmann2020}}
	\label{fig:binäres_schwellwertverfahren}
\end{figure}

Zu erkennen ist, dass nach einem einfachen Schwellwertverfahren das Bild nur
noch aus zwei unterschiedlichen Graustufen besteht. Betrachtet man das Ergebnis in
\ref{fig:binäres_schwellwertverfahren} genauer, so ist diese einfache
Segmentierung durchaus erfolgreich verlaufen. Der Grund dafür ist die gute Wahl des
Schwellwerts.

Die interessanteste Frage bei den Schwellwertverfahren ist die Wahl des Schwellwerts
$t$. Dieser entscheidet zwischen einer guten und einer schlechten Segmentierung.
Für die Wahl eines Schwellwerts empfiehlt sich der Blick auf das Bildhistogramm.
Dieses gibt Aufschluss über die Verteilung der Grauwerte in einem Bild \citep[vgl.][S.~361]{lehmann2013bildverarbeitung}.
Verfahren, welche einen guten Schwellwert gewährleistet, ohne dass zu viele
Informationen verloren gehen, sind die Verfahren \textit{Otsu} und \textit{Renyi}.

\pagebreak

\textbf{Das Verfahren nach Otsu} gehört zu den Schwellwertverfahren und bestimmt
den Schwellwert $t$ durch ein statistisches Gütekriterium. Hierzu bedient sich
das Verfahren des Bildhistogramms. Die räumliche Anordnung der Voxel und damit das
tatsächliche Bild, benötigt dieser Algorithmus nicht \citep[vgl.][S.~264]{lehmann2013bildverarbeitung}.

\begin{minipage}{0.40\textwidth}
	Ein solches Histogramm, welches die Grundlage für das Verfahren nach Otsu
	liefert, sei in Abbildung \ref{fig:histogramm} gezeigt. Dies gibt Aufschluss
	über die unterschiedlichen Grauwerte und wie oft sie in einem Bild vorkommen
	\citep[vgl.][S.~264]{lehmann2013bildverarbeitung}. Für eine genauere
	Beschreibung eines Histogramms sei an dieser Stelle auf \citet[S.~42]{burger2009}
	verwiesen.
\end{minipage}
\hfill
\begin{minipage}{0.50\textwidth}
	\centering
	\includegraphics[width=1\textwidth]{img/histogramm.jpg}
	\captionof{figure}{Histogramm einer CT-Aufnahme einer Zahnkrone nach \citet[S.~13]{hoffmann2020}}
	\label{fig:histogramm}
\end{minipage}

Das Otsu-Verfahren teilt die Grauwerte eines Bildes in verschiedenen Klassen ein,
die durch Schwellwerte getrennt werden. Die Klassen können beispielsweise mit
$K_{0}$ bis $K_{n}$ bezeichnet werden, wobei sich dieses konkrete Beispiel auf die
Klassen $K_{0}$ und $K_{1}$ beschränkt. Otsu wählt den Schwellwert $t$, der die Varianz
zwischen den Pixelklassen maximiert und gleichzeitig die Varianz innerhalb jeder
Klasse minimiert \citep[vgl.][S.~264]{lehmann2013bildverarbeitung}. Mathematisch
lässt sich dies wie folgt ausdrücken.
\begin{align}
	t = \text{max }(\sigma_{zw}^{2}/ \sigma_{in}^{2})
\end{align}
$\sigma_{zw}$ bildet die Varianz zwischen den beiden Klassen $K_{0}$ und $K_{1}$
und bildet sich aus den Wahrscheinlichkeiten, mit denen jeder einzelne Grauwert auftritt.
$\sigma_{in}$ hingegen, ist die Varianz innerhalb einer Klasse und entsteht durch
die Addition der Varianzen der einzelnen Klassen. Der Schwellwert $t$ ist nun
der, für den das Verhältnis maximal wird \citep[vgl.][S.~264]{lehmann2013bildverarbeitung}.

Laut \citet[S.~264]{lehmann2013bildverarbeitung} fällt auf, dass dieses
Verfahren vor allem bei bimodalen Bildern zum Einsatz kommt. Ein Bild ist bimodal,
wenn es zwei lokale Maxima aufweist. Vereinfacht gesagt, wenn es zwei lokale
Piecks enthält \citep[vgl.][S.~264]{lehmann2013bildverarbeitung}.

Eine ähnliche Technik für die Bestimmung des Schwellwerts liefert das Verfahren der
Rényi Entropie. Auch hier ist eine Einteilung der Voxel in Klassen vorgesehen.

\pagebreak
% ---------------------------------------------------------------------------------------

\textbf{Das Verfahren nach Rényi} ist ein weiteres Verfahren, das im Laufe dieser
Arbeit noch eine wichtige Rolle spielt. Wie bereits beschrieben gehört es
ebenfalls zu der Gruppe der Schwellwertverfahren und generiert demnach einen
Schwellwert $t$. Wie auch das Verfahren nach Otsu, benötigt Rényi keine
Information über die räumliche Anordnung der Bilder, es genügt das Bildhistogramm.
Dabei ist der optimal Schwellwert $t$ der, der eine maximale Entropie der Bildverteilung
erzeugt. Unter einer Entropie wird ein Konzept verstanden, das eine Unordnung,
Unsicherheit oder den Informationsgehalt innerhalb eines Systems beschreibt, so \citet[S.~102]{bein2006}.
Die Rényi-Entropie ist eine Verallgemeinerung der Shannon-Entropie und hängt von
einem Parameter $q$ ab. Die Entropie misst die Unsicherheit oder den
Informationsgehalt einer Wahrscheinlichkeitsverteilung, welche sich wie folgt ausdrücken
lässt. \citep[vgl.][K.~2]{bromiley2004}.
\begin{align}
	\label{equ:renyi}H_{q}(P) = \frac{1}{1-q}\ln \left( \sum_{i=1}^{N}p_{i}^{q}\right)
\end{align}
Besonderes Augenmerk verdienen hierbei die Parameter $p_{i}$ und $q$, welche die
charakteristischen Eigenschaften der Rényi-Entropie beschreiben. Der Parameter
$p_{i}$ ist die Wahrscheinlichkeit eines jeden Grauwertes im Bild. $i$ symbolisiert
hierbei jeden Grauwert. Wie viele Grauwerte genau betrachten werden sollen
definiert $N$. Die Variable $q$ hingegen beeinflusst die Gewichtung der Wahrscheinlichkeit
$p_{i}$ für jeden Grauwert. Setzt man den Parameter $q$ auf $q = 1$ so lässt
sich mittels Algebra die Shannon-Entropie zeigen \citep[vgl.][K.~2]{bromiley2004}.
Um nun mit der Rényi-Entropie den optimalen Schwellwert für ein Bild zu
berechnen, sieht Rényi ähnlich wie Otsu eine Einteilung in Klassen vor. Die
Einteilung erfolgt mittels des Parameters $N$. So kann nun für jede gebildete Klasse
die Gleichung \ref{equ:renyi} angewendet werden. Die Gesamtentropie des Systems
wird aus den beiden Teilentropien der jeweiligen Klassen bestimmt\citep[vgl.][K.~2]{bromiley2004}.
\begin{align}
	H_{q}(T) = H_{q}(P)^{(1)}+ H_{q}(P)^{(2)}
\end{align}
Um nun den optimalen Schwellwert $t$ bestimmen zu können muss der Wert genommen werden,
bei dem die Gesamtentropie des Systems maximal ist. Dieser Sachverhalt lässt sich
wie folgt ausdrücken \citep[vgl.][K.~2]{bromiley2004}.
\begin{align}
	t = max(H_{q}(T))
\end{align}
Gegen Ende des Kapitels konnte ein grundlegendes Verständnis über die anatomische
Segmentierung einer Zahn-\ac{CT}-Aufnahme gewonnen werden. Um dieses Verfahren,
wie von der Fragestellung gefordert, für den klinischen Alltag zugänglich zu machen,
ist eine benutzerfreundliche Lösung erforderlich. Im folgenden Kapitel wird daher
die automatische Segmentierung mit 3D Slicer behandelt, einer etablierten
Plattform, die eine effiziente Integration und Nutzung des Verfahrens ermöglicht.
% ---------------------------------------------------------------------------------------
	\chapter{Automatische Segmentierung mittels 3D Slicer}
\label{sec:3d_slicer} 3D Slicer ist eine Open-Source-Plattform, die speziell für
die Verarbeitung von Bilddaten im medizinischen Kontext eingesetzt wird. Dabei wird
sie von einer aktiven Community regelmäßig gewartet und weiterentwickelt \citep[vgl.][]{slicer2024},
\citep[vgl.][S.~1325]{fedorov2012slicer}. Für Slicer gibt es offiziell keine Nutzungsbeschränkung.
Jedoch sei auch gesagt, dass 3D Slicer nicht für die klinische Nutzung zugelassen
ist. \citet[S.~1331]{fedorov2012slicer} machen deutlich, dass 3D Slicer
ausschließlich für die Forschung gedacht ist. Um einen ersten Überblick über die
Komponenten von Slicer zu erlangen, soll die Abbildung
\ref{fig:3d_slicer_oekosystem} betrachtet werden.

\begin{figure}[h]
	\centering
	\includegraphics[width=1\textwidth]{img/3d_slicer_overview.jpg}
	\caption{3D Slicer Ökosystem nach \citet[S.~1326]{fedorov2012slicer}}
	\label{fig:3d_slicer_oekosystem}
\end{figure}

\citet[S.~1326]{fedorov2012slicer} teilt mit der Abbildung
\ref{fig:3d_slicer_oekosystem} die Plattform in drei Schichten auf. Auf der Obersten
wird klar, dass 3D Slicer aus der Kernanwendung und den installierbaren Modulen
besteht. Neben den bereits vorhandenen Modulen können von externen Entwicklern Module
über die Slicer Erweiterung entwickelt und bereitgestellt werden. Um eine
Weiterentwicklung möglich zu machen hat Slicer eine Reihe von Abhängigkeiten, die
jedoch portabel gehalten werden. Auf der untersten Schicht sind die
Plattformspezifischen Anforderungen zu sehen, die Slicer erfüllen soll. So kommt
es, dass das 3D Slicer Ökosystem sich durch einige Kriterien auszeichnet, die es
besonders attraktiv für die Bearbeitung von medizinischen Bilddaten machen. Zu den
wichtigsten Vorteilen gehört die Tatsache, dass die Software kostenfrei
verfügbar ist. Darüber hinaus bietet sie eine umfassende Plug-in-Infrastruktur,
die über den sogenannten Extension Manager zugänglich ist. Dies ermöglicht eine einfache
Erweiterung der Funktionalitäten nach Bedarf. Ein weiteres herausragendes
Merkmal ist die Möglichkeit, Skripte direkt in der integrierten Python-Konsole auszuführen,
was eine flexible und effiziente Automatisierung von Prozessen ermöglicht. Schließlich
ist 3D Slicer besonders für seine Vielseitigkeit bekannt, da es medizinische
Bilddaten aus sämtlichen Bereichen der Medizin – von Kopf bis Fuß – verarbeiten
kann.

3D Slicer hat für alle diese Punkte jeweils eine Lösung entwickelt, wobei der erste
Punkt durch die Open-Source-Philosophie schon gegeben ist. Die folgenden
Abschnitte decken diese Lösungen ab und bilden so eine erste Grundlage für die
Entwicklung mit 3D Slicer.
% ---------------------------------------------------------------------------------------

\section{Plug-in-Infrastruktur}
Der wohl bedeutendste Punkt ist die Plug-in-Infrastruktur, welche Slicer von sich
aus mitbringt. Um dieses Konzept genauer zu beleuchten, teilt man die Plattform am
besten in zwei Teile auf, die Kernanwendung und die Module, welcher jeder Anwender
personalisiert installieren oder deinstallieren kann. Diese Module werden als
\textit{Slicer lodabel module} bezeichnet \citep[vgl.][S.~1332]{fedorov2012slicer}.
Slicer realisiert die Struktur durch den \textit{Extension Manager}, welcher durchaus
vergleichbar ist mit einer Art App-Store. Über diesen können bequem und mit
wenig Klicks die gewünschten Erweiterungen in das Kernsystem installiert werden.
Neben der Möglichkeit Module zu installieren bietet Slicer noch die Möglichkeit eigenen
Module zu bauen und Sie im \textit{Extension Manager} zu veröffentlichen. Diese werden
als \ac{SEM} bezeichnet. Abbildung \ref{fig:3d_slicer_extension_index} soll diesen
Vorgang verdeutlichen \citep[vgl.][]{slicer2024}.

\begin{figure}[h]
	\centering
	\includegraphics[width=0.7 \textwidth]{img/slicer_extention_index.png}
	\caption{Funktionsweise der Plug-in-Infrastruktur von 3D Slicer nach \citet{extensionsIndex2024}}
	\label{fig:3d_slicer_extension_index}
\end{figure}

Slicer realisiert dies, indem die Plattform über ein zusätzliches Repository
verfügt, dass sich \href{https://github.com/Slicer/ExtensionsIndex?tab=readme-ov-file}{\textit{ExtensionIndex}}
nennt. Dieses öffentliche Repository ist eine Auflistung aller \ac{SEM}. Die
Auflistung erfolgt durch eine Reihe an \ac{JSON} Dateien, die auf die
Repositorien der einzelnen Entwickler verweisen. Dieser \href{https://github.com/Slicer/ExtensionsIndex?tab=readme-ov-file}{\textit{ExtensionIndex}}
ist über die Slicer Factory an den Extention Server und damit auch an den Extention
Manager angebunden. Die Slicer Factory ist ein System, das aus einem Slicer Extention
Repository ein lauffähiges Build erstellt, welches in den Extention Katalog
eingebunden werden kann. Ist eine Erweiterung in dem Erweiterungskatalog gelistet,
so sorgt der \textit{Extension Manager} dafür, dass die von der Slicer Factory
erstellt Build-Datei installiert werden kann.

Während die Module von Slicer gebaut werden, werden parallel auch immer die Tests
für die Kernanwendung ausgeführt. Diese sind über ein Dashboard einsehbar. So wird
sichergestellt, dass keines der Module einen Fehler im Kernsystem verursacht. Außerdem
ist so für alle Benutzer von Slicer transparent einsehbar, in welchem Zustand
sich die Software gerade befindet. Die Kernanwendung von 3D Slicer folgt einem Entwurfsmuster,
dass sich \ac{MVC} nennt. Bei der Erstellung einer \ac{SEM} soll dieser Ansatz ebenfalls
gepflegt werden. Eine High Level Betrachtung der Softwarearchitektur von 3D
Slicer bietet \cite[S.~1332]{fedorov2012slicer} mit der Abbildung
\ref{fig:3d_slicer_architektur}.

\begin{figure}[h]
	\centering
	\includegraphics[width=0.7\textwidth]{img/3d_slicer_architektur.jpg}
	\caption{3D Slicer High Level Architektur nach \citet[S.~1332]{fedorov2012slicer}}
	\label{fig:3d_slicer_architektur}
\end{figure}

Das Zusammenspiel zwischen \ac{MRML}, \ac{GUI} und der Logik bilden das MVC-Pattern
in der Kernanwendung. Das identische Pattern spiegelt sich auch in den einzelnen
Modulen von Slicer wieder. So wird sichergestellt, dass ein Softwareentwicklungsparadigma
eingehalten wird, was sich \textit{separation of concerns} nennt. Die Kapselung
von zusammengehöriger Logik. Bei der Erstellung einer eigenen Erweiterung ist die
Idee, dass nur die Logik implementiert werden muss und die komplexe Architektur
von Slicer erstmal nicht relevant ist.

Jedoch bietet sich in Slicer nicht nur die Möglichkeit eigene Erweiterungen zu
erstellen, es lässt sich hierfür auch die integrierte Python-Konsole nutzen.
% ---------------------------------------------------------------------------------------

\section{Python-Umgebung}
\label{subsec:pythob_umgebung} 3D Slicer bringt eine integrierte Python-Konsole mit,
über die mit der Datenstruktur interagiert werden kann. So ist es möglich,
Python- Skripte direkt in der Konsole auszuführen. Um dies zu realisieren, bringt
Slicer mit der Installation im jeweiligen Betriebssystem eine eigene Python-Umgebung
mit. Dieses sieht wie folgt aus.

\begin{center}
	\texttt{./Slicer/bin/PythonSlicer}
\end{center}

Diese Python-Umgebung verfügt über alle notwendigen Abhängigkeiten und Pakete.
Bei der Entwicklung eines \ac{SEM} kann dann auf das \ac{PyPi} in der
integrierten Python-Umgebung zurückgegriffen werden. So kommt es, dass für eine
Entwicklung mit Slicer. keine eigene Python-Umgebung auf der lokalen Maschine installiert
sein muss. Slicer bringt hier alles mit.

Für den letzten charakteristischen Punkt von Slicer aus Kapitel
\ref{sec:3d_slicer} führt der nächste Abschnitt in die durchaus komplexe
Datenstruktur \ac{MRML} ein, die bei einer Entwicklung mit Slicer unausweichlich
zu berücksichtigen ist.
% ---------------------------------------------------------------------------------------

\section{MRML-Datenstruktur}
\label{subsec:mrml_datenstruktur} Die \ac{MRML}, gesprochen \textit{"Murlm"} ist
ein Datenmodell, das dafür entwickelt wurde, alle möglichen Bilddaten zu visualisieren
und zu speichern, die für einen klinischen Zweck Einsatz finden \citep[vgl.][]{slicer2024}.
Laut \citet{slicer2024} wurde die \ac{MRML}-Datenstruktur völlig unabhängig von
der Slicer Kernanwendung entwickelt. Dies ermöglicht ein Portieren der Datenstruktur
auf andere Softwareapplikationen. Da Slicer die einzig große Plattform ist, die diese
Datenstruktur nutzt, wird der Quellcode für \ac{MRML} im Repository von 3D Slicer
gewartet und weiterentwickelt, so \citet{slicer2024}. Durch den Artikel von
\citet[S.~1331]{fedorov2012slicer} wird klar, dass \ac{MRML} mehr ist also nur eine
Datenstruktur. Sie beschreiben \ac{MRML} als Szenenorganisator von Bilder,
Annotationen, Layouts und Anwendungsdaten. \citet[S.~1327]{fedorov2012slicer} beschreiben
die \ac{MRML}-Datenstruktur als Schlüsselkomponenten innerhalb von 3D Slicer. Dies
ist auf die Softwarearchitektur von Slicer zurückzuführen, die in Abbildung
\ref{fig:3d_slicer_architektur} beschrieben wurde. Die Kernanwendung von Slicer arbeitet
wie bereits beschrieben nach dem \ac{MVC}-Pattern. \ac{MRML} übernimmt hier den
Teil des \textit{Models (M)} und bildet damit den Grundstein der Anwendung \citep[vgl.][S.~1332]{fedorov2012slicer}.
\citet{slicer2024} und der Artikel von \citet[S.~1327]{fedorov2012slicer} beschreibt
\ac{MRML} als \ac{XML}-Format. Wird also eine \ac{MRML}-Szene gespeichert, so
folgt eine Speicherung als \ac{MRML}-Datei und damit unter der Haube als \ac{XML}-Datei.
Dabei wird laut \citet{slicer2024} nur eine Referenz auf das Bild gespeichert. Die
zu bearbeitende Aufnahme selbst wird nicht innerhalb einer \ac{MRML}-Datei
abgespeichert. \ac{MRML} zeichnet sich vor allem dadurch aus, dass es eine Vielzahl
an Dateiformaten akzeptiert. Alle Formate, die für einen klinischen Zweck verarbeitet
werden, können durch \ac{MRML} unterstützt werden. Um dies zu gewährleisten, ist
die \ac{MRML}-Szene in sogenannte Knoten (engl.: nodes) aufgeteilt. Die Basistypen
für einen Knoten folgen der \citet{slicer2024} und sind in der folgenden
Aufzählung zu sehen.

\begin{minipage}{0.45\textwidth}
	\begin{itemize}
		\item Data node

		\item Display node

		\item Storage node

		\item View node
	\end{itemize}
\end{minipage}
\hfill
\begin{minipage}{0.45\textwidth}
	\begin{itemize}
		\item Plot node

		\item Subject hierarchy node

		\item Sequence node

		\item Parameter node
	\end{itemize}
\end{minipage}

Wird also ein Bild in eine \ac{MRML}-Szene geladen, so speichert Slicer die unterschiedlichen
Eigenschaften eines Bildes in unterschiedlichen Knoten. So werden Beispielsweise
Basiseigenschaften einer Probe im \textit{Data node} gespeichert, wo hingegen
ein \textit{Storage node} beschreibt wie ein Bild auf der Festplatte gespeichert
wird. In \textit{Display node} werden die Eigenschaften zur Darstellung eines Bildes
hinterlegt. Der Hintergrund für die Speicherung von Probendaten in
unterschiedlichen Knoten ist, dass beispielsweise dasselbe Bild in unterschiedlichen
Formaten vorliegt oder ein und dasselbe Bild auf zwei unterschiedliche Arten visualisiert
werden soll. So kann sich beispielsweise eine Struktur wie in Abbildung
\ref{fig:3d_slicer_class} ergeben.

\begin{figure}[h]
	\centering
	\includegraphics[width=1\textwidth]{img/slicer_class_index.jpg}
	\caption{Verkettung der einzelnen Knoten in der MRML-Datenstruktur nach
	\cite{slicer2024}}
	\label{fig:3d_slicer_class}
\end{figure}

Die Informationen in einem Bild werden also über diese Typen aufgeteilt und je nach
Sinn abgespeichert. Möchte man demnach auf die bestimmte Informationen in einer
Probe zugreifen. So kann diese Information über den Aufruf bestimmter Methoden erfolgen

\begin{lstlisting}[
	language={python},
	caption={Auslesen der Informationen aus den verschiedenen Knoten},
	label={lst:_auslehsen_nodes}]
# data node - vtkMRMLVolumeNode
currentVolume.GetImageData()
# storage node - vtkMRMLStorableNode
currentVolume.GetStorageNode()
# display node - vtkMRMLDisplayableNode
currentVolume.GetDisplayNode()
\end{lstlisting}

Wie die Kommentare in Listing \ref{lst:_auslehsen_nodes} bereits zeigen, gibt es
noch eine Besonderheit von \ac{MRML}. Damit eine Verwaltung aller Dateiformate
möglich ist, bedient sich \ac{MRML} einiger Tools, die sich bereits etabliert haben.
Die Wichtigsten sind hierbei \ac{VTK} und \ac{ITK} \citep[vgl.][K.~1.1]{vtk2006},
\citep[vgl.][K.~1.1]{itkguide2015}. Die beiden Tools sind echte Riesen in ihrer
Branche. \ac{MRML} nutzt diese, um einige Dateiformate zu lesen und zu schreiben.
Durch das Betrachten der \ac{MRML}-Szene wird klar, dass Slicer hierdurch viele
Möglichkeiten bietet. Speziell für die effiziente Speicherung der Proben in einer
Szene durch die unterschiedlichen Knotentypen. Ein besonderer Knoten, der gleichzeitig
auch die Brücke zu der interaktiven Benutzerschnittstelle von Slicer baut, ist
der \textit{Parameter node}. Warum dieser eine zentrale Rolle spielt und wie
Slicer die Schnittstelle grundsätzlich gestaltet, soll in Kapitel \ref{subsec:benutzerschnitstelle}
Benutzerschnittstelle diskutiert werden.

Mit dem Ende dieses Kapitels wurden alle wichtigen theoretischen Grundlagn beaprochen
die notwendig sind um die anatomische Segmentierung über ein 3D Slicer Modul bereitzuetllen.
Bevor die konkrete Methodik thematisiert wird, die für die Entwicklung des Moduls
angewendet wird, sollen im nächsten Kapitel kurz die Rahmenbedingungen gestzten
werden, innerhalb deren entwickelt werden soll.
% ---------------------------------------------------------------------------------------
	\chapter{Entwicklungsumgebung}
\label{chap:entwicklungsumgebung} Da bereits ein Framework feststeht, mit dem
gearbeitet werden soll, ist keine weitere Forschung nötig, um die richtige
Programmiersprache auszuwählen. Jedoch gibt es eine kleine Auswahl zu treffen.
3D Slicer unterscheidet zwischen zwei Arten von Modulen, die \ac{CLI}-Module, welche
in der Sprache C++ geschrieben werden und die \textit{Scripted Moduls}, die eine
Python Implementierung verlangen. Da die anatomische Segmentierung ohnehin in
einem IPython-Notebook bereitliegt, fiel die Wahl hier auf die Scripted Moduls.
So kann auch die breite Palette der Python-Pakete genutzte werden. Für eine detaillierte
Beschreibung des Frameworks selber sei an dieser Stelle auf das Kapitel \ref{sec:bildbearbeitung}f
verwiesen, indem das Framework und alle zugehörigen Eigenheiten noch genauer beschrieben
wurden. Um den Entwicklungsprozess etwas zu vereinfachen, wurde während der Entwicklung
auf ein Modul von Slicer zurückgegriffen, das speziell für Entwickler entworfen
wurde. Die Abbildung \ref{fig:entwicklungsumgebung} verdeutlicht dieses Tool.

\begin{figure}[h]
	\centering
	\includegraphics[width=0.6\textwidth]{img/Entwicklungsumgebung.png}
	\caption{Umgebung während der Entwicklung mit 3D Slicer und PyCharm}
	\label{fig:entwicklungsumgebung}
\end{figure}

Mit dem Debugging Tool lässt sich eine gewohnte Umgebung reproduzieren, in der der
Quellcode Schritt für Schritt analysiert werden kann \citep[vgl.][]{slicerdebuggingtools}.
Speziell bei der Fehlersuche ist dieses Tool eine sehr gute Unterstützung. Die
Abbildung beschreibt weiter, das als Umgebung für das Erstellen des Programmcodes
die Software Pycharm verwendet wird. Pycharm ist eine Lösung der Firma Jetbrains,
für das Erstellen von Python-Quellcode. Dieses Tool bietet eine breite Palette
an Funktionalitäten, die das Erstellen von Software vereinfachen und als \textit{State
of the Art} bezeichnet werden kann \citep[vgl.][]{jetbrains2024}.

Für die Interaktion mit der Slicer Kernanwendung stellt der Python-Interpreter das
Paket \texttt{slicer} zur Verfügung. Hierdurch lassen sich diverse Mechanismen steuern.
Wichtig ist hierbei, dass das Paket \texttt{slicer} nicht auf \ac{PyPi} zu finden
ist. Es ist nur in der Python umgebung vorhanden, die mit der Slicer Installation
einherget. Neben dem Paket \texttt{slicer} sind noch viele weitere Pakete verfügbar,
die sich in der medizinischen Bildbearbeitung durchgesetzt haben. Für eine komplette
Auflistung aller Python-Pakete sei an dieser Stelle auf die Dokumentation von Slicer
verwiesen \citep[vgl.][]{slicer2024}.

Damit während der Entwicklung auch Teilbereiche der Software bereits getestet
werden können, werden Testdaten benötigt. Bei diesen Testdaten handelt es sich
um originale Mikro-\ac{CT}-Aufnahmen als \ac{ISO}. Diese wurden auf einem Server
an der \ac{LMU} in München bereitgestellt. Der Zugiff auf den Server erfolgt über
den \textit{x2goclient}. Heruntergeladen wurden die Daten über eine \ac{SSH}
Verbindung. heruntergeladen werden. Bei den Mikro-CT-Bildern handelt es sich
ausschließlich um Aufnahmen der Zahnkrone. Mit dem Zugriff auf den Server an der
\ac{LMU} konnten auch diverse Pythonumgebungen zum Verarbeiten von Daten genutzt
werden. Dies war in erster Linie für ein Nachvollziehen und Verstehen der anatomischen
Segmentierung hilfreich.

Neben der eigentlichen Umgebung von Slicer und den Entwicklerwerkzeugen steht zur
Entwicklung auch ein Python Paket zur Verfügung, das von Herrn Prof. Rösch speziell
für die Klinik an der \ac{LMU} erstellt wurde. Dieses Tool beinhaltet diverse Funktionalität
für das Verarbeiten von medizinischen Bilddaten. Speziell für die Mikro-\ac{CT}-Aufnahmen
der Klinik.

Nachdem die Entwicklungsumgebung definiert und eingerichtet wurde, kann nun auf
die konkrete Methodik eingegangen werden, die zur Umsetzung dieser Schnittstelle
verwendet wird. Dabei werden sowohl die Anforderungen an die Erweiterung
beschrieben, also auch die konzeptionellen Überlegungen detailliert beschrieben.

% ---------------------------------------------------------------------------------------
	\chapter{Methodik}
\label{chap:methodik} Dieses Kapitel beschreibt das methodische Vorgehen, das
zur Beantwortung der Forschungsfrage gewählt wurde, um aussagekräftige und reproduzierbare
Ergebnisse zu erzielen. Eine nachvollziehbare Methodik ist essenziell, um die Ergebnisse
sowohl evaluierbar als auch für zukünftige Arbeiten nutzbar zu machen. Das
Hauptziel dieser Arbeit ist die Entwicklung einer stabilen und voll
funktionsfähigen Erweiterung für die Software 3D Slicer, die in der Klinik eingesetzt
werden kann. Zu Beginn wurde demnach eine umfassende Anforderungsanalyse
durchgeführt, um die spezifischen Anforderungen der Domäne zu erfassen und die Ausgangssituation
zu klären. Darauf aufbauend folgte eine detaillierte Literaturrecherche, um den aktuellen
Stand der Technik zu untersuchen und bestehende Lösungen zu identifizieren. Da
das Ziel dieser Arbeit die Entwicklung einer vollständigen Softwarelösung ist, wurde
das Problem anschließend in Teilaufgaben zerlegt. Dies ermöglicht eine gezielte
Bearbeitung einzelner Komponenten und erleichtert die iterative Entwicklung. Falls
für bestimmte Teilbereiche keine passenden Lösungsansätze aus der Literatur ableitbar
waren, wurden darauf basierend eigene methodische Ansätze erarbeitet.

Neben der praktischen Anwendung der entwickelten Erweiterung bietet diese Arbeit
auch wissenschaftlichen Mehrwert. Daher wird im folgenden Abschnitt die gewählte
Methodik detailliert begründet und deren Vorteile herausgearbeitet.
% ---------------------------------------------------------------------------------------

\section{Forschungsdesign}
Das Forschungsdesign dieser Arbeit folgt einem praktischen Entwicklungsansatz mit
einem Fokus auf softwaretechnische Methoden. Zum Erreichen der Ziele stützt sich
diese Arbeit so am Entwicklungsprozess und dokumentiert diesen. Dabei lässt sich
der gesamte Zeitraum dieser Arbeit in drei Phasen aufteilen, die jeweils einem
unterschiedlichen Zweck diene. Diese drei Phasen sollen auch eine grobe Orientierung
bezüglich der Reihenfolge während der Bearbeitung geben.

\pagebreak

\begin{description}
	\item[\textbf{Analysephase}] Diese erste Phase ist bei fast allen Softwareprojekten
		die wichtigste Phase und gleichzeitig aber die, die meist zu kurz kommt. Innerhalb
		der Analysephase werden also alle Anforderungen an die Software gesammelt.
		Diese basieren zum großen Teil auf der Literaturrecherche. Außerdem werden bestehende
		Lösungen analysiert und so die Kernfunktionalität herausgefiltert.

	\item[\textbf{Entwicklungsphase}] Die Entwicklungsphase bildet den größten Teil.
		Hier findet die konkrete Umsetzung statt. Hierzu wird das System in mehrere Subsysteme
		unterteilt. Dies ermöglicht eine isolierte Betrachtung. Während der Entwicklung
		wird ein Phototypenansatz verfolgt.

	\item[\textbf{Evaluationsphase}] Die letzte Phase dieser Arbeit beschäftigt sich
		ausschließlich mit der Evaluation der Ergebnisse. Hier soll eine Antwort auf
		die in \ref{chap:fragestellung} formulierten Fragestellungen gefunden werden.
\end{description}

Durch diese Unterteilung ist eine gutes strukturelles vorgehen Möglich um mittels
einer praktischen Umsetzungsmethodik zu einem guten Ergebnis zu kommen. Die
nächsten Kapitel blicken nun in die einzelnen Phasen, beginnend mit einer Anforderungsanalyse.
% ---------------------------------------------------------------------------------------

\section{Anforderungsanalyse}
\label{sec:anforderungsanalyse} Nach genauerem Betrachten der Fragestellung aus Kapitel
\ref{chap:fragestellung} und den Zielen aus \ref{sec:ziel_der_arbeit} können bereits
einige Anforderungen abgeleitet werden, die für die Erweiterung gelten sollen. Neben
diesen Anforderungen wurden auch die Klinik für Zahnerhaltung mit in diesen Prozess
eingebunden. Hierzu wurde innerhalb eines Meetings mit dem verantwortlichen Arzt,
Dr. Elias Walter, ein Anforderungskatalog ausgearbeitet \citep[vgl.][]{walter2025}.
Diese Anforderungen waren vor allem zu Beginn der Entwicklung sehr wichtig um einen
ersten Anhaltspunkt zu gewinnen. Im Laufe des Entwicklungsprozesses wurden
Statusberichte eingeplant, die ein Reagieren auf Anforderungsänderungen ermöglichen
sollen.

In erster Linie wird klar, dass im Rahme dieser vorliegenden Arbeit eine
Extension für die Plattform 3D Slicer entwickelt werden soll. Die Kernfunktionalität
soll dabei die anatomische Segmentierung bilden, wie sie in Kapitel
\ref{sec:verwwandte_arbeit} beschrieben wurde. Greift man das Ziel dieser Arbeit
aus der Einleitung \ref{sec:ziel_der_arbeit} nochmals auf, dann kann hierdurch
die nächste wichtige Anforderung abgeleitet werden. Die Erweiterung soll gut und
einfach über ein User Interface (UI) bedient werden können. Außerdem ist eine
stabile Anwendung gefragt, die sich gut in die Kernanwendung von 3D Slicer einfügt.
\citet[]{walter2025} machte im Interview deutlich, das ie Extension neben einer
Einzelbildbearbeitung auch einen Batch-Prozess ermöglichen so. So können Beispielsweise
Parameter an einem Bild erprobt werden und diese anschließend in einen Batch-Prozess
für viele Bilder überführt werden. Außerdem soll es möglich sein, verschiedenen Schwellwertverfahren,
die in der anatomischen Segmentierung vorgesehen sind, auch in der Extention auszuwählen.
Ein wichtiger Softwaretechnischer Anspruch an die Extension ist die
Erweiterbarkeit. Es soll ohne große Mühen möglich sein, ein weiteres Verfahren
zu integrieren, ohne das große Anpassungen an der UI oder der Erweiterung selbst,
unternommen werden müssen. Für ein solides Verständnis dieser Software soll es selbstverständlich
eine Dokumentation mit Benutzerhandbuch geben. Zudem wird großer Wert auf die
Qualitätssicherung gelegt, weshalb eine Reihe von Unit-Tests (Tests für einzelne
Programmeinheiten) vorgesehen ist. Um die Anforderungen an die Software besser
zu verstehen und zu strukturieren, ist neben der Sammlung technischer Spezifikationen
auch ein solides Verständnis für die zugrunde liegende Domäne essenziell. Die
Abbildung \ref{fig:3d_slicer_domäne} veranschaulicht dies durch ein UML-Domänenmodell
(Unified Modeling Language), das einen visuellen Überblick über die verschiedenen
Teile der Software bietet. Dazu sind auch einige der Anforderungen erkennbar
\citep[vgl.][]{walter2025}.

\begin{figure}[h]
	\centering
	\includegraphics[width=0.75\textwidth]{img/domaenenmodell.jpg}
	\caption{UML-Domänenmodell des gesamten Softwaresystems}
	\label{fig:3d_slicer_domäne}
\end{figure}

Durch diese breite Palette an Anforderungen ergeben sich verschiedene Aufgaben
für die Implementierung. Bevor jedoch mit der konkreten Umsetzung begonnen werden
kann, ist ein noch wichtigerer Schritt erforderlich: die Recherche. Sie dient
dazu, den aktuellen Stand der Technik zu erfassen und geeignete Lösungsansätze zu
identifizieren.
% ---------------------------------------------------------------------------------------

\section{Recherche zum Stand der Kunst}
Es wäre äußerst ungünstig, erst am Ende eines Projekts festzustellen, dass bereits
veröffentlichte Lösungen existieren, in die erhebliche Ressourcen investiert
wurden. Um dies zu vermeiden, ist eine umfassende Literaturrecherche essenziell,
die den aktuellen Stand der Technik abbildet. Dabei wird auf Fachliteratur sowie
domänenspezifische Quellen zurückgegriffen, um alle relevanten Aspekte abzudecken.

Für diese Arbeit spielt eine Quelle eine besonders wichtige Rolle: die
offizielle Dokumentation von \citet{slicer2024}. Sie bietet wertvolle
Anhaltspunkte für die Implementierung und hilft dabei, die technischen Gegebenheiten
von 3D Slicer zu verstehen. Zudem enthält sie Best-Practice-Ansätze, die bei der
Entwicklung berücksichtigt wurden. 3D Slicer stellt außerdem einen Developer
Guide zur Verfügung, der Teil der offiziellen Dokumentation ist und den Einstieg
in das Framework erleichtert. Ein weiterer zentraler Referenzpunkt ist der 3D
Slicer Extension Index, in dem bereits entwickelte Erweiterungen einsehbar sind.
Ein konkretes Beispiel ist das Modul \textit{Airway Segmentation}, dessen
Analyse dazu beiträgt, bewährte Konventionen für die Entwicklung der eigenen Erweiterung
abzuleiten.

Neben einer konkreten Implementierungshilfe dient die Literaturrecherche auch dazu,
ein fundiertes Verständnis für die Domäne der medizinischen Bildverarbeitung und
deren zugrunde liegende Strukturen zu entwickeln. Mithilfe verschiedener domänenspezifischer
Publikationen kann ein tieferes Wissen über diesen Fachbereich gewonnen werden.
Besonders relevant sind hierbei die verschiedenen Verfahren für die Verarbeitung
der Micro CT Aufnahmen. Konkret handelt es sich hier um die unterschiedlichen Algorithmen
zur Filterung und Segmentierung von Micro CT Bildern in der Zahnmedizin.

Darüber hinaus ermöglicht die Recherche einen Blick auf alternative Plattformen zur
Bildverarbeitung, wie beispielsweise die weit verbreitete Software ITK-SNAP. Ein
kurzer Vergleich ergab jedoch, dass diese Lösung aufgrund ihrer Struktur in
diesem speziellen Fall nicht mit 3D Slicer konkurrieren kann.

Die Recherche bietet somit einen ersten fundierten Überblick über mögliche
Lösungen für die einzelnen Anforderungen. Um nun detaillierter auf die Umsetzung
einzugehen, nimmt das nächste Kapitel eine Unterteilung der Gesamtheit der
Anforderungen in kleinere Teilsysteme vor.
% ---------------------------------------------------------------------------------------

\section{Zerlegung in Teilprobleme}
\label{sec_zerlegung_in_teilprobleme} Durch die Aufteilung des Gesamtsystems in
mehrere kleine Teilaufgaben wird die Software für den Entwicklungsprozess
übersichtlicher. Die einzelnen Domänen können so schneller und besser verstanden
werden. Es gibt viele Möglichkeiten ein Softwaresystem in kleine Teile
aufzuteilen, sodass es am Ende auf den konkreten Anwendungsfall ankommt. Diese
Arbeit sieht folgenden Teilaufgaben für das Gesamtsystem vor:

\begin{description}
	\item[\textbf{Architekturdesign:}] Mithilfe von UML Diagrammen soll die Architektur
		dieses Systems abgebildet werden und sukzessive immer detaillierter beschrieben
		werden. Es soll dann verglichen werden, welche Entwurfsmuster für dieses System
		infrage kommen. Durch die Bearbeitung dieses Teilproblems kann die
		Anforderung an eine flexible Architektur erfüllt werden. Anschließend kann mit
		der Entwicklung des UI-Designs begonnen werden.

	\item[\textbf{UI Design:}] Es soll ein Design erstellt werden, dass sich an
		erfolgreichen und etablierten 3D Slicer Extensions orientiert. Jedoch sollen
		die Wünsche des Endnutzers auch nicht zu kurz kommen. Für eine
		Visualisierung des Designs bedient sich diese Arbeit der Wireframes.

	\item[\textbf{Pseudo-Extension:}] Befor der tatsächliche Algorithmus eingebunden
		werden kann, ist es wichtig eine funktionierende Erweiterung zu haben, die noch
		keine konkrete Aufgabe hat, aber funktioniert und in Slicer eingebunden
		werden kann.

	\item[\textbf{Kappselung Hoffmann:}] Nachdem die leere Extension lauffähig ist
		und auch einige Hilfsfunktionen bereitstehen, kann mit der Paketerstellung
		des Hoffmann begonnen werden. Hier soll das Verfahren von einem Python Notebook
		in eine Bibliothek überführt werden, sodass dieses Verfahren in der
		Extention ausführbar ist. Die konkrete Art des Paketes ist noch nicht festgelegt.

	\item[\textbf{Parameter:}] Der Benutzer steuert das Verfahren über die Parameter
		in der UI. Für die Speicherung der Parametereinstellungen hat Slicer den
		Mechanismus ParameterNode entworfen. Diese wurde bereits in Abschnitt \ref{subsec:benutzerschnitstelle}
		erwähnt. Dieser Mechanismus ist nicht trivial, erhöht die Benutzerfreundlichkeit
		des Systems aber erheblich und soll demnach auch in diese Extention Anwendung
		finden.

	\item[\textbf{Single Prozess:}] Sobald alle notwendigen Vorbereitungen getroffen
		sind, kann der Algorithmus nun eingebettet werden. Hierzu betrachtet man isoliert
		den Single Prozess. Auch die UI wird erst nur so weit entwickel, wie es für den
		einfachen Prozess nötig ist. Hierbei wird auf das erstellte Paket für das
		Hoffmann Verfahren und die zuvor erstellen Hilfsfunktionen zurückgegriffen.

	\item[\textbf{Batch Prozess:}] Ist das einfache Verfahren fertig implementiert
		und funktioniert, so kann der Batch Prozess hinzukommen. Hier bedarf es einer
		zusätzlichen Arbeit in der UI, da der Benutzer über das Verwenden dieser
		Funktion gewarnt werden muss. Der Batch Prozess bedarf nämlich erheblicher Ressourcen.
		Hinzukommt die Implementierung einer Fortschrittsanzeige, sodass zu erkennen
		ist, dass ein Hintergrundprozess läuft.

	\item[\textbf{Dokumentation und Benutzerhandbuch:}] Abschließend ist eine ausführliche
		Dokumentation der Architektur erwünscht, sodass zukünftige Entwickler wissen,
		wo sie ansetzten müssen. Hinzu kommt ein Benutzerhandbuch für eine
		Verwendung der Erweiterung. Das Benutzerhandbuch und die Architekturdokumentation
		erfolgen in einer README.md innerhalb der Extension.

	\item[\textbf{Tests:}] An letzter Stelle sollen noch Softwaretests implementiert
		werde, um die Richtigkeit der Extension sicherzustellen. 3D Slicer sieht hier
		Unittests vor, die über den Developer Modus in Slicer direkt in der
		jeweiligen Extension ausgeführt werden können.
\end{description}

Die Ordnung dieser Punkte gibt eine grobe Orientierung bezüglich der Reihenfolge
während der Umsetzung an. Damit eine Umsetzung überhaupt realisiert werden kann,
sind unterschiedliche Werkzeuge und Mittel notwendig. Diese sollen im nächsten
Kapitel kurz erläutert werden.
% ---------------------------------------------------------------------------------------

\section{Entwicklungsumgebung}
Da bereits ein Framework feststeht, mit dem gearbeitet werden soll, ist keine
weitere Forschung nötig, um die richtige Programmiersprache auszuwählen. Jedoch
gibt es eine kleine Auswahl zu treffen. 3D Slicer unterscheidet zwischen zwei Arten
von Modulen, die CLI-Module (Commend Line Interface), welche in der Sprache C++ geschrieben
werden und die Scripted Moduls, die eine Python Implementierung verlangen. Da die
anatomische Segmentierung ohnehin in einem Python Notebook bereitliegt, fiel die
Wahl hier auf die Scripted Moduls. So kann auch die breite Palette der Python Pakete
genutzte werden. Für eine detalierte Beschreibung des Frameworks selber sei an
dieser Stelle auf das Kapitel \ref{sec:bildbearbeitung} verwiesen, indem das Framework
und alle zugehörigen Eigenheiten noch genauer beschrieben wurden. Um den
Entwicklungsprozess etwas zu vereinfachen, wurde während der Entwicklung auf ein
Modul von Slicer zurückgegriffen, das speziell für Entwickler entworfen wurde. Die
Abbildung \ref{fig:entwicklungsumgebung} verdeutlicht dieses Tool.

\begin{figure}[h]
	\centering
	\includegraphics[width=0.6\textwidth]{img/Entwicklungsumgebung.png}
	\caption{Umgenbung während der Entwicklung mit 3D Slicer und PyCharm}
	\label{fig:entwicklungsumgebung}
\end{figure}

Mit den Debugging Tools lässt sich eine gewohnte Umgebung reproduzieren, in der der
Quellcode Schritt für Schritt analysiert werden kann. Speziell bei der
Fehlersuche ist dieses Toll eine sehr gute Unterstützung. Die Abbildung beschreibt
weiter, das als Umgebung für das Erstellen des Programmcodes die Software
Pycharm verwendet wird. Pycharm ist eine Lösung der Firma Jetbrains, für das
Erstellen von Python Quellcode. Dieses Tool bietet eine breite Palette an Funktionalitäten,
die das Erstellen von Software vereinfachen und kann als \textit{State of the
Art} bezeichnet werden.

Neben der eigentlichen Umgebung und den Entwicklerwerkzeugen steht zur Entwicklung
auch ein bereits erstelltes Python Paket zur Verfügung, das von Herrn Prof. Rösch
speziell für die Klinik für Zahnerhaltung an der LMU in München erstellt wurde.
Dieses Tool beinhaltet diverse Funktionalität für das Verarbeiten von
medizinischen Bilddaten. Speziell für die Micro CT Aufnahmen der Klinik.

Nachdem die Anforderungen, die Recherche, die konkreten Aufgaben und die verfügbaren
Werkzeuge erläutert wurden, bleibt noch die Evaluation der Arbeite. Das Kapitel Forschungsevaluation
erläutert die Methodik, mit dem das Erreichen des Forschungsziels messbar
gemacht werden kann.

% ---------------------------------------------------------------------------------------

\section{Forschungsevaluation}
Die Evaluation kann grob in zwei Teile unterteilt werden. Der erste Teil ist der
wohl wichtigste und beschäftigt sich mit dem Testen der Anwendung durch die
Benutzer.
% ---------------------------------------------------------------------------------------
	\chapter{Ergebnisse}
\label{chap:ergebnisse} test
	\chapter{Evaluierung}
\label{subsec:evaluierung} Neben der Implementierung der Features wurden auch diverse
Tests durchgeführt, die eine Bewertung der Ergebnisse möglich machen. Hierzu wurden
Softwaretests, Benutzertests und Messungen durchgeführt. Dieses Kapitel beschäftigt
sich mit den verschiedenen Tests und deren Auswertung. Begonnen wird mit einer
Analyse der Softwaretests, gefolgt von Laufzeitmessungen. Abschließend werden
die Anwendungsfälle des Tooth Analyser näher beleuchtet und auf die
Limitierungen der Anwendung eingegangen.

\section{Softwaretests}
\label{subsec:softwaretests} Betrachtet man die Dokumentation von Slicer genauer,
so fällt auf, dass dies eine recht starre Struktur für die Implementierung von
Tests vorgibt. Dabei ist jedoch nicht festgelegt, welche Art von Tests verwendet
werden soll. Im Rahmen des Tooth Analyser wurden hier ausschließlich Unittest implementiert,
welche die einzelnen Einheiten und Funktionen im Tooth Analyser abdecken. Der grobe
Testaufbau sei hier gezeigt.

\begin{lstlisting}[
    language={python},
    caption={Ausschnitt der Testklasse zum ausführen der Unittests},
    label={lst:tests}]
class ToothAnalyserTest(ScriptedLoadableModuleTest):
    def setUp(self):
	    slicer.mrmlScene.Clear()
	    self.loadSampleData()

    def runTest(self):
	    self.setUp()
	    self.testIsSmoothed()
	    # add more tests here...

    def testIsSmoothed(self):
	    from ToothAnalyserLib.AnatomicalSegmentation.Segmentation import isSmoothed
	    sampleDate = self.getSampleDataAsITK()
 	    result = isSmoothed(sampleDate)
	    self.assertFalse(result)
	    self.delayDisplay("Test 1 passed")
\end{lstlisting}

Wie gleich zu erkennen ist, wurden alle Softwaretests in der Klasse \texttt{ToothAnalyserTest}
gekapselt. Diese ist wie auch bei einigen anderen Klassen eine generalisierte Klasse
der \texttt{ScriptedLoadableModuleTest}. Der grundsätzliche Aufbau der Testklasse
ist simpel gehalten. Es gibt eine Methode \texttt{setup()} in der die Testumgebung
bereitgestellt wird und eine Methode \texttt{runTest()} in der die einzelnen
Testfälle ausgeführt werden.

Betrachtet man die konkrete Testmethode \texttt{testIsSmoothed()} genauer, so
fällt die Methode \texttt{getSampleDataAsITK()} auf, die hier kurz thematisiert
werden soll. Viele der geschriebenen Methoden und Funktionen benötigen für einen
guten Test ein konkretes Bild. Hierfür stellt der Tooth Analyser Beispielbilder
zur Verfügung, mit denen die Tests ausgeführt werden können. Da diese Bilder mit
ca. 500 \ac{MB} eine ausgeprägte Größe haben, wurden diese in einem separaten Repository
bereitgestellt. So müssen nicht erst einige \ac{GB} an Bildern heruntergeladen werden,
wenn das Modul installiert werden soll. Die Bilder werden erst dann heruntergeladen,
wenn sie benötigt werden. Um dieses Herunterladen zu ermöglichen, werden die
Bilder beim Starten des Moduls erstmals in Slicer registriert, sodass sie dann im
Modul \texttt{sampleData} zur Verfügung stehen. Damit ist nicht nur gewährleistet,
dass zukünftige Entwickler Tests ausführen könne, es können so auch Benutzern Beispielbilder
bereitgestellt werden, um erste Erfahrungen mit dem Tool zu machen. Die Abbildung
\ref{fig:sample_data} zeigt das Modul \texttt{SampleData} mit besonderem
Augenmerk auf das Bild \textit{ToothCT}.

\begin{figure}[h]
	\centering
	\includegraphics[width=1\textwidth]{img/sampleData.png}
	\caption{Ausschnitt des Moduls SampleData in 3D Slicer mit dem Beispielbild
	für das Starten eines Verfahrens im Tooth Analyser}
	\label{fig:sample_data}
\end{figure}

Ein Testfall der vielen soll hier Beispielhaft genauer betrachtet werden. Hierbei
geht es um den Test der Funktion \texttt{smoothImage()}. Diese Nimmt ein Bild und
führt eine Glättung durch. Um solch eine Funktion zu testen, bedarf es etwas mehr
als ein simplen Unittest, jedoch liefert der fertige Test eine gute Lösung um
den gesamten Umfang der Methode zu Testen. Vergleicht man ein verrauschtes Bild mit
einem geglätteten, dann unterschieden sie sich bis auf die visuelle Darstellung
auch in der Streuung der Pixelwerte. So kann mittels der Standardabweichung kontrolliert
werden, ob das Bild nach einer Filterung eine kleinere Standardabweichung hat
als vor der Filterung. Ist dies der Fall, so kann von einer erfolgreichen
Filterung ausgegangen werden. Zu Beachten ist an dieser Stelle, das eine
Filterung unter Umständen einige Zeit in Anspruch nehmen kann. Es sei gesagt,
dass diese bei einer Testausführung zu berücksichtigen ist. Eine gute Option dem
entgegenzuwirken ist es, das bereits gefilterte Bild mit in den Testdaten zur
Verfügung zu stellen. Die konkrete Implementierung eines solchen Tests liefert der
Quellcode \ref{lst:test_smooth_image}.

\begin{lstlisting}[
    language={python},
    caption={Implementierung eines Tests zum überprüfen einer Funktion},
    label={lst:test_smooth_image}]
def testSmoothImage(self):
    from ToothAnalyserLib.AnatomicalSegmentation.Segmentation import smoothImage
   
    data = self.getSampleDataAsITK()
    dataFiltered = smoothImage(data)
    dataStdDev = np.std(sitk.GetArrayFromImage(data))
    dataFilteredStdDev = np.std(sitk.GetArrayFromImage(dataFiltered))
    self.assertTrue(dataFilteredStdDev < dataStdDev)
    self.delayDisplay("Test 1 passed")
\end{lstlisting}

Im ersten Schritt wird ein Beispielbild geladen und in ein \ac{ITK} Format umgewandelt.
Anschließend folgt die Glättung des Bildes. Ist diese Glättung fertig, so können
die Streuungen der beiden Bilder verglichen werden.

Abschließend lässt sich über die Softwaretests sagen, dass einige Testfälle abgedeckt
wurden. Jedoch wurden nicht alle Methoden und Funktionen getestet. Viele bilden eine
sehr konkrete Lösung, die nicht einfach zu testen ist und deshalb einiges an
Entwicklungszeit beanspruchen. Gerade die Funktionen der Pipeline. Aus diesem
Grund konzentriert sich diese Arbeit nicht auf eine 100 prozentige Testabdeckung,
sondern soll noch weitere Metriken bieten. Eine ebenso gute Aussage lässt sich
anhand der Laufzeit des Moduls treffen. Hierzu wurde die Performance des Tooth Analyser
genauer unter die Lupe genommen und in verschiedenen Szenarien gemessen.

\pagebreak
% ---------------------------------------------------------------------------------------

\section{Performance}
Die Performance des Systems war nie ein wichtiges Kriterium und stand deshalb zu
keiner Zeit im Fokus dieser Arbeit. Dennoch ergaben sich interessante Ergebnisse,
die hier kurz erläutert werden sollen. Unter der Performance versteht dieses Kapitel
das konkrete Laufzeitverhalten der Anwendung also jene Zeit die zwischen Start und
Ende vergeht. Grundsätzlich lässt sich dazu sagen, dass die Laufzeit bei der Bearbeitung
von 3D Mikro-\ac{CT}-Aufnahmen sehr stark vom Typ des Bildes abhängt. So kommt es
beispielsweise darauf an, wie groß das Bild ist, oder ob es bereits eine Filterung
erfahren hat. Bei der Verarbeitung der Mikro-\ac{CT}-Bilder aus dem Klinikum für
Zahnerhaltung wurde eine konkrete Messung durchgeführt, die hier in Abbildung
\ref{fig:laufzeit} gezeigt wird.

\textbf{Vorbedingungen:}
\begin{itemize}
	\item 16 bit sigend integer

	\item Type .ISQ

	\item Mit Filterung

	\item Mit Berechnung der medial Flächen
\end{itemize}

\begin{figure}[h]
	\centering
	\includegraphics[width=0.8\textwidth]{img/laufzeit_diagramm.png}
	\caption{Verteilung der Laufzeit über den gesamten Bearbeitungszeitraum. 100\%
	entsprechen 16:27 Minuten}
	\label{fig:laufzeit}
\end{figure}

Zu sehen ist, dass unter diesen Bedingungen die Bearbeitung eines einzelnen
Bildes ca. 17 Minuten beansprucht. Dabei fallen über dreiviertel der Zeit auf
die Filterung zurück. Der Bereich Auffüllung beinhaltet ebenfalls eine Filterung
der einzelnen Segmente, weswegen dieser den zweitgrößten Teil ausmacht. Ein Weiterer
wesentlichen Teil stellen die beiden medialen Flächen dar. Um dieser doch
enormen Laufzeit etwas entgegen zu Wirken wurden zwei Mechanismen implementiert.
Das Verfahren kann einerseits erkennen, ob ein Bild bereits gefiltert wurde und
andererseits die medialen Flächen optional berechnen. So kommt es, dass sich ein
\textit{Best Case} ergibt der grob nur noch ein Viertel der Zeit benötigt.
Dieser \textit{Best Case} tritt ein, wenn ein Bild in den Algorithmus gegeben
wird, das bereits gefiltert wurde und keine medialen Flächen erfordert. Unter
Berücksichtigung von Abbildung \ref{fig:laufzeit} kann dann die Zeit für die
Filterung und die medialen Flächen abgezogen werden. So kommt es, dass das
Verfahren nur noch 16 Prozent der ursprünglichen Zeit benötigt.

Überträgt man das Laufzeitverhalten eines einzelnen Bildes auf die Bearbeitung
der Bilder in einem Batch-Prozess so lässt sich die Laufzeit mit der Anzahl der
zu bearbeitenden Bilder steigend linear ausdrücken. Das Diagramm aus \ref{fig:laufzeit_batch}
zeigt dies.

\begin{figure}[h]
	\centering
	\includegraphics[width=1\textwidth]{img/runtimeBatch.png}
	\caption{Konstruktion des Laufzeitverhaltens bei einer Bearbeitung mehrere
	Bilder}
	\label{fig:laufzeit_batch}
\end{figure}

Wie zu sehen ist, ist die Laufzeit eines Batch Prozesses maßgeblich davon geprägt,
wie lange ein einzelnes Bild benötigt. Dies entspricht dem Y-Achsenabschnitt im Diagramm
\ref{fig:laufzeit_batch}. Ist die Zeit eines einzelnen Bildes bekannt, so lässt sich
gut prognostizieren, wie lange eine Verarbeitung von $n$ vielen Bildern dauert. Wie
bereits angedeutet, hängt die Zeit eines einzelnen Bilds von vielen Faktoren ab.
Nimmt man den Fall aus Abbildung \ref{fig:laufzeit}, würde die Bearbeitung von
zehn Bilder etwa 170 Minuten beanspruchen. Dies entspricht zwei Stunden und 50 Minuten.
% ---------------------------------------------------------------------------------------

\section{Benutzertests}
\label{sec:benutzertests}Um die entwickelte Software objektiv zu bewerten, wurden
gezielte Benutzertests mit Anwendern durchgeführt. Dabei nahmen drei Zahnärzte
der Poliklinik für Zahnerhaltung und Parodontologie an der LMU an der Testphase
teil, in der sie den Tooth Analyser in ihren Forschungsalltag integrierten. Über
einen Zeitraum von drei Wochen konnten sie die Software in realen
Anwendungsszenarien erproben.

Zusätzlich wurde ein kleiner Stresstest durchgeführt, bei dem 103 CT-Aufnahmen in
einem Batch-Prozess mit dem Tooth Analyser verarbeitet wurden. Die Berechnung
erfolgte auf einem leistungsstarken Server der LMU, der über ausreichend Rechenkapazitäten
verfügt. Die durchschnittliche Bearbeitungszeit pro Bild lag hier bei etwa neun
Minuten. Basierend auf der in Abbildung \ref{fig:laufzeit_batch} dargestellten Verteilung
ergab sich eine prognostizierte Gesamtverarbeitungszeit von 15 Stunden und 27 Minuten.
Ein Wert, der in der Praxis gut bestätigt wurde. Besonders erfreulich war, dass sämtliche
Bilder bis zu einem gewissen Grad an Karies erfolgreich anatomisch segmentiert werden
konnten. Die Tests in der Klinik zeigten zudem, dass auch die vollständige Segmentierung
ganzer Zähne mit komplexen Wurzeln möglich ist.

Die Ergebnisse der Benutzertests aus der Klink verdeutlichen, dass der Tooth
Analyser in der Praxis eine wertvolle Unterstützung bietet, und durch die Segmentierung
ganzer Zähnen den Anwendungskreis noch weiter spannt. Um ein besseres
Verständnis für die realen Einsatzmöglichkeiten der Software zu gewinnen, werden
im folgenden Kapitel verschiedene Anwendungsszenarien vorgestellt.
% ---------------------------------------------------------------------------------------

\section{Anwendungsszenarien}
In erster Linie bietet der Tooth Analyser eine Visualisierungshilfe, die für
Ärzte unterstützend wirken soll. Wie auch von Slicer empfohlen wird dieses Modul
von den Ärzten rein zur Forschung eingesetzt. Mit dem Tooth Analyser lässt sich
mittels einer Mikro-\ac{CT}-Aufnahme ein 3D Modell des Zahnes erstellen. Man
spricht hier von einer Rekonstruktion des Zahnes. Durch die Segmentierung erlaubt
dieser rekonstruierte Zahn auch eine Segmentbetrachtung von Dentin und Schmelz.
Die Abbildungen \ref{fig:3d_view}, \ref{fig:3d_view_dentin} und \ref{fig:3d_view_schmelz}
zeigt diese Rekonstruktion genauer.

\begin{figure}[h]
	\centering
	\begin{minipage}[b]{0.32\textwidth}
		\centering
		\includegraphics[width=\textwidth]{img/3dView.png}
		\caption{Rekonstruktion eines Zahns aus einer CT-Aufnahme}
		\label{fig:3d_view}
	\end{minipage}
	\hfill
	\begin{minipage}[b]{0.32\textwidth}
		\centering
		\includegraphics[width=\textwidth]{img/3dViewDentin.png}
		\caption{Dentinsegment eines rekonstruierten Zahns}
		\label{fig:3d_view_dentin}
	\end{minipage}
	\hfill
	\begin{minipage}[b]{0.32\textwidth}
		\centering
		\includegraphics[width=\textwidth]{img/3dViewEnamel.png}
		\caption{Schmelzsegment eines rekonstruierten Zahns}
		\label{fig:3d_view_schmelz}
	\end{minipage}
\end{figure}

Das Betrachten der einzelnen Segmente wie sie in Abbildung
\ref{fig:3d_view_dentin} gezeigt wird, erfolgt nicht in der Erweiterung Tooth Analyser.
Hierzu wird auf das Modul \textit{Data} verwiesen, das eine hierarchische
Darstellung aller Daten in der Szene liefert. Über die Sichtbarkeitseinstellungen
der einzelnen Datenelemente können dann die Segmente sichtbar oder unsichtbar geschaltet
werden.

Einen weiteren Fall, indem die Anwendung unterstützen kann, ist die Klassifizierung
von Karies. Die Abbildung \ref{fig:classification} zeigt diesen Fall. Hierfür sind
die medialen Flächen der einzelnen Segmente nötig. Diese sind im Bild als rote
und grüne Linie sichtbar, verteilen sich aber über das ganze 3D Bild, was daraus
eine Fläche macht. Legt man nun diese Flächen über das originale Bild, so lässt
sich mittels dieser Linie der Karies auf einem \ac{CT} gut klassifizieren. Diese
besagten Linien bilden dann die Grenzen. Ragt der Karies über diese mediale Fläche
hinaus hat er bereits einen sehr ausgeprägten Zustand und wird anders eingeordnet
als ein Karies, der die mediale Fläche noch nicht überschritten hat.

\begin{figure}[h]
	\centering
	\includegraphics[width=0.4 \textwidth]{img/classification.png}
	\caption{Klassifizierung von Karies mittels der medialen Flächen}
	\label{fig:classification}
\end{figure}

Die konkreten Anwendungsfälle zeigen, dass der Tooth Analyser duchaus Anwendung
im Forschungsaltag der Klinik finden. Dennoch traten während der Nutzung auch einige
Einschränkungen auf, die die Einsatzmöglichkeiten des Tooth Analysers begrenzen.
% ---------------------------------------------------------------------------------------

\section{Limitierungen}
\label{sec:limitierungen} Dieses Kapitel soll alle Punkte transparent aufdecken,
die im Modul Tooth Analyser noch Probleme machen, oder gar nicht erst umgesetzt
wurden. Der limitierende Faktor in der Erweiterung ist das eingeschränkte Format
der Bilder. Es können innerhalb dieser Erweiterung nur Bilder verarbeitet werden,
die das Format \ac{16Int} haben. Führt man dennoch die Segmentierung mit einem anderen
Format durch (z.B. \ac{8UInt}), so stellt man fest, dass der Algorithmus zwar ein
Ergebnis generiert, dieses aber nicht verwendbar ist. Die Abbildung \ref{fig:3d_error}
zeigt ein solches falsche Ergebnis. Zu sehen ist ein 3D Modell, das nur aus einem
Segment besteht. In diesem Fall wurde der gesamte Zahn mit Pulpa als Dentin
markiert.

\begin{figure}[h]
	\centering
	\includegraphics[width=0.5\textwidth]{img/3d_view_error.png}
	\caption{Fehlerhafte Segmentierung einer CT-Aufnahme im Format 8UInt}
	\label{fig:3d_error}
\end{figure}

Durch diese Limitierung ergibt sich eine weitere. Zu Beginn in der Analysephase
des Projektes, war eine Vorverarbeitung der Bilder vorgesehen, dass die
Komprimierung eines Bildes vornimmt und so die Bilder deutlich handlicher macht.
Das Verfahren hierzu wurde in Kapitel \ref{subsec:datensätze} erläutert. Da
dieses Verfahren allerdings einen Formatwechsel von \ac{16Int} nach \ac{8UInt}
bewirkt und diese Bilder nicht richtig segmentiert werden können, scheidet die Vorverarbeitung
von Bildern vorerst aus.

Eine weitere Limitierung der Software liegt im Batch Modus. Wird ein Batch Modus
ausgeführt, so werden im Anschluss die Bilder nicht in die Szene geladen. Dies
muss manuell über den Import des Kernsystems erfolgen. Soll für die unterschiedlichen
Bilder ein \ac{3D} Modell erzeugt werden müssen diese ebenfalls manuell als
Segmentierung geladen werden.

Zuletzt sei noch auf eine Limitierung hingewiesen, die der Erweiterbarkeit des Moduls
dient. Soll die Erweiterung um zusätzliche Funktionen erweiter werden, dann sind
kleine Änderungen in einer bestehenden Klasse notwendig. Konkret geht es hier um
die Klasse \texttt{ToothAnalyserWidget}. Hier muss je nach Ausprägung der \ac{UI}
der neuen Funktion, Methoden hinzugefügt werden.
% ---------------------------------------------------------------------------------------
	\chapter{Diskussion und Fazit}
\label{chap:diskussion} Am Ende dieser Arbeit lohnt es sich, erneut einen Blick
auf die in Kapitel \ref{sec:ziel_der_arbeit} formulierte Forschungsfrage zu
werfen und noch einige Aspekte hinzuzufügen:

\textit{Wie kann eine benutzerfreundliche Schnittstelle innerhalb 3D Slicer
entwickelt werden, die das Verfahren der anatomischen Segmentierung effizient integriert,
den Zugang für Anwender vereinfacht und zugleich eine flexible Erweiterbarkeit
für zukünftige Funktionalitäten gewährleistet?}

Die Ergebnisse dieser Arbeit zeigen, dass die Integration der anatomischen Segmentierung
erfolgreich in das Ökosystem von 3D Slicer eingebunden werden konnte. Das zuvor
aufwendig zu bedienende Verfahren wurde durch die entwickelte Benutzeroberfläche
erheblich vereinfacht. Anstelle einer manuellen Ausführung über das Terminal lässt
sich die Segmentierung nun mit wenigen Klicks starten, was die Anwendung
besonders für Praktiker in der Zahnmedizin zugänglicher macht.

Ein zentraler Aspekt der Forschungsfrage betraf die Effizienz der Integration. Hier
konnte gezeigt werden, dass das entwickelte Modul die gewünschten Ergebnisse
liefert, ohne dabei die Qualität der anatomischen Segmentierung zu beeinträchtigen.
Durch die enge Verzahnung mit 3D Slicer bleibt das Modul kompatibel mit
bestehenden Workflows, wodurch eine nahtlose Nutzung innerhalb der Klinik ermöglicht
wird. Mit dem Tooth Analyser entstand so eine Struktur die ein gutes Vorgehen
bei der Verarbietung von Mikro-CT-Aufnhamen beschreibt, das für die gesamte Zahnmedizin
eine Bereicherung bietet. Der Luxus einer interaktiven Schnittstelle ist hier
nur die Spize des Eisberges.

Auch die Erweiterbarkeit des Systems wurde berücksichtigt. Dank der modularen Architektur
kann das zugrunde liegende Segmentierungsverfahren ohne größere Anpassungen ausgetauscht
oder durch weitere Funktionalitäten ergänzt werden. Dies stellt sicher, dass das
System auch in Zukunft flexibel bleibt. Es wurde auch deutlich, dass eine zu große
Funktionsvielfalt zu einer Überladung des Moduls führen könnte. Hier bietet sich
eine mögliche Lösung in der Aufteilung in Submodule.

Während der Entwicklung erwiesen sich die umfangreiche Dokumentation und die
bestehende Infrastruktur von 3D Slicer als große Vorteile. Die Implementierung einfacher
Algorithmen gestaltete sich dadurch effizient, und erste funktionale Prototypen
konnten zügig erstellt werden. Auch zeigte sich, dass eine Veröffentlichung über
den \textit{Extension Manager} nicht zwingend erforderlich ist – für bestimmte Nutzergruppen
kann das Modul problemlos lokal eingebunden werden.

Dennoch gab es Herausforderungen. Nicht alle Prinzipien einer idealen
Softwarearchitektur konnten konsequent umgesetzt werden. Diese Entscheidungen wurden
jedoch bewusst getroffen, um die Entwicklung pragmatisch und
anwendungsorientiert zu gestalten. Auch eine Vorverarbeitung der Bilder konnte nicht
implementiert werden. Zwar wurde diese Funktionalität vorgesehen und muss nur
noch befüllt werden, aber ein konkrete Ausführung ist zu dieser Zeit nicht mögliche.

Auch die Anforderungen aus Kapitel \ref{sec:anforderungsanalyse} sollen hier noch
einmal von kritischer Seite betrachtet werden. Neben der Frage, ob alle Anforderungen
erfüllt wurden, stellt sich hier auch die Frage, ob die Anforderungen auch passend
gewählt wurden. Wie die ersten Anwendertests zeigten, liefert der Tooth Analyser
einen guten Mehrwert im Forschungsaltag, sodass keine der Anforderungen als
misslungen bezeichnet werden kann. Auch die Ergenisdarsstellung in der Slicer-Szene
erlang in der ersten Testreihe ein gutes Feedback.

Zusammenfassend lässt sich sagen, dass die in dieser Arbeit entwickelte Lösung
die Forschungsfrage in wesentlichen Punkten beantworten konnte: Die Segmentierung
wurde effizient in 3D Slicer integriert, die Benutzerfreundlichkeit erheblich
verbessert und eine Erweiterbarkeit der Software gewährleistet. Darüber hinaus konnte
ein Modell ausgearbeitet werden, nach dem diverse Mikro-CT-Aufnahmen effizient
bearbeitet werden können. Allerdings zeigt ein Blick auf die in Kapitel \ref{sec:relevanz_der_arbeit}
diskutierte Relevanz der Arbeit, dass der Tooth Analyser noch nicht den
endgültigen Reifegrad erreicht hat. Um das System weiter zu optimieren, sind
zusätzliche Benutzertests und potenzielle Fehlerbehebungen erforderlich. Erst mit
diesen weiteren Schritten kann das Ziel einer langfristig etablierten und klinisch
einsetzbaren Lösung vollständig erreicht werden.
% ---------------------------------------------------------------------------------------
	\chapter{Ausblick}
\label{chap:schlussfolgerung} Der Tooth Analyser bietet mit dem aktuellen Stand bereits
einen großen Mehrwert für die Ärzte in der Klinik für Zahnerhaltung. So lässt
sich sagen, dass diese vorliegende Arbeit durchaus von Erfolg gekrönt ist. Es
wird aber auch klar, dass noch viel Potenzial im Tooth Analyser steckt. Das
meiste Potenzial ist hierbei in den Limitierungen zu finden. Eine hervorragende Ergänzung
dieser Arbeit würde eine Vorverarbeitung der Bilder bieten. Es soll also in
Zukunft möglich sein, die \ac{CT}-Aufnahmen in einer Vorverarbeitung zu
komprimieren und sie dann mittels eines Verfahren zu bearbeiten. Hierbei muss sich
die Vorverarbeitung nicht auf die Komprimierung beschränken, wie die Diskussion
gezeigt hat. Dies schließt auch eine Verarbeitung aller Formate ein. Ein schöner
Nebeneffekt dieser Ergänzung ist, dass durch eine Komprimierung der Bilder die Laufzeit
sinkt. Der verlorene Detailgrad dieser Komprimierung ist dabei nicht störend. Somit
lässt sich sagen, dass durch das Akzeptieren unterschiedlicher Formate der
größte Mehrwert für den Tooth Analyser gewonnen werden kann. Des Weiteren lässt der
Bereich Analysen im Tooth Analyser noch viel Spielraum. Hier gibt es diverse
Möglichkeiten, das Modul noch mit weiteren Funktionen zu bestücken. Eine gute
Idee lieferte hier die Rückmeldung aus den Benutzertests. Diese ergaben, dass
eine Integration des Python-Pakets \textit{radiomics} den Tooth Analyser gut ergänzen
würde. Mit diesem Paket lassen sich Radiomics-Merkmale aus Bilder Extrahieren
und isoliert analysieren. Den letzten Ausblick, der hier gegeben werden soll, bezieht
sich wieder auf das Verfahren der anatomischen Segmentierung. In der aktuellen
Version nimmt das Verfahren nur eine Aufteilung in Schmelz und Dentin vor. Der Teil
der Pulpa wird dem Hintergrund zugeordnet und nicht betrachtet. Die zusätzliche
Segmentierung der Pulpa würde die Arbeit ebenfalls hervorragend ergänzen. Jedoch
sei auch gesagt, das dies die wohl herausforderndste Aufgabe ist. Die Pulpa hebt
sich nur sehr leicht aus dem Hintergrund hervor und ist demnach schwer zu segmentieren.

Ordnet man die hier genannten Punkte nach ihrer Wichtigkeit ein, so lässt sich
sagen, das durch das Erweitern des Moduls auf mehrere Bildformate der größte Mehrwert
für den Tooth Analyser gewonnen werden kann. Dies würde weitere Funktionalität
nach sich ziehen. Jedoch sei auch gesagt, dass die übrigen Punkte ebenfalls
einen großen Mehrwert für den Tooth Analyser und damit für die gesamte
Zahnmedizin bietet.
% ---------------------------------------------------------------------------------------

	% --------------------------------------------------
	% Bibliographie
	% --------------------------------------------------
	\renewcommand{\bibfont}{\footnotesize}
	\printbibliography
	[title={Literaturverzeichnis}, heading=bibintoc]

	% --------------------------------------------------
	% Anhang
	% --------------------------------------------------
	\appendix
	\chapter{Anhang}
\label{chap:anhang}

- hier all Datensätze eines Bildes, das erfolgreich Segmentiert wurde

- dann alle Datensätze eines Bildes, das NICHT erfolgreich Segmenteirt wurde
	\AuthorDeclaration % Selbständigkeitserklärung

	% --------------------------------------------------
	% Index
	% --------------------------------------------------
	{\setkomafont{section}{\Huge} % temporarily set chapter font
	\printindex }
\end{document}